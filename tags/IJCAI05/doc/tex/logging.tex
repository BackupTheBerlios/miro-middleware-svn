\chapter{Logging}
\label{sec:logging}

The logging of debug, information and error messages can help a great
deal in the recovery from failure situations. This is especially true
for robotic applications. Unfortunately the sheer mass of console
output from the different modules can become too complex quite fast. So
\miro comes with a set of logging facilities that cover different
levels of the system functionality. The first framework, that is
introduced within this chapter is based on the logging framework
provided by ACE. It is a system log oriented framework to organize printf
like output on various levels. Chapter \ref{sec:ecLogging} will cover
another, more sophisticated set of logging functionality.

\section{Log Levels and Categories}

\miro uses two mechanisms to customize logging output.  Log levels are
used to give fine grained control over the verbosity of log output.
\miro defines log levels from 0 to 9. The higher log levels are
considered to be used for debugging purposes only. They are therefore
also referred to as debug levels. They usually are only meaningful for
developers. For debug levels log categories 
are used to restrict the logging output to one or more subsystems of
the robot application. Output from debug levels is only displayed if
its log category is enabled, too.
Additional log categories can be defined for your own
programs.

The different log levels are:
\begin{enumerate}
\setcounter{enumi}{-1}
\item \emph{Emergency} Log level of messages reporting an emergency.
  Your robot is on fire etc. This log level is not maskable, except if
  you turn of logging at configure time.
\item \emph{Alert} Log level of messages reporting an alert. This is
  when the red lights start blinking and this unnerving honking sound
  is played.
\item \emph{Critical} Log level of messages reporting a critical
  condition. This usually is an unrecoverable error, that leads to
  the termination of reporting program.
\item \emph{Error} Log level of messages reporting an error.
  This indicates a real error, but the program will usually try to
  continue anyway.
\item \emph{Warning} Log level of messages reporting a warning.  A
  warning should be fixed, but the program is likely to work anyways.
\item \emph{Notice} Log level of messages reporting a notice. Make
  a post-it and add it to the other 500 ones.
\item \emph{Ctor\_Dtor} Debug level of messages reporting a
  constructor/destructor entry.  This debug level is designed to hunt
  segmentation faults on startup and exit. --- This is when all the big
  ctors/dtors are run.
\end{enumerate}

The debug levels are:
\begin{enumerate}
\setcounter{enumi}{6}
\item \emph{Debug} Log level of messages reporting debug output.
\item \emph{Trace} Log level of messages reporting program trace
  output.
\item \emph{Prattle} Log level of messages reporting really verbose
  comments on the program execution.
\end{enumerate}

The different log categories used within \miro are the
following. (All brand names have to be marked as such, as soon as we
find the time):
\begin{description}
\item[Miro] This category is used by the \miro core components,
  namely those located in the library libmiro.so
\item[Video] Used by the components of the video image processing
  system. See section \ref{sec:VIP} for details.
\item[NMC] The notify multicast based event channel deliberation, as
  used for team communication. See section \ref{sec:NMC} for details.
\item[B21] Used by the components exclusive to the B12 robots of RWI.
\item[Pioneer] Used by the components exclusive to the Pioneer robots
  or Active Media.
\item[Sparrows] The custom built Sparrows platform.
\item[Faul] The motor controllers from Faulhaber.
\item[DP] The pan-tilt unit from DirectPerception.
\item[Sick] The sick laser scanners have their own category.
\item[Sphinx] The sphinx speech components.
\item[Dtlk] The DoubleTalk speech components.
\item[BAP] The reactive control subsystem logs messages within this
  category. See section \ref{sec:BAP} for details.
\end{description}

\section{Run-Time Configurability}

The log level and log categories that are displayed can be configured
at program startup by command line parameters.

\begin{description}
\item[-MiroLogLevel $<$n$>$] or -MLL $<$n$>$ for short, selects the log level
  up to which data is logged. The log levels are referenced by number.
  The default log level is 4. Log level 0 can not be masked by command
  line parameters, but only by turning off logging entirely at compile
  time.
\item[-MiroLogFilter $<$name$>$] or -MLF $<$name$>$ for short, selects a
  category to log. This option can be specified multiple times to
  enable multiple categories. The category enabled by default is
  \miro.
\end{description}

\section{Usage in Source Code}

This logging facility is used extensively in the \miro sources. It can
also be used and extended for the usage in robotic projects based on
\miro.

\subsection{Miro::Log}

The class \texttt{Miro::Log} is defined in \texttt{miro/Log.h}. It
works mostly as a namespace for the logging facility. It holds
constants for all log levels as well as log categories. The main
method is \texttt{Miro::Log::init()}. This is a static method that
takes \textit{argc} and \textit{argv} as arguments, to parse for the
\texttt{-MiroLogLevel} and \texttt{-MiroLogFilter} command line
options. Additionally, accessor methods and predicates are defined to
query log levels etc.  at runtime.

\subsection{Macros}

\texttt{miro/Log.h} also defines a set of macros that are usable for
development with, and within \miro.
\begin{description}
\item[MIRO\_LOG] Produce some log output, if the required log
  level is enabled. It takes two arguments. The first is the log level,
  the second is a single character string. The log level is specified
  by name. The canonical name format for log levels is the log level
  name in capitals, prefixed by \texttt{LL\_}.
\item[MIRO\_LOG\_OSTR] Like \texttt{MIRO\_LOG}, but the second
  parameter is used as right hand side argument for an output stream
  operator $\ll$. So it can contain some operator $\ll$ concatenated
  expression.
\item[MIRO\_DBG] Produce some log output, if the required log level as
  well as the log category is enabled. It takes three arguments. The
  first is the log category, the second the level, and the third is a
  single character string. The canonical name format for log
  categories is the log category name in capitals.
\item[MIRO\_DEBUG\_OSTR] Like \texttt{MIRO\_DBG}, but the last
  parameter is used as right hand side argument for an output stream
  operator $\ll$. So it can contain some operator $\ll$ concatenated
  expression.
\item[MIRO\_LOG\_CTOR] For constructor tracing. Accepts a single
  parameter containing the class name as string. Logged with log level
  6.
\item[MIRO\_LOG\_DTOR] For destructor tracing. Accepts a single
  parameter containing the class name as string. Logged with log level
  6.
\item[MIRO\_DBG\_TRACE] For method call traces. Takes the log
  category as argument.
\item[MIRO\_ASSERT] The standard assert macro. It is provided to
  enable disabling of assert macros in (inline) code of \miro without
  disabling them for user code too.
\end{description}

\section{Compile-Time Configurability}

Log and debug messages can be disabled entirely at compile time by two
configure flags: 
\begin{description}
\item[--disable-DebugInfo] disables debug information, that is namely
  log levels above 6. The \texttt{MIRO\_DBG\_...} macros will be
  replaced by no-op implementations, removing footprint and performance
  overhead of the programs, introduced by debug information. -
  Especially the performance overhead is mostly negligible, so this
  option might well be omitted even for release versions.
\item[--disable-LogInfo] disables log information entirely. Like
  above, this can save some footprint but is hardly measurable in
  performance.
\end{description}

\section{Test and Example Programs}

The programs located at tests/log provide some testing facilities
for the logging facility and can serve as a practical example on the
setup of the logging facility as well as on the usage of
the logging and debug macros. The only program currently located there
is TestLog.cpp. The source code is documented to help understanding.

%%% Local Variables: 
%%% mode: latex
%%% TeX-master: "miro_manual"
%%% End: 
