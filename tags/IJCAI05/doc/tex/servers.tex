\chapter{Available Services}
\label{sec:services}

The \miro framework abstracts the robots hardware devices as active
services, that export the sensors and actuators functionality via CORBA
interfaces that can be accessed transparently from other programs,
probably running on totally different machines. Which services there
are available depends on the individual robot type. To provide easy
access to the services interface, each service registers itself under
a standard name at the CORBA Naming Service. To allow multiple
robots to be accessed at a time, each individual robot creates its
one naming context within the Naming Service under which its services
register. The default naming context is {\tt Miro}, but a given robot
should register itself under its own name (for example {\tt stanislav}).
The naming context to use can be specified at the
command line of every service and example program via the
{\tt -MiroNamingContext} option.


\section{Command Line Options}
\label{sec:command-line}

For each service the following command line options can be specified:

\begin{description}
\item[-MiroConfigFile {\em fileName}:]
  The services are widely configurable. These parameters are specified
  in config files within the directory {\tt \$MIRO\_ROOT/etc}. The
  file name is derived from the computers {\tt \$HOST} environment
  variable: {\tt \$HOST}.xml. If you want to specify another
  configuration file, you can do with this command line option.
\item[-MiroNamingContext {\em contextName}:]
  The short form of this option is \textbf{-MNC}.
  As mentioned before all services register their interfaces at the
  naming service within a naming context. The default naming context
  is {\tt Miro}. In multi robot scenarios for instance it would be
  necessary to assign a different naming context to each robot. This
  can be done by this command line option. Note that since it is
  necessary to specify none default naming context names also at the
  client side, this option is also recognized by every client program.
\item[-MiroRebindIOR:]
  To prevent services from different robots to accidentally
  deregister each other from the naming service, an already existing
  entry in the naming service isn't unbound by a service, if it tries
  to register under the same name. Anyhow, if you need to overwrite an
  existing entry, specify this option at the command line.
\end{description}


\section{Individual Interfaces}

In the following, we list the different services and their
interface names in the naming service, all interfaces are
registered within the specified naming context in the naming
service. For multi robot scenarios the canonical naming convention
for the individual robots naming contexts is the robots name.
Don't get confused by the fact that many services are collocated
within a single binary, while for others there exists a dedicated
binary. Note that in general the specialized interfaces of
individual robots register at the Naming Service under their
ancestors name.  The interface documentation can be looked up
within the \miro reference and in the \miro online documentation
as well.  For some services specializations for individual robot
types do exist. Those are described in the subsequent sections.

\begin{description}
\item[Odometry:]
  The motion service registers the {\tt Miro::Odometry} interface as
  {\tt Odometry}. It encapsulates
  the odometry (dead reckoning) sensor.
\item[Motion:]
  The motion service registers the {\tt Miro::Motion} interface as
  {\tt Motion}.
\item[RangeSensor:] The range sensor interface is the general
  abstraction of all range sensor devices, such as sonar, laser range
  finders etc. It offers a method for querying the sensors physical
  configuration as well as the latest sensor reading.  See the {\tt
    QtRangeSensor} program in the utils directory for an example on
  how to use this information.  The range sensor interface becomes
  registered at the naming service under the name of the actual
  sensory device.
  \begin{description}
  \item[Sonar:] The sonar service registers as {\tt Sonar}.  It
    provides an interface to the very common ultrasonic sensors for
    robots.
  \item[Infrared:] It registers as
    {\tt Infrared}.
  \item[Tactile:] Some robots have bumpers as some sort of ''it's too
    late'' sensors.  Still it is better to stop when you crashed than
    just moving on. It is registered as {\tt Tactile}.  Since it can
    be interpreted as a (very limited) range sensor, it is also a {\tt
      Miro::RangeSensor}.
  \item[Laser:] The laser scanner service registers as {\tt Laser}. It
    provides very accurate 180� distance measurements, with up to
    48Hz.
  \end{description}
\item[Stall:]
  Similar to the bumpers, the stall detection monitors the robots
  motion and detects, when the robot is stuck in its movement. The
  {\tt Miro::Stall} interface registers at the naming service as
  {\tt Stall}.
\item[Video:] The video service registers the {\tt Miro::Video}
  interface as {\tt Video}. It provides camera images at a rate of up
  to 25 images per second captured by frame grabber cards. Currently
  supported are {\tt matrox meteor} and cards that are supported
  by {\em video for linux} (bttv8).  Since the bandwidth needed for
  uncompressed image data transportation exceeds the capabilities of
  most todays network devices and even memory copying can introduce
  excessive overhead, the video service enables access
  to the grabbed images via shared memory keys.
\item[PanTilt:] Cameras often are mounted on top of a pan-tilt unit,
  allowing the robot to look sideways while moving in another
  direction. Lightweight versions of such a device are panning or
  tilting units.  The associated services register the
  {\tt Miro::Pan}, {\tt Miro::Tilt} and {\tt Miro::PanTilt}
  interfaces as {\tt Pan, Tilt} and {\tt PanTilt} at the naming
  service.
\item[Buttons:]
  For simple user interaction, some robot types provide push
  buttons. The buttons service registers the {\tt Miro::Buttons}
  interface as {\tt Buttons}.
\item[Speech:]
  To provide a more natural way of communication, some robots are
  equipped with speech synthesizer cards. The speech service registers
  the {\tt Miro::Speech} interface as {\tt Speech}.
\item[Gps:]
  Some robots may be equipped with a GPS receiver to allow easy and
  Odometry-independent positioning. The GPS service registers the
  {\tt Miro::Gps} interface as {\tt GPS}. Specialized services for
  particular GPS devices may also register an extended interface derived
  from {\tt Miro::Gps}, e.g. {\tt Miro::CtGps}.
\end{description}


\section{Asynchronous Sensory Information}
\label{SEC:ASYNCHRONOUS_SENSORY_INFORMATION}

Another feature of the \miro framework is the asynchronous
distribution of sensory data via the CORBA Notification Service
\cite{OMG:Notify-Spec}. These event channels allow filtered and
priority based event processing for time critical sensory information
distribution under high load.  Therefore in every robots naming
context exists a reference {\tt EventChannel} under which the
robots notification service can be accessed.

The usage of the notification service within the \miro framework is
explained in more detail in \ref{sec:notify}. In the following we give
only a brief description. The notification service allows simple event
filtering on the bases of the domain name and the type name of the
event. This way the individual consumer can easily subscribe for the
messages it is interested in. As the domain name, the naming context
is used within the \miro framework (note that in a multi robot
scenario, an event channel can transport data from multiple robots).
In the following the type names of the events and the data contained
in the event are listed for the individual services:

\begin{description}
\item[Odometry:] The odometry servant and all derived services
  generates periodical {\tt Miro::MotionStatusIDL} events), that
  propagate the robots current position and velocities. The event type
  name is {\tt Odometry}.
\item[RangeSensor:] The range sensor servant and all derived services
  can generate periodical range sensor events. The event type name is
  that of the derived sensor interface (Sonar, Laser etc.). It is also
  returned by the {\tt getScanDescription} method within the {\tt
    ScanDescriptionIDL} struct. The event data is one of three types,
  as also specified within {\tt Miro::SensorDescriptionIDL}.
  \begin{description}
  \item[Miro::RangeScanEventIDL]
    The data of a range sensor that acquires data in a continuous scan
    pushes its data within this struct. (Currently, there does exist
    no actual sensor that behaves in this way, but anyways...)
  \item[Miro::RangeGroupEventIDL]
    The data of a range sensor that acquires data groupwise
    pushes its data within this struct. A range sensor group
    normally is formed by a set of sensors that are mounted on the
    same hight, pointing in different directions. The laser scanners
    are organized that way.
  \item[Miro::RangeBunchEventIDL] The data of a range sensor that
    acquires data in a discontinuous fashion pushes its data within
    this struct. Each sensor reading contains its own group id as
    well as its index within that group. Especially sonar sensor
    are organized that way, since neighbouring sensor mustn't be
    fired simultaneously to minimize crosstalk.
  \end{description}
  The event type name is that of the actual sensor: {\tt Sonar}, {\tt
    Laser}, {\tt Tactile}, etc. as also specified within {\tt
    Miro::SensorDescriptionIDL}.

\item[Sonar:] The sonar service generates range sensor events with the
  type name {\tt Sonar}. Note that the sonar sensors are fired
  interleavingly to avoid crosstalk between them. Therefore the
  payload of an event emitted by a sonar device is normally a
  \texttt{RangeBunchEventIDL}. That way you have to analyze multiple
  {\tt Sonar} messages to get a full sonar scan.
\item[Infrared:] The infrared service generates range sensor events
  with the type name {\tt Infrared}.
\item[Tactile:] The tactile service generates range sensor events with
  the type name {\tt Tactile}. Due to the hopefully low frequency
  at which those events occur it might be better to listen to the
  {\tt Tactile} events, instead of polling the tactile status. Tactile
  devices usually contain a \texttt{RangeBunchEventIDL} struct.
\item[Stall:] The stall service also emits events. These contain the
  type name {\tt Stall}. For the event data see the individual
  robots descriptions.  Due to the hopefully rare occasions at which
  those events occur it might be better to listen to these events,
  instead of polling the stall interface.
\item[Buttons:] They also provide the event triggered communications
  model that uses the event type {\tt Button}. The events payload is
  of type {\tt ButtonStatusIDL}.
\item[Gps:] The GPS service sends out events for every position or
  dilution update received from the device. Position updates are denoted
  by the type name {\tt GpsPosition} and carry a {\tt
  GpsPositionEventIDL} payload containing updated information for the
  global position as well as a local position relative to a freely
  placeable reference point. Dilution events contain the type name {\tt
  GpsDilution} and carry a {\tt GpsDilutionEventIDL} type payload.\\
  Besides that the GPS service is also able to forward selected GPS
  sentences received from the device for evaluation by the client. If
  this feature is enabled the generated events will contain the {\tt
  GpsSentence} type name and carry a {\tt GpsSentenceEventIDL} type
  payload.
\end{description}


\section{\bXXI}

The \bXXI robot has a large number of available sensors and actuators. The
relevant binaries for its specialized services are:

\begin{description}
\item[B21Base:] This is a collection of the robots main services. This
  binary incorporates the motion service as well as the sonar, tactile
  and infrared services. Furthermore it gives access to the four
  colored buttons, that are mounted on top of the robot. The
  corresponding interfaces are:
  \begin{itemize}
  \item {\tt Miro::B21Motion} \\
    The {\tt Miro::B21Motion} interface is derived from the
    {\tt Miro::Motion} interface and gives access to the
    specialized movement commands of the B21 robot. It registers under it
    ancestors name {\tt Motion}.
  \item {\tt Miro::Sonar} \\
    The derived {\tt Miro::RangeSensor} interface registers itself
    as {\tt Sonar}.
  \item {\tt Miro::Infrared} \\
    The derived {\tt Miro::RangeSensor} interface registers itself
    as {\tt Infrared}.
  \item {\tt Miro::Tactile} \\
    The derived {\tt Miro::RangeSensor} interface registers itself
    as {\tt Tactile}.
  \item {\tt Miro::B21Buttons} \\
    The B21 has four buttons, that can be used as a simple user
    interface. The {\tt Miro::B21Buttons} interface is supported by
    this service. It sends {\tt Miro::ButtonStatusIDL} events, if a
    button is pressed or released.
  \end{itemize}

\item[SickLaserService:]
  This is the service to access the SICK laser scanner. It supports
  the {\tt Miro::RangeSensor} interface.
\item[DtlkSpeech:]
  This service lets you control the DoubleTalk Speech cards. The
  interface is named {\tt Miro::Speech}
\item[VideoService:]
  It gives you access to the two frame grabbers of the robot. It
  supports the {\tt Miro::Video} interface.
\item[DpPantilt:]
  It supports the {\tt Miro::DirectedPerceptionPanTilt} interface,
  which is derived from {\tt Miro::PanTilt}. It therefore registers
  itself as {\tt PanTilt}
\end{description}

Note that the B21 has two internal computers, to which the
different sensors and actuators are attached. You have to start
the relevant binaries at the correct computer. The
sensory/actuatory distribution in our laboratory is as follows:
Start B21Base and LaserService on the left computer, VideoService
and DpPantilt and DtlkSpeech on the right.


\section{\sparrow}

The \sparrow soccer robots have a similar diversity of sensors and
actuators, but for efficiency and internal design reasons, they are
grouped together in less binaries:

\begin{description}
\item[SparrowBase:]
  Within this file the following services are grouped together:
  Motion, Stall, Kicker, Buttons, Sonar, Infrared and Pantilt. The
  corresponding interfaces are:
  \begin{itemize}
  \item {\tt Miro::SparrowMotion} \\
    The {\tt Miro::SparrowMotion} interface is derived from the
    {\tt Miro::Motion} interface and gives access to the
    specialized movement commands of the \sparrow robot. It registers
    under it ancestors name {\tt Motion}.
  \item {\tt Miro::Stall} \\
    The {\tt Miro::Stall} interface is registered as {\tt Stall}.
  \item {\tt Miro::Kicker} \\
    Due to its purpose as football robot, the \sparrow robot has a
    kicker, that can be accessed via this interface. It registers at
    the naming service as {\tt Kicker}.
  \item {\tt Miro::Buttons} \\
    The {\tt Miro::Buttons} interface registers as
    {\tt Buttons}.
  \item {\tt Miro::Sonar} \\
    The {\tt Miro::RangeSensor} interface registers itself
    as {\tt Sonar}.
  \item {\tt Miro::Infrared} \\
    The {\tt Miro::RangeSensor} interface registers itself
    as {\tt Infrared}.
  \item {\tt Miro::PanTilt} \\
    The {\tt Miro::SparrowPanTilt} interface is registered as
    {\tt PanTilt}.
  \end{itemize}

  Note that the
  stall service sends {\tt Miro::SparrowStallIDL} events.
\item[VideoService:]
  It gives you access to the frame grabber of the robot. It supports
  the {\tt Miro::Video} interface.
\end{description}


\section{Pioneer}

Due to the limited variety of sensors and actuators on the Pioneer robots, there exists only
the most fundamental services for this robot.

\begin{description}
\item[PioneerBase:]
  The following services are grouped together: Motion, Stall and Sonar.
  The corresponding interfaces are:
  \begin{itemize}
  \item {\tt Miro::PioneerMotion} \\
    The {\tt Miro::PioneerMotion} interface is derived from the
    {\tt Miro::Motion} interface and gives access to the
    specialized movement commands of the Pioneer robot. It registers
    under it ancestors name {\tt Motion}.
  \item {\tt Miro::PioneerStall} \\
    The {\tt Miro::PioneerStall} interface (derived from {\tt
      Miro::Stall}) is registered as {\tt Stall}.
  \item {\tt Miro::Sonar} \\
    The {\tt Miro::RangeSensor} interface registers itself
    as {\tt Sonar}.
  \end{itemize}

  Note that the stall service sends {\tt Miro::PioneerStallIDL} events.
\end{description}


\section{Frame Grabbers and Digital Cameras}

The {\tt VideoService} provides an interface to several video image
sources.  All BTTV cards that are supported by the video for linux
project are also supported here (namely the Bt848/849/878/879 based
frame grabbers). Support for IEEE 1394 digital cameras (aka Firewire)
is provided.  The Matrox Meteor frame grabbers cards are also
supported, but note that kernel drivers are only available for the
older 2.2.x kernel series. See the chaper \ref{sec:video} for a
detailed discussion of the topic.


\section{SICK Laser Scanner}

The {\tt sickLaserService} provides an interface to a laser range
finder of type Sick PLS.
%%pls200 (TODO: check this?).
This sensor is connected to the controlling PC via a serial line. It
is delivered either with a RS-232 or a RS-422 compliant interface.

The idl-interface implemented by the {\tt sickLaserService} is
{\tt Miro::Laser}, which is derived from
{\tt Miro::RangeSensor}. The server can be driven in two different
modes, either the sensor automatically provides a scan about 40 times
a second, or every scan is polled, which results in a maximum scan
frequency of 10 times a second. The second mode also allows to choose
a different (i.e. lower) scan frequency. At the moment no other
special features of the Sick PLS200 are supported.

The configuration of this service is provided in xml, within the
section {\tt sick}. The following parameters are understood:

\begin{description}
\item[device:]
  The filename of the device which should be used.
\item[baudrate:]
  The baudrate for communication with the sensor,
  possible values are: 9600, 19200, 38400 and 500000.
\item[stdcrystal:]
  For achieving the unusual baud rate of 500000
  baud, we use a serial adapter, that is equipped with a 16MHz
  crystal, instead of the standard 14.??MHz. Due to this reason, the
  standard baud rates must be generated by modifying the divisor of
  the UART. If you have an off-the-shelf serial adapter, use
  {\tt true} here, you will probably not be able to use 500000 baud
  then.
\item[continousmode:]
  If {\tt true} is provided for this
  parameter the sensor will provide data automatically, about 40 times
  a second. If {\tt false} is used, the service will poll a
  measurement after an interval that can be specified with the
  following parameter.
\item[pollintervall:]
  The interval between two consecutive
  measurements, measured in microseconds. This parameter has no effect
  if continousmode is set to {\tt true}.
\end{description}

The following example shows the configuration used for our hardware:

\begin{lstlisting}[frame=tb]{}
<sick>
  <device>/dev/laser</device>
  <baudrate>500000</baudrate>
  <continousmode>true</continousmode>
  <stdcrystal>false</stdcrystal>
  <pollintervall>100000</pollintervall><!-- pollinterval in usec -->
</sick>
\end{lstlisting}

As the laser provides a specialization of a range sensor, we also
provide a scandescription for it. See section \ref{scandescrption} for
details.


\subsection{Possible Problems}

Due to the unusual and high data rate of 500000~baud a special serial
interface card is required. The card recommended by Sick provides a
16550A compatible interface, for that reason the current
implementation is able to use the card through the standard linux
serial device drivers, which keeps the implementation of the service
more portable. Unfortunately this introduces a big drawback: receiving
up to 50000 characters per second, results in at least $50000/16=3125$
interrupts per second, this is an enormous load. It is essential, that
every interrupt is handled on time, because the sensor does neither
support hardware, nor software flow control, and provides a packet
with data every 25~ms. If you have the chance to use an interface card
with a larger FIFO buffer than the standard 16 bytes, this will
provide a large improvement in performance, stability and system
reactivity.  We solved most of our stability problems, that were
caused by lost packages without a larger FIFO: We enabled irqs
throughout IDE interrupt processing (this may be extremely dangerous,
depending on your configuration, see man page of {\em hdparm}). Another
solution could be using {\em irqtune} to give the serial irq a higher
priority, or using {\em RT-Linux}.

\textbf{Due to the adaptations of the divisor for the serial line,
the {\tt SickLaserService} has to be suid root. If you use a
standard crystal, and you do not use 500000~baud, this is not
necessary.}


\section{DoubleTalk Speech Card}


\section{Directed Perception Pantilt}

\section{Global Positioning System}

The {\tt GpsService} provides an interface for any NMEA-0183 compatible
GPS receiver connected via a serial interface. The service implements
the idl interface {\tt Miro::Gps} or, depending on its configuration, a
specialized interface derived from it.

The service reads the {\tt GPS::Parameters} from the {\tt Gps} section
of the XML configuration file. The parameter format is derived from {\tt
Miro::TtyParameters} and includes the following fields:

\begin{description}
\item[Device:]
  The filename of the device which should be used. If no device is
  specified {\tt /dev/gps} is used.
\item[Baudrate:]
  The baudrate for communication with the GPS device. Most receivers
  communicate at 4800 baud.
\item[ReceiverType:]
  The type of the GPS receiver. If this parameter is not specified the
  {\tt generic} type is used.
\item[NmeaNotify:]
  This parameter is a string vector and can hold an arbitrary number of
  NMEA sentences that should be forwarded to the client. By default this
  feature is disabled.
\end{description}

A typical GPS configuration section might look like this:

\begin{lstlisting}[frame=tb]{}
  <section name="Gps" >
    <parameter name="GPS::Parameters" >
      <parameter value="/dev/ttyS1" name="Device" />
      <parameter value="4800" name="Baudrate" />
      <parameter value="generic" name="ReceiverType" />
      <parameter name="NmeaNotify" >
        <parameter value="GPRRE" />
        <parameter value="GPVTG" />
      </parameter>
    </parameter>
  </section>
\end{lstlisting}

In this example any NMEA sentences containing the type name {\tt GPRRE}
or {\tt GPVTG} are wrapped inside a {\tt GpsSentenceEventIDL} event and
forwarded to the client.

\subsection{Generic GPS receivers}

The generic GPS implementation is used if the {\tt ReceiverType}
configuration parameter is either omitted or set to {\tt generic}. In
this mode the service remains completely passive, i.e. it will read
sentences sent out by the GPS unit and evaluate any known and verified
message, but it will not send any commands to the device. If your device
requires any configuration (i.e. to enable Altitude Aiding or
Differential GPS) you have to use a separate tool to prepare it and, if
possible, make the configuration permanent before starting the GPS
service.

The generic implementation registers the {\tt Miro::Gps} interface as
{\tt GPS} in the CORBA Naming Service. It provides methods to poll for
the current position (both global and relative to a relocatable position
reference), to poll for the dilution of precision values and to place
the reference point for relative positioning.

\subsection{Star Track GPS receivers}

This implementation is specialized for use with GPS receivers
manufactured by CT Communication Technology GmbH Friedberg. It was
developed and tested using a Star Track GSX-1 GPS receiver, but it
should be compatible compatible with other models by the same vendor (or
maybe even by other vendors) as well. Set the {\tt ReceiverType} config
parameter in GPS::Parameters to {\tt ct} in order to use this interface
instead of the generic one.

In this mode the service implements the {\tt Miro::CtGps} IDL interface
which is derived from {\tt Miro::Gps} and registers it as {\tt GPS} in
the Naming Service. The specialized interface provides all methods of
the generic interface as well as additional methods to configure
Differential GPS, enable and disable Altitude Aiding, select the rate at
which each particular NMEA sentence is sent and store the current
configuration.

%\section{Higher-level Servers}

%In the current version of \miro, no high-level services are included.

%\subsection{MapServer}

%\subsection{RegionServer}

%\subsection{SaliencyServer}

%\subsection{ClassificationServer}


%%% Local Variables:
%%% mode: latex
%%% TeX-master: "miro_manual"
%%% End:
