\chapter{Writing Your First Programs}

This chapter tries to help you with the first steps of writing
programs that use the \miro framework. The CORBA environment and the
\miro framework seem to raise the bar for an easy entry into robot
programming. While this can hardly be denied they facilitate
tremendously the task of writing distributed programs. And since robot
control software is inherently distributed (ever thought of multiple
robots?) it seems the only way to go.

As you will see, most of the distributed programming complexity is
initially hidden from the user:

\begin{itemize}
\item The programmer simply calls methods of the devices interface.
\item The programmer communicates via proxy object to the remote
  service. The most tricky part is getting the object reference.
\item The distributed environment is transparent. The
  remote method invocation (RMI) is hidden entirely.
\end{itemize}


\section{Makefiles}
\label{sec:makefile}
In contrary to earlier versions of \miro, it is no longer based on the
ACE/TAO Makefile hierarchy, but uses the GNU autotools for a simpler
and more portable way to generate the Makefiles (beside that, this
inherently adds new nifty features, like e.g.\ the automatically
distribution via a simple \texttt{make dist}). Nevertheless it is
still possible to use the ACE/TAO style of writing Makefiles in
derived projects, as we didn't delete the \miro include rules and
macros (but please note that we do not maintain them any longer and
they may be deleted in future version).

Makefile templates reside in the directory {\tt
  \$MIRO\_ROOT/templates}, that should facilitate the creation of
further subprojects within the \miro directory structure but can just
as well be used to start new projects outside the {\tt \$MIRO\_ROOT}
directory. The makefiles are designed to either build subdirectories,
libraries or binaries. To use a template, simply copy it into the
corresponding directory under the name {\tt Makefile.am} and adapt it
as explained in the following sections. Additionally, you have to put
a line in \texttt{configure.ac} to the macro named
\texttt{AC\_CONFIG\_FILES} to let automake and autoconf know, that they
have to build a \texttt{Makefile} (\texttt{Makefile.in} respectively).

The Makefile templates are explained in more detail in the appendix
\ref{sec:new_project}.


\section{A Simple Sonar Client}

To discuss things using actual code, let's look at the simple task of
obtaining data from a sonar sensor device. Since this is a range
sensor, the sonar is queried via the generalized range sensor
interface. The only difference in querying an infrared or a sonar
device is the name under which these sensors are registered within the
naming service.

\lstinputlisting[frame=tb, caption={examples/sonar/SonarPoll1.cpp}]{SonarPoll1.cpp}
\label{lst:SonarPoll1}

A step by step walk trough the code:

\lstinputlisting[frame=tb, first=1, last=1]{SonarPoll1.cpp}

The file \lstinline!miro/Client.h! contains the definition of the class
\lstinline!Miro::Client!. See below.

\lstinputlisting[frame=tb, first=2, last=2]{SonarPoll1.cpp}

The file \lstinline!miro/RangeSensorC.h! provides the classes for the interface of the
sonar service. The 'C' at the end of the name stands for client. This
file is automatically generated from the idl-description of the interface.

\lstinputlisting[frame=tb, first=6, last=6]{SonarPoll1.cpp}

A standard main function is used.

\lstinputlisting[frame=tb, first=8, last=8]{SonarPoll1.cpp}

This class just wraps the few lines that are necessary at the beginning of any CORBA
application. It will therefore be normally instantiated in all \miro
applications first. This is a simple helper class that sets up the
CORBA environment in a standard way. Nothing tricky there.  It
initializes the orb etc. The call uses the command line arguments for
finding the naming service, and parsing other commands...

\lstinputlisting[frame=tb, first=9, last=10]{SonarPoll1.cpp}

This initializes a proxy object to the sonar service. This proxy
object is used as if it was the sonar service itself. Let's look
at the call a bit closer. \lstinline!Miro::RangeSensor_var! is the
CORBA equivalent of a standard C++ auto pointer for a
\lstinline!Miro::RangeSensor! object. That is the proxy becomes
automatically destroyed if the pointer goes out of scope. The
method {\tt resolveName()} is a template member function. Its
argument specifies the name that shall be resolved in the default
naming context of the naming service (see Chapter
\ref{sec:using}). The \lstinline!Miro::RangeSensor! in French
brackets specifies the type of reference that shall be returned by
the call. Note that this will only succeed if the reference stored
under the name of Sonar refers indeed to a {\tt Miro::RangeSensor}
object or to a derived ancestor of this class.

\lstinputlisting[frame=tb, first=11, last=11]{SonarPoll1.cpp}

The RangeSensor interface returns a pointer to the sensor scan, which
the caller obtains ownership of. So we use another auto pointer to
hold the return value of the call and to prevent us from memory leaks.
The IDL in the type name reflects the fact, that this is a IDL defined
data type. (This is just a naming convention of \miro that shall help
you to trace the roots to the documentation.) In the CORBA mapping for
C++ these types are mapped to plain C structs: No methods, no
inheritance, just public data members.

\lstinputlisting[frame=tb, first=13, last=13]{SonarPoll1.cpp}

This gets a sonar scan from the service. Note, that we do not see where
the service runs.

\lstinputlisting[frame=tb, first=15, last=20]{SonarPoll1.cpp}

These lines are used to print the received sonar scan to the
console.

%\lstinputlisting[frame=tb, first=22, last=22]{SonarPoll1.cpp}

%Ends the program. Returning zero indicates, that no error occurred.

%%\bigskip
%%- Note the {\tt Miro::} prefixes which define the namespace ...


\section{Using Namespaces}

By inserting the lines

\lstinputlisting[frame=tb, first=6, last=9]{SonarPoll2.cpp}

we map the relevant ``Miro'' data types from the \lstinline!Miro!
namespace into our global namespace. This shortens the type name
specifiers, but also hides from where they are coming from. Note that
a simple

\begin{lstlisting}[frame=tb]{}
using namespace Miro;
\end{lstlisting}

would do the same trick. But mapping a namespace completely is
generally not a good idea, since it tends to produce name conflicts
and secondly code reviewers can trace the origin of the types less
easily. Be especially careful in header files. You are flattening the
namespace for everyone that has to include your header file, which can
lead to bad surprises.

See the following listing which shows the same code again, but without
the \lstinline!Miro::! prefixes. The initial lines containing the includes
are skipped.

\lstinputlisting[frame=tb, first=6, caption={examples/sonar/SonarPoll2.cpp}]{SonarPoll2.cpp}
\label{lst:SonarPoll2}


\section{Adding Exception Handling}

Handling of error conditions itself is error prone. Exceptions make
the handling of error conditions easier, but errors are errors and
therefore stay somehow nasty. What's quite easy to achieve by the use
of exceptions is to print some diagnostic output and exit instead of
gracefully segfaulting. This is done in this example by enclosing the
code in the main function in a try/catch block.

Since \miro defines ostream operators for every \lstinline!Miro::...IDL!
type in \lstinline! miro/IO.h!, we include that file and abandon the
handcrafted streaming of the sonar data.

Note the different kinds off exceptions. There are \miro
exceptions. Exception types defined in \miro are beginning with a
big E as a naming convention. These indicate problems on the
service side, like hardware problems (may be the batteries?), bad
service calls (trying to accelerate the robot to warp 1?) or load
problems. Then there are CORBA exceptions. Those occur if there
arise some communication problems: A service went down, the robot
is loosing the radio ethernet connection etc. Since all \miro
exceptions derived from \lstinline!CORBA::UserException!, those
also are catched within the first \lstinline!catch! block of the
example code. You do not have to catch all exceptions. An uncaught
exception will lead to program termination. Just as if you'd catch
them at the end of main and exit...

Note also, that the instantiation of Client isn't within the try/catch
block. This is intentionally. Exceptions that can arise in the
construction of a Client instance are CORBA exceptions. The ostream
operators for CORBA exceptions used by TAO need an ORB
instance. Since the ORB is instantiated within the Client class, it
will not exist after destruction of the Client --- and this would be
done at the end of the try block. Therefore we catch exceptions within
the constructor of Client, print them to stderr and exit. There is
little to do for the client program anyhow, if it can't access the
services.

\lstinputlisting[frame=tb, first=14, caption={examples/sonar/SonarPoll3.cpp}]{SonarPoll3.cpp}
\label{lst:SonarPoll3}

\section{An Asynchronous Sonar Client}
\label{sec:notify}

By now we were actively requesting for the data of a service. But
think of polling for tactile events that way. Having a good
collision avoidance, those events should hardly ever occur.
Nevertheless, as soon as there is a tactile signalling a
collision, the robot should immediately react to this event. By
polling it would have to call the tactile interface thousands of
times, just not to miss the one event it can't effort to miss. And
since the sensors are actively collecting their data, shouldn't
they be able to trigger the data processing within the robot? - Oh
yes, they can.

For this purpose the notification framework within \miro does
exist. It is based on the CORBA Notification Service
\cite{OMG:00-5}, and precustomized by some utility classes. They
enable clients to subscribe to arbitrary events of a notification
channel. The data gets pulled to them as soon as it becomes
available at the producer (e.g. a range sensor device).

To show how asynchronous event processing works within \miro, lets
look at a small example. First we look at the code to handle the
events.

\lstinputlisting[frame=tb, first=24, last=53, caption={examples/sonar/SonarNotify.cpp}]{SonarNotify.cpp}
\label{lst:SonarNotify}

A step by step walk trough the code:

\lstinputlisting[frame=tb, first=24, last=24]{SonarNotify.cpp}

The class \lstinline!SonarNotify! provides a callback for the event
channel. It is derived from \lstinline!Miro::StructuredPushConsumer!,
which handles the registration at the notification service etc. All
the initialization and registering is done within the constructor of
this super class. The method \lstinline!push_structured_event! is the
callback that is called from the notification service.

\lstinputlisting[frame=tb, first=31, last=31]{SonarNotify.cpp}

The \lstinline!EventTypeSeq! class is for specifying the events that you
want to be subscribed for. It is an incremental protocol. So you make
a list of events you want to receive from now on and another with the
events you wish to no longer become informed about. Since we are just
subscribing, the second list is empty.

\lstinputlisting[frame=tb, first=33, last=35]{SonarNotify.cpp}

Events are subscribed by \lstinline!domain_name! and
\lstinline!type_name!. Never forget to set the length of a
sequence explicitly. Specifying the length as a constructor
parameter just reserves the number of elements, the length of the
sequence is still zero.

\lstinputlisting[frame=tb, first=37, last=37]{SonarNotify.cpp}

Tell the consumer admin what we want to subscribe for.

\lstinputlisting[frame=tb, first=48, last=49]{SonarNotify.cpp}

The payload of a structured event is contained within the
\lstinline!remainder_of_body! field with is of type
\lstinline!CORBA::Any!.  Therefore you can get a const pointer to the
data with overloaded the \lstinline!operator >>= ()!. The return value
is a \lstinline!CORBA::Boolean! indicating success of the operation.
Even though a \lstinline!CORBA::Any! can hold any IDL defined struct,
you can only extract type \lstinline!T! from a \lstinline!CORBA::Any!
if it actually contains an object of type \lstinline!T!.

\lstinputlisting[frame=tb, first=54, caption={examples/sonar/SonarNotify.cpp}]{SonarNotify.cpp}
\label{lst:SonarNotify2}

Lets now have a look at the \lstinline!main! function:

\lstinputlisting[frame=tb, first=57, last=57]{SonarNotify.cpp}

Since a push consumer is called (pushed) by the event producer, it
is actually a server instead of a client. Therefor the
\lstinline!Miro::Server!  class is instanciated. It performs the
necessary calls to init the CORBA environment. It is basically the
same as the \lstinline!Miro::Client!  but we also need a POA to
register the consumer object.

\lstinputlisting[frame=tb, first=59, last=59]{SonarNotify.cpp}

Just like the former \lstinline!Miro::RangeSensor! interface, we
resolve the \lstinline!EventChannel! by name at the naming service.

\lstinputlisting[frame=tb, first=61, last=61]{SonarNotify.cpp}

Instanciate the push consumer. It does all the necessary
initialization within its constructor.

\lstinputlisting[frame=tb, first=63, last=63]{SonarNotify.cpp}

Now we enter the CORBA event loop. This will not return, until the
process is signaled by SIGINT or SIGTERM.

%%% Local Variables:
%%% mode: latex
%%% TeX-master: "miro_manual"
%%% End:
