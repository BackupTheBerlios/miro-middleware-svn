\chapter{Using \miro}
\label{sec:using}

In order to use \miro programs, the following steps have to be performed:
\begin{itemize}
  \item start the naming service
  \item start the services
  \item start the client program(s)
\end{itemize}

These steps are described in the following.


\section{Starting the CORBA Naming Service}

The CORBA naming service is some sort of telephone book service for
the lookup of remote objects. Its organized like a file system. All
entries in the naming service can be either object references, the
pointer to a remote object, or naming context references that point to
another subdirectory of the naming service.  All CORBA objects of
\miro are registered by {\em name} at a CORBA {\em naming service}. In
the \miro project, the TAO implementation of the CORBA naming service
is used and therefore must be started before any service is run.

The TAO naming service is started by the following command:

\begin{lstlisting}[frame=tb]{}
cd ${TAO_ROOT}/TAO/orbsvcs/Naming_Service
./Naming_Service
\end{lstlisting}

In order to get debug messages it can also be started like this:

\begin{lstlisting}[frame=tb]{}
cd ${TAO_ROOT}/TAO/orbsvcs/Naming_Service
./Naming_Service -ORBDebug -ORBDebugLevel <level>
\end{lstlisting}

where {\tt <level>} is a debug level of 1 to 10 (10 gives most debug information).


\subsection{Naming Service Lookup by IP-Multicast}

Programs that use the TAO ORB can locate the naming service by IP
multicast as long as it resided in the same subnet, or within the same
multicast group. This is a special (proprietary) feature of the TAO
naming service. You have to enable this
feature explicitly by the command line option {\tt -m 1}. To avoid
conflicts, there should only be one naming service in a subnet. To
make sure no other naming service is active or to check the entries in
an active naming service, TAO provides a utility called {\tt nslist}.
It can be started by calling:

\begin{lstlisting}[frame=tb]{}
${TAO_ROOT}/TAO/utils/nslist/nslist 
\end{lstlisting}
%%$

It lists the contents of the currently running naming service and
returns an error, if no naming service is found.

\subsection{Naming Service Lookup by Environment Variable}

Additionally it is possible to not use the multicast feature of the
TAO naming service and provide the IOR of the naming service
as command line arguments or via environment variables to any program
that uses TAO. This is explained in detail in the TAO documentation
under:

\begin{lstlisting}[frame=tb]{}
${TAO_ROOT}/TAO/docs/INS.html
\end{lstlisting}
%% $

Note, that for retrieving the IOR of an already running naming service the
little tool {\tt nsIOR} resides in the bin directory of \miro.

You can also tell the Naming Service to listen to a specific endpoint
instead of choosing its own by the use of the {\tt -ORBEndpoint}
command line option. Ensuring that the IOR of the naming service is
the same, after each restart. 

\begin{lstlisting}[frame=tb]{}
\${MIRO_ROOT}/scritps/naming_service
\end{lstlisting}

is a small init script for the SuSE Linux distribution, that can be
installed in {\tt /etc/rc.d} for starting the naming service
automatically at boot time.


\subsection{Naming Contexts}

To provide the possibility to operate multiple robots in parallel,
\miro uses the concept of naming contexts to distinguish
between services of individual robots. One can imagine {\em naming
  contexts} as different folders, entries in different folders do not
interact. The default naming context \miro services use is
{\tt Miro}. The canonical naming convention in a multi robot
scenario is, that every robot uses its own name. Therefore every
\miro service and client accepts an additional command-line argument
that sets the {\em naming context}: 
{\tt -MiroNamingContext <name>}. Or as short form: 
{\tt -MNC <name>}.
To access services of different robots simultaneously, the
{\em naming context} has to be specified for the resolution of
each individual service (see Chapter \ref{sec:clients}).


\section{Starting Services}

When the naming service is running and can be accessed from all
machines, the most difficult part of the setup of the distributed
system environment is mastered. From now on we are back to the
unbounded problem set of robotics.

Simply change to the directory {\tt \$MIRO\_ROOT/bin} and start the
service you want to run. For example {\tt b21Base},
{\tt laserServer}, {\tt videoService}, {\tt dpPanTilt}. For
an extensive list of all available services see Chapter
\ref{sec:services}. Since the services depend on the {\tt *.conf}
files provided in this directory, strange effects are bound to occur
if the services are started from a different directory. Keep in mind,
that the hardware dependent services must run on the computers the
corresponding hardware is attached to.

%Services register their interfaces at the naming service. To allow
%multiple robots to register the same interfaces, each robot chooses a
%naming context in the naming service. This can be seen as
%subdirectories within a file system. Files with the same name can live
%peacefully side by side within different directories. To chose a naming
%context different than the default (which is {\tt Miro}) the command
%line option {\tt -MiroNamingContext context} can be used.

When a service is ended disgracefully, it probably doesn't deregister
its interfaces at the naming service. This will prevent the restated
service from registering the new interface references. TAO provides a
utility to manually remove references from the naming service. 

\begin{lstlisting}[frame=tb]{}
${TAO_ROOT}/TAO/utils/nslist/nsdel
\end{lstlisting}
%%$

But \miro also provides a command line option to force the service to
rebind the interface: {\tt -MiroRebindIOR}.

For a complete list of command line option see Chapter
\ref{sec:command-line}.


\section{Starting Client Programs}

\miro comes with a rich set of client programs and utilities that help
you to explore the robots sensory and actuatory devices. Starting a
client program is basically the same as starting a service. Just
change to the directory {\tt \$MIRO\_ROOT/bin} and start whatever
you want. Since these programs aren't bound to specific hardware
devices they can be run on almost any machine.

Note, that to address a service that is registered within other than
the default naming context at the naming service you also have to
provide the {\tt -MiroNamingContext} {\em context} argument at the
command line (in short {\tt -MNC} {\em context}).

%%% Local Variables: 
%%% mode: latex
%%% TeX-master: "miro_manual"
%%% End: 
