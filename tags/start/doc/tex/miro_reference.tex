\documentclass[10pt]{book}

%% insert the common definitions
\usepackage{a4} % A4 paperformat
\usepackage{isolatin1} % for input of 8 bits character
\usepackage{makeidx} % enable indexing
\usepackage{verbatim} % better verbatim environment
\usepackage{epsfig}
\usepackage{xspace} % add extra space at the end of the word if necessary
\usepackage{color} % provides standard LaTeX colors
\usepackage{fancyhdr}
\usepackage{float} % float environment enhancements
\usepackage{alltt} % defines alltt environment which is like verbatim, but allows some additional formating
\usepackage{listings} % pritty print of code-listings

%\usepackage[german]{babel} % multilingual package
%\usepackage[normal]{subfigure} % support for the inclusion of small subfigures
%\usepackage{latexsym} % some extra mathematical symbols
%\usepackage{exscale} % implements scaling of the `cmex' fonts


% configure the listings-environment
\lstloadlanguages{C++}
\lstset{
  basicstyle=\sf,
  commentstyle=\hfill \sl,
  language=C++,
  defaultdialect=ANSI,
  flexiblecolumns=true,
  indent=\parindent,
  extendedchars
  %%  texcl
  %% formfeed={}
}


% command definitions

% logo of the miro-software writing
\newcommand{\miro}{\textit{Miro}\xspace}
% logo of the sparrow writing
\newcommand{\sparrow}{\textit{Sparrow-99}\xspace}
% logo of the b21 writing
\newcommand{\bXXI}{\textit{B21}\xspace}


% font definitions
\newcommand{\Xbombastic}{\fontfamily{ppl}\fontseries{b}\fontsize{1.5in}{1.8in}\selectfont}


% color definitions
\definecolor{LightGrey}{rgb}{0.9,0.9,0.9}


% page definitions
\setlength{\parindent}{0pt} % Keine Absatzeinr�ckung
\setlength{\parskip}{7pt plus 1pt minus 1pt} % Absatz Abstand 7pt


% misc
\makeindex
\bibliographystyle{plain}

%=============================== Examples ====================================

%% code-parts as an extra paragraph
%% \begin{lstlisting}[frame=tb]{}
%% for (unsigned int i = 0; i < sonarScan.length(); ++i)
%%   cout << sonarScan.[i] << " ";
%% \end{lstlisting}

%% Included code-parts from a file
%% \lstinputlisting[frame=tb]{simpleTestClient.cc}

%=============================== Examples (end) ====================================

%%% Local Variables:  
%%% mode: latex
%%% TeX-master: "miro_manual"
%%% End: 


%% special definitions
\usepackage{doxygen} % special package for the reference-part created by doxygen
\setcounter{tocdepth}{1}

%% =============================================================================

\begin{document}
%\maketitle

\thispagestyle{empty}   
\begin{center}
  \vfill
  \includegraphics[width=5in]{../fig/signature.ps}\\
  \vspace{30 mm}
  {\Xbombastic \miro}\\
  \vspace{10 mm}
  {\Large \textbf{Reference Manual}}\\
  \vspace{10 mm}
  \textbf{Version 0.7}\\
  \vspace{10 mm}
  \today\\
  \vfill
\end{center}

\newpage

\begin{center}
  \includegraphics[width=5in]{../fig/pincolla.ps}
  \bigskip 
  For more paintings, see {\tt http://www.bcn.fjmiro.es/}
\end{center}

\pagestyle{headings}

\newpage
\tableofcontents

%% =============================================================================

\chapter{Introduction}

\section{The \miro Group}

The \miro developers are (in alphabetical order): 
\begin{itemize}
  \item Stefan Enderle
  \item Gerhard Kraetzschmar
  \item Stefan Sablatn�g
  \item Steffen Simon
  \item Hans Utz
\end{itemize}

%% =============================================================================

\chapter{Requirements}

\begin{itemize}
  \item The Linux-kernel must be compiled with multicast enabled.
  \item ACE 5.1.x and TAO 1.1.x (we currently recommend the latest beta)
  \item gcc 2.95.3 
  \item QT 2.3 (and above)
  \item doxygen / LaTeX (for documentation)
  \item {\tt /soft/ACE\_wrappers/ace} must be part of {\tt LD\_LIBRARY\_PATH}
\end{itemize}

%% =============================================================================

\chapter{Directory Structure}

\begin{verbatim}
Miro
  |
  +--- idl                        all idl descriptions
  +--- include (*)                all include files
  |      |
  |      +-- auto_idl (*)        header files automatically produced by the idl compiler
  +--- src
  |      |
  |      +-- abus                 abus utilities
  |      +-- miro                 miro utilities
  |      +-- c1lib                c1  utilities  (RENAME!)
  |      +-- mcp                  mcp utilities
  |      +-- msp                  msp utilities
  |      +-- base                 base server (sonar, IR, tactile servers)
  |      +-- laser                laser server
  |      +-- auto_idl (*)        source files automatically produced by the idl compiler
  +--- bin (*)                    all binaries (except demos)
  +--- demo                       
  |      |
  |      +-- qtLaser              a simple graphical laser client
  |      +-- qtSonar              a simple graphical sonar client
  +--- doc
  |      |
  |      +-- html                 html documentation (miro.html)
  |      +-- tex                  TeX documentation (miro.ps)
  +--- lib (*)                    all libraries
\end{verbatim}

Note: \\ 
The directories marked with {\tt (*)} are not part of the \miro
repository but are dynamically built when calling
{\tt make}.

%% =============================================================================

\chapter{Make}

The following targets can be called in the main Makefile:
\begin{itemize}
  \item all
  \item clean 
  \item distclean 
  \item depclean 
\end{itemize}

Simply typing {\tt make} calls the target {\em all}.


\section{Target: build}

The main {\tt Makefile} calls 
\begin{itemize}
  \item {\tt idl/Makefile}
  \item {\tt src/Makefile}
  \item {\tt doc/Makefile}
\end{itemize}


{\tt idl/Makefile} creates the directories
{\tt include/auto\_idl} and {\tt src/auto\_idl}. Then it calls
the idl-compiler in order to translate all idl-files. The generated
.hh- and .cc-files are moved to the created directories.

{\tt src/Makefile} calls make in all subdirectories.

{\tt doc/Makefile} generates both HTML and \TeX documentation by
invoking {\em doxygen} and \LaTeX.


\section{Target: clean}

- *.o 


\section{Target: distclean}

- dependency files

- libs

- binaries

%% =============================================================================

\chapter{Styleguide}
\input{styleguide}

%% =============================================================================

\chapter{Source Code Documentation}

doxygen is used

Example:

\begin{lstlisting}[frame=tb]{}
/**
 * Some comments about this class are written here.
 * You can use multiple lines.
 *
 * An empty line produces a new paragraph.
 *      
 * @author Stefan Enderle  (an author can be given!)
 */
class MyClass : public some other class
{
private:
  /**
   *  say something about this variable
   */
  double d;

public:
  /**
   *  say something about this method 
   *
   * @exception SOME_EXCEPTION                    
   *
   * @param a This param is used for something nice
   */
  bool method(int a);
};
\end{lstlisting}

%% =============================================================================

\input{refman}

%% !!! Be sure, that no other stuff comes here. particularly no \end{document}

%\chapter{Class Reference}
%\label{SEC_CLASS_REFERENCE}
%\input{miro_class_reference.nh}
%\input{miro_idl_reference.nh}
%\printindex
%\end{document}




