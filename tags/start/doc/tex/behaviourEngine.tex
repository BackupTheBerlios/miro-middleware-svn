\chapter{Behaviour Engine}

Controlling the actuators of an autonomous mobile robot is one of the
central aspects of mobile robot research. As \miro exposes the
interfaces to the motor controllers etc. of the mobile platform it
enables researchers to easily evaluate new approaches to model sensor
actor control loops. Be aware that at this point the latencies
introduced by various levels of the robot architecture can become a
critical issue. To eliminate the network latency and prevent yourself
from occasional network bandwidth problems we recommend that you run
your control programs collocated on the same computer with the service
that accesses the actuators device.

\miro supports the behavioural control paradigm introduced by Brooks
\cite{Brooks} by its own behaviour engine. It is designed to allow for
a quick start into behaviour robotics writing your own behaviours. Yet,
due to its open and extensible design it is also capable of handling
sophisticated control tasks as demonstrated by its use within the
RoboCup-scenario by our \sparrow robots.

\section{The Concept of Behaviours}

The basic idea of the behaviour approach to robot control is as
follows. Instead of the sense-plan-act paradigm of classical AI, the
task is splitted into a set of reactive behaviours, that each try to
fulfill a small subtask of the problem set. For each of the task only
a very limited part of world modeling is needed (often, even raw
sensor readings are sufficient). By combining the output of the
various behaviours by an arbiter, the emergent higher level behaviour
of the system is achieved, solving the recommended high level task.

An accepted bottleneck of this approach is the arbitration and
calibration of large sets of behaviours necessary to fulfill different
aspects of a high level task. Therefore in the \miro framework the
behaviour engine also support the hierarchical decomposition of
behaviour sets by allowing them to be grouped in so called action
patterns that can be activated alternately by so called transition
messages. A set of action patterns is called a policy within the \miro
framework.

\section{Introductory Examples}

Lets look at the simple action pattern, who's high level task is to
explore the environment by performing a random walk. This is easily
splitted into two distinct behaviours. The first subtask is not to
collide with the environment. This can be achieved by an avoid
behaviour, that reads the front sonar sensors to determine how far it
is away from the nearest obstacle. If the minimal distance is below
some threshold, it tells the arbiter to turn away into another
direction. The second behaviour would be a wander behaviour. It selects
from time to time just randomly some translational and rotational
velocity, making the robot move around. It does not have to care about
obstacles, since those are taken care of by the avoid behaviour.

The task of choosing the actual velocities to be applied to the motors
is performed by the arbiter. It therefore plays a central role in the
behaviour approach. There exist various kinds of arbiters, all
choosing different policies for this task. But a simple priority based
arbiter suffices for many scenarios. In \miro there currently just
exists a priority based arbiter, but since it is an extensible
framework you can easily plug in your own one. The priorities in this
example could be applied straight forward. The avoid behaviour has
higher priority as the wander behaviour.

Note the easy extensibility of this approach. For example, if we have
some bump sensors that indicate collisions with the environment. We
could just add another, let's call it emergency stop behaviour and
assign it the highest priority.  If one of the bumpers is pressed it
makes the robot stop and wait for rescue by one of the operators.

As a further extension we assume we have two tasks. The first is the
random walk described above, the second would be a wall following
behaviour, that simply is capable to drive the robot in a defined
distance along a wall. The decision what to do is provided by an
external source, say, a button located on the robot and pressed by the
operator when demonstrating the management the capabilities of the
newly bought autonomous mobile robot. The second action pattern looks
quite like the first one, except that the wander behaviour is
exchanged by a wall following behaviour, that drives the robot along
the wall. (Note that this behaviour does not have to care about walls
thar are blocking the way at the end of the corridor, since this can
be taken care of by the avoid behaviour.) These two action patterns
now form a policy. The pressing of the button sends a transition
message, that disables the currently running action pattern and
enables the other one.

The action pattern / transition message mechanism also fits naturally
for coupling reactive behavioural control with deliberative planning
architectures. The transition could also be raised by a path planner
and the corresponding action patterns could be 'move to point', 'drive
through door' or 'dock at power supply'. Indeed this was already done
within \cite{Hans-Diplom}.

\section{Implementing a Behaviour}

All behaviours are derived from the base class {\tt Miro::Behaviour}. The
central method of this class is the method {\tt action()}. It is to be
overwritten by the programmer and has to contain the behavioural code.
The method becomes invoked by
the behaviour framework, and is expected to call an arbitration method
or send a transition message. Note the inversion of control flow. A
behaviour is not allowed to jump into some infinite loop, but the
{\tt action()} method is called continously as long as the behaviour is
active within some action pattern.

From an implementation perspective, there are three kinds of
behaviours, depending on how the data, the behaviour bases its
decisions on, is delivered.

\subsection{Miro::TimedBehaviour}

This is the base class for a timer scheduled behaviour. A behaviour
derived from this class runs with all its brothers and sisters
cooperatively multi-threaded in one thread of control. The pace at
which its {\tt action()} method is called is selectable by a parameter of
the base class.

This is the most simple form of behaviour design and in many cases most
straight forward. It is especially suitable for behaviours that poll
their sensory information of the world model, like for instance the
current scan of the laser range finder for collision avoidance. Also
the above mentioned wander behaviour - which doesn't use any sensory
information would be best implemented as a child of the
{\tt Miro::TimedBehaviour} class.

See {\tt \$MIRO\_ROOT/examples/b21Behaviours/wander.cc/hh} for a
complete source code example.

\subsection{Miro::EventBehaviour}

This is the base class behaviours using asynchronous sensory
information published by the Notification Service. It subscribes for
the events it likes to get pushed and the action method is called
whenever a new message arrives. A pointer to the current structured
event is then available as member variable.

The emergency stop behaviour described
above would be a good candidate for such a behaviour, since it would
have to poll excessively as a timed behaviour in order to minimize the
latency between a bumper pressing a the actual stop, while it would
only need to call an arbitration method in very rare occasions.

See {\tt \$MIRO\_ROOT/examples/b21Behaviours/tactileStop.cc/hh} for a
complete source code example.

\subsection{Miro::TaskBehaviour}

Behaviours derived from this calls run within their own thread of
control, not blocking others even if they need fairly long for their
decisions. Note that such a behaviour is likely to be miss designed,
since it contradicts the behavioural approach to need excessive time
to come to a decision. But if you have need for something like this,
since you are doing something we didn't think of, this is the class to
base your behaviour on. Note however, that since behaviours are shut
down cooperatively, also a task behaviour is not allowed to loop
indefinitely within its action method. If the behaviour is still part
of the currently running action pattern, its action method will be
call immediately after giving control back to the behaviour framework.

\subsection{Behaviour Parameters}

Behaviours usually get used within different action patterns. But they
are often expected to behave slightly different within each
constellation. Therefore each behaviour has an associated
{\tt BehaviourParameters} class which is designed to hold the different
parameter sets for the different use cases of a behaviour. These
parameter classes become initialized on startup of the behaviour
framework and are expected to be static during the run of an entire
policy. (How to handle dynamic parameters like destination coordinates
of a 'move to position' behaviour is explained in section \ref{sec:init})

Each behaviour is expected to have its own {\tt BehaviourParameters}
class factory method, for dynamic instantiation. In the constructor,
the default settings of the parameters are expected to be set. There
are two more methods for each {\tt BehaviourParameters} class to be
populated. The first is a method to parse its parameters from an xml
DOM tree, the second is for debug output. If your behaviour doesn't
need any additional parameters, you can skip this work. Its parents
Parameters class will then be used. --- The {\tt BehaviourParameters}
framework is about to be subject to automatic code generation (see
section \ref{sec:makeParams}). So hopefully you don't have to worry
about it too much.

\subsection{Behaviour Initialization}
\label{sec:init}

Before an action pattern becomes activated all its behaviours {\tt
  init()} methods are called after the call to its {\tt init()}
method.. This allows behaviours to initialize their per task
parameters (like destination coordinates) in a convenient way. Note
however, that the {\tt init()} method can be called while the
behaviour is already active (see section \ref{sec:active-inactive} and
so its {\tt action()} method can be concurrently running. Therefore a mutex
is needed to avoid race conditions.

Parameters that are valid for the whole lifetime of a behaviour, such
as references to the robots services or other objects within the
behaviours address space are best to be passed during construction
of the behaviour, forming a so called initializing constructor.

\subsection{Behaviour Activation and Deactivation}
\label{sec:active-inactive}

When a behaviour is to become active due to it being part of an action
pattern its {\tt open()} method is called. If the behaviour is no longer
part of the next to be running action pattern its {\tt close()} method is
called. If a transition from one action pattern to another is
performed and the behaviour is part of both behaviour sets, than no
calls to {\tt close()} and {\tt open()} methods will be issued. Only the
behaviours {\tt init()} method will be called, to allow it to update its
parameter sets. This is useful to avoid unnecessary behaviour
shutdown.

Note also, that the {\tt close()} method can be called while the behaviour
is concurrently within the {\tt action()} method. --- Its call to the
arbiter will then just be ignored. On the call of {\tt open()} however it
is guaranteed that the behaviour currently is not running.

\section{Arbiters}

The arbiter framework is similar to the behaviour classes. It will be
explained in more detail as soon as we do some more work on arbiters.

\section{Building Action Patterns}

``Now I built all my behaviours. What code do I have to write to make
them an action pattern?'' Well, you don't have to write code. Action
patterns and policies are defined within an xml file. Allowing for
fast and convenient modifications. Which is especially cool when
debugging. So let us take a look at how to put those things together.

\subsection{The Policy File}

A policy file can have the following tags and attributes:

$<$!--MiroPolicyDocument--$>$

\begin{description}
  \item[policy]
    One file allways describes one entire policy. A policy defines a
    state machine that consists of a set of action patterns. Each
    action pattern describing on state of the machine.
  
  \begin{description}
  \item[actionpattern]
    An action pattern describes a set of behaviours, that can run
    simultaniously and produce some emergent higher level behaviour.
    

    Attributes:
    \begin{itemize}
    \item \textit{name} The name of the action pattern.  
    \item \textit{start (true/false)} The action pattern to be active
      at startup is to be marked true, the others false or unmarked.
    \item \textit{x, y} For GUI purpose, just ignore them.
    \end{itemize}
  
    \begin{description}
    \item[behaviour]
      That's what all the chapter is about. Read it again.

      Attributes: 
      \begin{itemize}
      \item \textit{name} Name of the behaviour.
      \end{itemize}  

      \begin{description}
      \item[parameter]
        Behaviours parameters can be specified within the policy file
        for each action pattern. 

        Attributes: 
        \begin{itemize}
        \item \textit{name} Name of the parameter.  
        \item \textit{value} The value of the parameter.
        \end{itemize}
      
      \end{description}  

    \item[transition]
      Message that triggers a transtition to another
      action pattern.  

      Attributes: 
      \begin{itemize}
      \item \textit{target} The action pattern to activate next.
      \end{itemize}  

    \item[arbiter]
      To decide which action to choose from the different outputs of
      the different behaviours one needs to arbitrate one way or the other.

      Attributes: 
      \begin{itemize}
      \item \textit{name} The name of the arbiter to use.
      \end{itemize}
    \end{description}
  \end{description}
\end{description}

\subsection{The Repositories}

A policy can be built on the basis of an xml description. For the
parser of this description to be able to construct the policy, it has
to be able to refer to instances of behaviours and arbiters that are
mentioned within the xml file by their name.  For this purpose a {\tt
  Miro::BehaviourRepository} and a {\tt Miro::ArbiterRepository} class
do exist.  At these Repositories an instance of each behaviour and
arbiter has to be registered. The name of the behaviour has to be
reported by the {\tt behaviourName()} method.  Since we anly need one
instance of each of these repositories, there do exist a global
instance of eacht. A pointer to such an instance can be obtained by
the classes static method {\tt instance()}.

\subsection{A Behaviour Engine}

All the initialization stuff necessary before constructing an instance
of the Policy class can be done within the main function of your
program. This task consists mostly of obtaining the needed object
references, instancing the behaviours and arbiters, instancing
behaviours and arbiters and registering them at their respective
repositories.  This can be bundled within a so called behaviour engine
class as can be seen in the behaviour example at:

{\tt \$MIRO\_ROOT/examples/b21Behaviours/behaviourEngine.cc|hh}

\section{The Policy Editor}

Editing large xml files is tedious and error prone. Therefore \miro
offers an GUI based editor, with which you can build and edit your
policies. You can also edit the parameters defined within the
Parameters classes of your behaviours.

\subsection{Describing the Available Behaviours}

To be able to use your own behaviours within the PolicyEditor, you
have to describe their properties (name and parameters) within another
xml file, witch looks like this:

$<$!--MiroParametersConfigDocument--$>$

\begin{description}
  \item[config]
    A parameters config file describes the types, names and defaults
    for parameter classes as used within the \miro framework. 

  \begin{description}
  \item[config\_group]
    Configurations can be grouped together for various reasons. 

    Attributes:
    \begin{itemize}
    \item \textit{name} The name of the config group. Within the
      behaviour parameters framework the valid names are \textbf{behaviour} and
      \textbf{arbiter}. Within each group only config items of the
      specified type are allowed to occure.
    \end{itemize}

    \begin{description}
    \item[config\_item] A configurable item to built a parameters
      class for.

      Attributes:
      \begin{itemize}
      \item \textit{name} The name of the behaviour or the arbiter.
      \item \textit{parent} The name of the super class.
      \item \textit{namespace} The namespace, the class resides.
      \end{itemize}
  
      \begin{description}
      \item[config\_parameter] A parameter for the class.

        Attributes:
        \begin{itemize}
        \item \textit{name} Guess what.
        \item \textit{type} int, double, bool, std::string, etc.
        \item \textit{default} default value (set in the constructor)
        \item \textit{measure} Possible types: mm, msec, �, mm/s, �/s.
          Those will be used for input checking by the GUI interface.
        \item \textit{inherited true/false}
          This attribute is used to 
          overwrite defaults for inherited variables within the
          constructor of the class. If set to true, no new member
          variable will be added to the class, and only the
          attributes name and default will be evaluated.
        \end{itemize}

        text tag: some description for debug output or bubble help
      \end{description}
    \end{description}
  \end{description}
\end{description}


\subsection{Auto-generating Parameter Class Code}
\label{sec:makeParams}

Having written the description of your behaviours parameters within
xml, it is possible to auto-generate the necessary code for your
behaviours associated Parameters class. This is done by the tool {\tt
  makeParams} which will be soon available within the \miro bin
directory.


%%% Local Variables: 
%%% mode: latex
%%% TeX-master: "miro_manual"
%%% End: 
