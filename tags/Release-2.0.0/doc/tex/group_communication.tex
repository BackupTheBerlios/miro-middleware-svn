\chapter{Group Communication in Robot Teams}
\label{sec:NMC}

When it comes to multi-robot scenarios, group communication becomes an
issue of interest. As \miro is designed for usage in distributed
environments, most of the prerequisites for communication in teams of
robots are in place. But in the presence of WLANs, that have extremely
varying bandwidth, that dependents on the location of the computer
etc. some more requirements have to be met. Especially the TCP/IP
based transport protocol can become a bottleneck, when temporary
network breakdowns can regularly happen during operation.

\section{Event Channel Federation}

The event based communication of \miro is based upon the CORBA
Notification Service. In a team of robots, one central event
channel could be set up and used for team communication. While this is
most convenient from a user perspective, it creates a single point of
failure for a team of otherwise independent autonomous robots.

Using a single virtual event channel, that is distributed over
multiple channels, that run on the different robots helps both, the
programmers, as well as the systems fault tolerance. As such an event
channel federation is not part of the specification of the CORBA
Notification Service, \miro provides its own implementation. Namely
the Notify Multicast (NMC) module. Its implementation is based upon
the work done for the event channel federation of the RT-EC of TAO
\cite{gore01designing}. A successful application of the NMC module is
described in \cite{Utz++:04:RCS}.

\section{Notify Multicast}

One notification service is run on each robot (computer). The NMC
module attaches one consumer and one supplier to the robots local
event channel. Additionally it subscribes at a multicast group. Events
that are subscribed by other robots are exchanged via the multicast
group.

For the automatic arbitration of the set of events that need to be
exchanged via the multicast group, the offer/subscription protocol of
the Notification Service is used. Offers to and subscriptions at the
event channel are determined by the \texttt{domain\_name} and
\texttt{type\_name} fields of the event header. In \miro the domain
name is by convention the name of the robot. The NMC module
posts all events that subscribed for at the local channel, but that
are not offered locally to the multicast group and collects these
posts from the other robots. Events that are requested by other
robots and offered locally, are posted to the multicast group on arrival
on the local event channel. Those events, that are subscribed for at
the local channel, but not offered locally, are published to the local
channel as they become posted on the multicast group.

\begin{figure}
%\vspace*{-0.5cm}
\begin{center}
\includegraphics[width=10.0cm]{federatedchannel}
\caption{\label{fig:federated_channel}
A Federated Notification Channel Setup}
\end{center}

%\vspace*{-0.3cm}
\end{figure}

See also figure \ref{fig:federated_channel} for a possible
configuration of subscriptions and offers in a federated event channel.

\section{Usage}

The NMC module is part of most robot servers by default. It can be
enabled by the command line option \texttt{-MiroNotifyMulticast} or
\texttt{-MNMC} for short. It also exists, as a separate program:
NotifyMulticast.

\subsection{Parameters}

The parameters can be set in the robots configuration file. They are
located in the configuration group \texttt{Notification}. The
available parameters are.
\begin{description}
\item[MulticastGroup] The multicast group used for group communication.
\item[Timeout] The maximum age for events send over the group. This
  requires proper synchronization of the team clock (i.e. by NTP
  \cite{RFC2030}). A value o zero indicates, that all events should be
  processed regardless of their age.
\item[Subscription] These events are shared with other robots event if
  no one seems to listen. - Events that are exchanged on the bases of the
  offer/subscription management protocol need some time, until the
  request is acknowledged by the suppliers. By providing events with
  this option, this time can be skipped. Note, that only the type name
  is specified here. The domain name is assumed to be the robots name.
\end{description}


%%% Local Variables: 
%%% mode: latex
%%% TeX-master: "miro_manual"
%%% End: 
