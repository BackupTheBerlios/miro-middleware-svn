%% styleguide.tex

\section{Files}

All files start with a leading upper case character.

\subsection{Header Files vs.\ Source Files}

\begin{description}
\item[Header files:]
  These files contain the {\em declarations} of
  classes, functions, and constants that must be known to use the code. Header
  files must not contain function bodies, methods, and variable
  definitions. This is important, because each header file is included in
  all source files that need the declarations and so it may only contain
  declarations that may be duplicated.

\item[Source files:]
  These files contain the {\em implementations} of
  methods, functions, and variables, declared in the corresponding header file.
\end{description}

Exceptions: 

\begin{description}
\item[Inline code:] Methods and functions that are very small or
  mainly used in loops are often implemented as inline code. The
  function is not called, but the function body is directly
  substituted in the code.
\item[Template definitions:]
  These are generic class or function
  definitions, eg.\ a generic vector. The exact type (eg.\ vector of strings)
  is given at compile time. Template are mostly used in general class
  libraries. Those also are to be located at the end of the header file.
\end{description}

Both exceptions must be known at compile time. This means that they can not be
implemented in a source file ({\tt .c}-file), then compiled to an object
file, and later linked to the application. 
Therefore they are to be located at the end of the
header file. Please do not inline inline function within the class
declarations.

\subsection{File Extensions}

\begin{tabular}{ll}
  {\tt .cpp}  & C++ file \\
  {\tt .c}   & C file \\
  {\tt .h}   & C header file
\end{tabular}

Note that especially the build rules for libraries and dependcies will
fail, if other extensions are used.


\subsection{File Names}

\begin{enumerate}
\item File names begin with a uppercase letter.
\item In names which consist of more than one word the words are
  written together and each word that follows the first is begun with an
  uppercase letter. \\ 
  Example: {\tt SickLaserService.cc}
\end{enumerate}


\subsection{File Structure}
\label{SEC_FILE_STRUCTURE}

\begin{enumerate}
\item In C files, {\tt \#ifdef c\_plus\_plus extern "C"} must be
  used in order to compile the files with a C++ compiler. \\
  Remark: See the example in \ref{SEC_GENERIC_HEADER}.
\item The directive {\tt \#include "Filename.h"} is used for
  user-prepared include files.
\item The directive {\tt \#include <filename.hh>} is used for
  include files from libraries.
\item To avoid multiple includes of files, each header file must
  define the constant {\tt \_FILENAME\_H} and contain an
  {\tt \#ifndef/\#define} block at the beginning and an
  {\tt \#endif} at the end of the file. \\
  Remark: See the example in \ref{SEC_GENERIC_HEADER}. 
\end{enumerate}


\subsubsection{Generic Header File}
\label{SEC_GENERIC_HEADER}

\lstinputlisting[frame=tb]{Generic_header.cpp}

%% -----------------------------------------------------------------------------

\section{Identifier Names}
\label{SEC_IDENT_NAMES}

\begin{enumerate}
\item Classes begin with an uppercase letter. \\
  Example: {\tt LaserScan, Vector}
\item Methods, functions, objects (instances), and variables begin
  with a lowercase letter. \\ 
  Example: {\tt write(..), sonar.get(..), int number}
\item Pointer variables have the prefix {\tt p}. \\
  Example: {\tt Laser* pLaser;}
\item \label{ITEM_NAME_CONVENTION} 
  In names which consist of more than one word the words are
  written together and each word that follows the first is begun with an
  uppercase letter. \\ 
  Example: {\tt sonarScanVector}
\item Do not use underscores to begin identifiers. \\
  Remark: Underscores at the beginning are often used by internal compiler
  variables. One should be able to distinguish them easily from user variables.
\item Do not use underscores to separate words in identifiers. \\
  Exception: Item \ref{ITEM_UPPERCASE_CONSTANTS}.
\item \label{ITEM_UPPERCASE_CONSTANTS} Constants are written in
  uppercase letters where underscores are used to separate words. \\ 
  Example: {\tt VECTOR\_LENGTH, PI\_TIMES\_PI} \\
  Remark: See section \ref{SEC_GENERAL} item \ref{ITEM_NO_DEFINES} for the use
  of constants.
\item Names must differ by more than only the use of uppercase and
  lowercase letters.
\end{enumerate}

%% -----------------------------------------------------------------------------

\section{General}
\label{SEC_GENERAL}

\begin{enumerate}
\item In {\tt \#include} statements, absolute path names must not be used. \\
  Remark: Use compiler option {\tt -I} to specify include paths.
\item In {\tt \#include} statements, relative path names should not be used. \\
  Remark: Use compiler option {\tt -I} to specify include paths.
\item \label{ITEM_NO_DEFINES} Constants should not be defined with
  {\tt \#define}. Use the directive {\tt const} instead to indicate
  constants whenever possible. \\ 
  Example: {\tt const int VECTOR\_LENGTH = 20;} \\ 
  {\em Not:} {\tt \#define VECTOR\_LENGTH 20} \\ 
  Remark: This allows strict type checking by the compiler.
\item \label{ITEM_NO_MACROS} Do not use macros ({\tt \#define}.) \\
  Remark: Use inline functions for fast computation and templates functions for
  genericity.
\item Each function should contain a header explaining the
  functionality, the parameters, and the return value.
\item If data structures are passed as parameters to functions,
  references should be used instead of pointers. \\ 
  Example: {\tt void writeLaserScan(LaserScan\& laserScan);} \\
  {\em Not:} {\tt void writeLaserScan(LaserScan* pLaserScan);} 
\item The keyword {\tt const} has to be used to indicate that
  parameters are not changed. \\ 
  Example: {\tt void writeLaserScan(const LaserScan\& laserScan);} \\
  {\em Not:} {\tt void writeLaserScan(LaserScan\& laserScan);}
\item Names of objects should contain the class name. \\
  Example: {\tt LaserScan laserScan;} \\
  {\em Not:} {\tt LaserScan scan;} 
\end{enumerate}



%%% Local Variables: 
%%% mode: latex
%%% TeX-master: "servers"
%%% End: 
