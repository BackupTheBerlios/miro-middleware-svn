\chapter{Behaviour Engine}
\label{SEC:BEHAVIOURS}

Controlling the actuators of an autonomous mobile robot is one of the
central aspects of mobile robot research. As \miro exposes the
interfaces to the motor controllers etc. of the mobile platform it
enables researchers to easily evaluate new approaches to model sensor
actor control loops. Be aware that at this point the latencies
introduced by various levels of the robot architecture can become a
critical issue. To eliminate the network latency and prevent yourself
from occasional network bandwidth problems we recommend that you run
your control programs collocated on the same computer with the service
that accesses the actuators device.

\miro supports the behavioural control paradigm introduced by Brooks
\cite{Brooks} by its own behaviour engine. It is designed to allow for
a quick start into behaviour robotics writing your own behaviours. Yet,
due to its open and extensible design it is also capable of handling
sophisticated control tasks as demonstrated by its use within the
RoboCup-scenario by our \sparrow robots.

\section{The Concept of Behaviours}

The basic idea of the behaviour approach to robot control is as
follows. Instead of the sense-plan-act paradigm of classical AI, the
task is splited into a set of reactive behaviours, that each try to
fulfill a small subtask of the problem set. For each of the tasks only
a very limited part of world modeling is needed (often, even raw
sensor readings are sufficient). By combining the output of the
various behaviours by an arbiter, the emergent higher level behaviour
of the system is achieved, solving the recommended high level task.

An accepted bottleneck of this approach is the arbitration and
calibration of large sets of behaviours necessary to fulfill different
aspects of a high level task. Therefore in the \miro framework the
behaviour engine also support the hierarchical decomposition of
behaviour sets by allowing them to be grouped in so called action
patterns that can be activated alternately by so called transition
messages. A set of action patterns is called a policy within the \miro
framework. 

There do exist two different kinds of transitions within the behaviour
framework: local and global transitions. Local transitions are emitted
by the behaviours by name. For example a transition named
``GoalReached'' might be emitted by a behaviour when it decides, it
has fulfilled its task. Within an action pattern a transition is
linked to the successor pattern, that will be activated as the
transition message occurs. That way totally different action patterns
can be activated, when the behaviour emits the ``GoalReached''
transition message within different action patterns. Global
transitions are somehow simpler. They contain directly the action
pattern, that is to be activated next. This mechanism is designed for
modules of the robot, that are external to the behaviour framework.
For example if a task planner decided it would be time for the robot
to go home, it might activate the ``GoHome'' pattern just by issuing a
global ``GoHome'' transition. The robot will then switch to its GoHome
pattern regardless of what it was doing before.

\section{Introductory Examples}

Lets look at the simple action pattern, who's high level task is to
explore the environment by performing a random walk. This is easily
splited into two distinct behaviours. The first subtask is not to
collide with the environment. This can be achieved by an avoid
behaviour, that reads the front sonar sensors to determine how far it
is away from the nearest obstacle. If the minimal distance is below
some threshold, it tells the arbiter to turn away into another
direction. The second behaviour would be a wander behaviour. It selects
from time to time just randomly some translational and rotational
velocity, making the robot move around. It does not have to care about
obstacles, since those are taken care of by the avoid behaviour.

The task of choosing the actual velocities to be applied to the motors
is performed by the arbiter. It therefore plays a central role in the
behaviour approach. There exist various kinds of arbiters, all
choosing different policies for this task. But a simple priority based
arbiter suffices for many scenarios. In \miro there currently just
exists a priority based arbiter, but since it is an extensible
framework you can easily plug in your own one. The priorities in this
example could be applied straight forward. The avoid behaviour has
higher priority as the wander behaviour.

Note the easy extensibility of this approach. For example, if we have
some bump sensors that indicate collisions with the environment. We
could just add another, let's call it ``EmergencyStop'' behaviour and
assign it the highest priority.  If one of the bumpers is pressed it
makes the robot stop and wait for rescue by one of the operators.

As a further extension we assume we have two tasks. The first is the
random walk described above, the second would be a wall following
behaviour, that simply is capable to drive the robot in a defined
distance along a wall. The decision what to do is provided by an
external source, say, a button located on the robot and pressed by the
operator when demonstrating the management the capabilities of the
newly bought autonomous mobile robot. The second action pattern looks
quite like the first one, except that the wander behaviour is
exchanged by a wall following behaviour, that drives the robot along
the wall. (Note that this behaviour does not have to care about walls
thar are blocking the way at the end of the corridor, since this can
be taken care of by the avoid behaviour.) These two action patterns
now form a policy. The pressing of the button sends a transition
message, that disables the currently running action pattern and
enables the other one.

The action pattern / transition message mechanism also fits naturally
for coupling reactive behavioural control with deliberative planning
architectures. The transition could also be raised by a path planner
and the corresponding action patterns could be 'move to point', 'drive
through door' or 'dock at power supply'. Indeed this was already done
within \cite{Hans-Diplom}.


\section{Example Usage}

As an example of the behaviour framework in use,
let us start the obstacle avoid demo.
\begin{itemize}
\item
  Start all the robot main services from within {\tt \$MIRO\_ROOT/bin/}
  with: \\
  {\tt PioneerBase} \\
  This server has to run on the robot.
\item
  Start the Behaviours, specifying the policy at the command line: \\
  {\tt Behaviours ../examples/behaviours/PpbStraight.xml} \\
  This client can run on any computer, however,
  it is a good idea to run the behaviours on the robot,
  to avoid that network latencies stop the control loop.
\item
  Start the policy controller GUI from within
  {\tt \$MIRO\_ROOT/bin/} with: \\
  {\tt PolicyController} \\
  to start and stop the policy run by Behaviours.
  Run this from any computer.
\end{itemize}

\subsection{The Behaviour Control Loops}

The core of the behaviour framework gets instantiated and initialized
in the file {\tt Behaviours.cpp} in {\tt
  \$MIRO\_ROOT/examples/behaviours/} with its {\tt main()} procedure.
It instantiates all available behaviours and arbiters starts a thread
for the timer service (ACE\_Reactor) needed by timed behaviours (see
section \ref{SEC:MIRO_TIMED_BEHAVIOUR}) and runs the CORBA event loop
for its BehaviourEngine interface, as well as for the event consumers
like event behaviours (see section \ref{SEC:MIRO_EVENT_BEHAVIOUR})
that get pushed by {\tt PioneerBase} for instance.  Behaviours
themselves do not have an own {\tt main} procedure.  They are
implemented in the {\tt action()} method of ordinary C++ objects which
have to be known in {\tt Behaviours.cpp}.  Necessarily, all behaviours
run on one computer.

\begin{figure}[!ht]
  \ifpdf
  \includegraphics[width=\textwidth]{../fig/Miro_behaviours.png}
  \else
  \includegraphics[width=\textwidth]{../fig/Miro_behaviours.eps}
  \fi
  \caption{The Behaviour Architecture Overview}
  \label{FIG_BEHAVIOUR_ARCHITECTURE}
\end{figure}


As seen in Fig.~\ref{FIG_BEHAVIOUR_ARCHITECTURE}, {\tt Behaviours.cpp}
runs two loops synchronously: the timer loop for timed behaviours and
the CORBA loop.
The timer loop is started in a separate thread (invoked by the command
"task$\rightarrow$open(NULL);").
It schedules the timed behaviours
by invoking their {\tt action()} methods one after another. 

The CORBA loop of {\tt Behaviours.cpp} (invoked by the command
"server.run(5);").  processes requests for the BehaviourEngine
interface (which it registers under the name "BehaviourEngine" at the
CORBA Naming Service) from the PolicyController, as well as it
administers all CORBA requests from {\tt PioneerBase} and invokes the
event behaviours.  The {\tt action()} method of an event behaviour is
invoked whenever the CORBA notification service publishes an event to
which the behaviour has subscribed.  Events are subscribed to by their
names, e.g. "Tactile" or "Sonar" (see {\tt TactileStop.cpp} or 
{\tt SonarNotify.cpp}). The published sensory events and their associated
payload are described in section \ref{SEC:ASYNCHRONOUS_SENSORY_INFORMATION}.

Invocation of an {\tt action()} method by the event "Tactile"
is reminiscent of a call to a server registered as "Tactile" by a client,
e.g.\ {\tt PioneerBase}. However, there are a few differences:
{\sl (i)} the event "Tactile" calls not a program with its own
{\tt main()}, but only an {\tt action()} method,
{\sl (ii)} not one unique method is invoked,
but all {\tt action()} methods which have subscribed to that event and
{\sl (iii)} unlike proper services, the TAO {\tt nslist} utility does
not list these event consumers,
because they cannot be meaningfully invoked by an external program.


The Behaviour Engine Implementation loads an XML Policy File into a
policy instance and starts and stops the policy. On activation of an
action pattern, all behaviour that are part of the action pattern get
connected to the event chanel or the timer respectively.

\section{Arbiters and Messages}

Arbiters select the action from all the suggestions the behaviours of
an action pattern make. The behaviours make a suggestions by sending
an arbitration message to the arbiter. Each arbiter has its own
arbitration message format, depending on the arbitration scheme used
and the actuatory device(s) it controls.

In \miro currently exist three arbiters. The {\tt MotionArbiter} used
in all examples, the specialized {\tt PowerArbiter}, that uses the
SparrowMotion interface and the experimental {\tt WindowArbiter}.

\subsection{MotionArbiter}

The arbiter acts only on the motor output of the behaviours:
instead of sending motor commands directly to the {\tt Motion} interface,
the behaviours invoke the method arbitrate() of the arbiter,
passing it a message.

The {\tt MotionArbiter} uses a simple fixed priority, winner
takes all, arbitration scheme. That is, each behaviour has a fixed
priority that corresponds with the order the behaviours are defined
within the action pattern. In the {\tt MotionArbiterMessage}, that is,
the parameter passed by the {\tt arbitrate()} call, the behaviour can
specify whether it wants to suggest the arbiter an action or not. The
action is defined by the translational and rotational velocity for the
motion. The arbiter then decides whether to transform the content of
the message into motor commands or whether to issue the motor commands
of another, conflicting behaviour which has a higher priority.

Note that the arbitration of messages before conversion into motor
commands, however, is only for the wheels of the robot by the {\tt
  MotionArbiter}.  The other actuators, for instance the camera
pan-tilt unit or the grippers, are to be controlled directly in this
case. No arbiter has to be defined for them, as long as a only single
behaviour is expected to control them.

\section{Implementing a Behaviour}

All behaviours are derived from the base class {\tt Miro::Behaviour}.
The central method of this class is the method {\tt action()}.
It is to be overwritten by the programmer
and has to contain the behavioural code.
The method becomes invoked by the behaviour control file,
i.e.\ {\tt Behaviours.cpp}, which includes the behaviour header files
and which links the behaviours to their corresponding control loops.
However, if only a new policy is combined from existing behaviours,
then {\tt Behaviours.cpp} does not need to be modified.

The behaviours' {\tt action()} method is expected to call an arbitration
method or send a transition message. Note the inversion of control flow.
A behaviour is not allowed to jump into some infinite loop,
but the {\tt action()} method is called continously
as long as the behaviour is active within some action pattern.
So every {\tt action()} method must return quickly enough after its
invocation to give time to the others. This implements the idea of
simple reactive behaviours rather than complex strategies.
If it is necessary to run a behaviour continuously in parallel to other
behaviours, then own threads have to be created, e.g. using ACE.

Only one instance of each behaviour object is used within the behaviour
engine, even for different action patterns.  Therefore data that needs
to be stored between subsequent {\tt action()} method invocations can
be easily stored within member variables.

From an implementation perspective, there are three kinds of
behaviours, depending on how the data, the behaviour bases its
decisions on, is delivered.

\subsection{Miro::TimedBehaviour}
\label{SEC:MIRO_TIMED_BEHAVIOUR}

This is the base class for a timer scheduled behaviour. A behaviour
derived from this class runs with all its brothers and sisters
cooperatively multi-threaded in one thread of control. The pace at
which its {\tt action()} method is called is selectable by a parameter of
the base class.

This is the most simple form of behaviour design and in many cases most
straight forward. It is especially suitable for behaviours that poll
their sensory information of the world model, like for instance the
current scan of the laser range finder for collision avoidance. Also
the above mentioned wander behaviour - which doesn't use any sensory
information would be best implemented as a child of the
{\tt Miro::TimedBehaviour} class.

See {\tt \$MIRO\_ROOT/examples/behaviours/simple/Wander.cpp|h} for a
complete source code example.

\subsection{Miro::EventBehaviour}
\label{SEC:MIRO_EVENT_BEHAVIOUR}

This is the base class behaviours using asynchronous sensory
information published by the Notification Service. It subscribes for
the events it likes to get pushed and the action method is called
whenever a new message arrives. A pointer to the current structured
event is then available as member variable.

The emergency stop behaviour described
above would be a good candidate for such a behaviour, since it would
have to poll excessively as a timed behaviour in order to minimize the
latency between a bumper pressing a the actual stop, while it would
only need to call an arbitration method in very rare occasions.

See {\tt \$MIRO\_ROOT/examples/behaviours/simple/TactileStop.cpp|h} for a
complete source code example.

\subsection{Miro::TaskBehaviour}
\label{SEC:MIRO_TASK_BEHAVIOUR}

Behaviours derived from this calls run within their own thread of
control, not blocking others even if they need fairly long for their
decisions. Note that such a behaviour is likely to be miss designed,
since it contradicts the behavioural approach to need excessive time
to come to a decision. But if you have need for something like this,
since you are doing something we didn't think of, this is the class to
base your behaviour on. Note however, that since behaviours are shut
down cooperatively, also a task behaviour is not allowed to loop
indefinitely within its action method. If the behaviour is still part
of the currently running action pattern, its action method will be
call immediately after giving control back to the behaviour framework.

\subsection{Behaviour Parameters}

Behaviours usually get used within different action patterns. But they
are often expected to behave slightly different within each
constellation. Therefore each behaviour has an associated {\tt
  BehaviourParameters} class which is designed to hold the different
parameter sets for the different use cases of a behaviour. These
parameter classes become initialized on startup of the behaviour
framework and are expected to be static and constant during the run of
an entire policy. (How to handle dynamic parameters like destination
coordinates of a 'move to position' behaviour is explained in section
\ref{SEC:DYNAMIC_PARAMETERS})

Behaviours, which subclass the {\tt BehaviourParameters} class,
need to have their own {\tt BehaviourParameters}
subclass factory methods, for dynamic instantiation. This is captured by
the two macros:
\begin{itemize}
\item BEHAVIOUR\_PARAMETERS\_FACTORY(X)
\item BEHAVIOUR\_PARAMETERS\_FACTORY\_IMPL(Y, X)
\end{itemize}
X denotes the name of the behaviour parameters class and Y denotes the
name of the behaviour class. The first macro has to be placed within
the class definition and the second within the cpp-file containing the
behaviour implementation.

The behaviour parameters classes are handled by the parameters
framework described in the previous chapter. Section
\ref{sec:BehaviourDescription} covers the details for parameter
classes for behaviours.

\subsection{Behaviour Initialization}
\label{sec:init}

Before an action pattern becomes activated all its behaviours {\tt
  init()} methods are called successively. This allows behaviours to
initialize their per task parameters (like destination coordinates) in
a convenient way. Note however, that the {\tt init()} method can be
called while the behaviour is already active (see section
\ref{sec:active-inactive}) and so its {\tt action()} method can be
concurrently running. Therefore a mutex is needed to avoid race
conditions. For an example see:

{\tt \$MIRO\_ROOT/examples/behaviours/simple/Timer.cpp|h}

Parameters that are valid for the whole lifetime of a behaviour, such
as references to the robots services or other objects within the
behaviours address space are best to be passed during construction
of the behaviour, forming a so called initializing constructor.

\subsection{Behaviour Activation and Deactivation}
\label{sec:active-inactive}

When a behaviour is to become active due to it being part of an action
pattern its {\tt open()} method is called. If the behaviour is no longer
part of the next to be running action pattern its {\tt close()} method is
called. If a transition from one action pattern to another is
performed and the behaviour is part of both behaviour sets, then no
calls to {\tt close()} and {\tt open()} methods will be issued. Only the
behaviours {\tt init()} method will be called, to allow it to update its
parameter sets. This is useful to avoid unnecessary behaviour
shutdown.

Note also, that the {\tt close()} method can be called while the behaviour
is concurrently within the {\tt action()} method. --- Its call to the
arbiter will then just be ignored. On the call of {\tt open()} however it
is guaranteed that the behaviour currently is not running.

\subsection{Changing Behaviour Parameters within an Action Pattern}
\label{SEC:DYNAMIC_PARAMETERS}

Another way for interaction with the behaviour framework from outside
is by changing the parameters of a behaviour within an action pattern
during execution of the behaviour engine. In this case the behaviour
name as well as the action pattern it belongs to has to be known. You
can then query as well as set the behaviour parameters instance via
the interface of the policy.

However there are some issues that have to be take care of. First, the
type of the derived behaviour parameters struct has to be known, as
you get a pointer to a BehaviourParameters instance, that has to be
down casted appropriately. Second, to avoid memory leaks and
segmentation faults, the memory handling has to be understood. When
querying the parameters, the caller takes ownership of the object
returned. When setting the parameters, the action pattern takes
ownership of the passed parameters instance. - Therefore the object
has to be allocated on the heap (that is, with {\tt new}). But
actually that's not too much of a problem, if you stick to the
intended protocol: 
If you first retrieve the parameters instance by querying the policy,
change the parameters and write them back by the set method,
everything works okay.

A similar interface exists for the behaviours themselves. But they only
can query and set the parameters of behaviours in action patterns that
are linked to the current action pattern by a local transition. The
behaviours can therefore only query the parameters by the transition
name instead of the action pattern name.

\section{Arbiters}

The arbiter framework is similar to the behaviour classes. It will be
explained in more detail as soon as we do some more work on arbiters.

\section{Building Action Patterns}

``Now I built all my behaviours. What code do I have to write to make
them an action pattern?'' Well, you don't have to write code. Action
patterns and policies are defined within an XML file. Allowing for
fast and convenient modifications. Which is especially cool when
debugging. An Instance of each arbiter and behaviour has to be
registered within repositories. Afterwards, the policy can configure
the action patterns and transitions by parsing the XML file. 
To make live easier, the editing of action patterns, transitions and
behaviour parameters can also be done using the GUI based PolicyEditor
(see section \ref{SEC:POLICY_EDITOR}), but
let us first take a look at how those things are put together in the
XML syntax.

\subsection{The Policy File}

A policy file can have the following tags and attributes:

$<$!--MiroPolicyDocument--$>$

\begin{description}
  \item[policy]
    One file always describes one entire policy. A policy defines a
    state machine that consists of a set of action patterns. Each
    action pattern describing one state of the machine.
  
  \begin{description}
  \item[actionpattern]
    An action pattern describes a set of behaviours, that can run
    simultaneously and produce some emergent higher level behaviour.
    

    Attributes:
    \begin{itemize}
    \item \textit{name} The name of the action pattern.  
    \item \textit{start (true/false)} The action pattern to be active
      at startup is to be marked true, the others false or unmarked.
    \item \textit{x, y} For GUI purpose, just ignore them.
    \end{itemize}
  
    \begin{description}
    \item[behaviour]
      That's what all the chapter is about. Read it again.

      Attributes: 
      \begin{itemize}
      \item \textit{name} Name of the behaviour.
      \end{itemize}  

      \begin{description}
      \item[parameter]
        Behaviours parameters can be specified within the policy file
        for each action pattern. 

        Attributes: 
        \begin{itemize}
        \item \textit{name} Name of the parameter.  
        \item \textit{value} The value of the parameter.
        \end{itemize}
      
      \end{description}  

    \item[transition]
      Message that triggers a transition to another
      action pattern.  

      Attributes: 
      \begin{itemize}
      \item \textit{target} The action pattern to activate next.
      \end{itemize}  

    \item[arbiter]
      To decide which action to choose from the different outputs of
      the different behaviours one needs to arbitrate one way or the other.

      Attributes: 
      \begin{itemize}
      \item \textit{name} The name of the arbiter to use.
      \end{itemize}
    \end{description}
  \end{description}
\end{description}

\subsection{The Repositories}

A policy can be built on the basis of an XML description. For the
parser of this description to be able to construct the policy, it has
to be able to refer to instances of behaviours and arbiters that are
mentioned within the XML file by their name.  For this purpose a {\tt
  Miro::BehaviourRepository} and a {\tt Miro::ArbiterRepository} class
do exist.  At these Repositories an instance of each behaviour and
arbiter has to be registered. The name of the behaviour has to be
reported by the {\tt behaviourName()} method.  Since we only need one
instance of each of these repositories, there do exist a global
instance of each. A pointer to such an instance can be obtained by
the classes static method {\tt instance()}.

\subsection{The Behaviour Factory}

All the initialization stuff necessary before constructing an instance
of the Policy class can be done within the main function of your
program. This task consists mostly of obtaining the needed object
references, instancing
behaviours and arbiters and registering them at their respective
repositories.  This can be bundled within a so called behaviour factory
class as can be seen in the behaviour example at:

{\tt \$MIRO\_ROOT/examples/behaviours/simple/BehaviourFactory.cpp|h}

It uses the simple base class:

{\tt \$MIRO\_ROOT/src/miro/BAFactory.cpp|h}

\subsection{The Behaviour Engine}

There does exist a small CORBA based interface for interaction with
the behaviour framework. It is the {\tt BehaviourEngine} interface and
the {\tt PolicyController} (section \ref{SEC:POLICY_CONTROLLER}) is a
simple GUI based client for this interface. To allow your own
behaviour sets to be controlled via this interface, the policy
controller implementation has to be instantiated within your
executable. This can be seen in the example at:

{\tt \$MIRO\_ROOT/examples/behaviours/engine/SimpleBehaviourEngine.cpp}

\section{The Policy Editor}
\label{SEC:POLICY_EDITOR}

Editing large XML files is tedious and error prone. Therefore \miro
offers an GUI based editor, with which you can build and edit your
policies. You can also edit the parameters defined within the
Parameters classes of your behaviours.

The policy editor can be started with no parameters or with the policy
file to edit as first parameter. It has a menu bar and displays the
policy graph within the big scroll view area of the program.

\subsection{The Menu}

\subsubsection{File}
\begin{itemize}
\item New \\
  Create a new policy.
\item Open ... \\
  Open a policy file. The file dialog will be displayed.
\item Save \\
  Save the current policy to disc. If no file name is yet specified,
  Save as will be invoked instead.
\item Save as ... \\
  Save the current policy under a new file name. The file dialog will be
  displayed.
\item Send to ... \\
  Send the current policy to a robot. A dialog box will be displayed to
  enter the robots name (the naming context, its behaviour engine is
  registered at). The behaviours and arbiters within the policy have to
  be present within the behaviour and arbiter repositories in the
  behaviour engine running on the robot. - Otherwise the parsing will
  fail.
\item Quit \\
  Quit the policy editor.
\end{itemize}
\subsubsection{Options}
\begin{itemize}
\item Behaviour Descriptions \\
  Display the behaviour descriptions dialog. The dialog consist of a
  listbox with the currently loaded behaviour parameter description
  files. (See the chapter \ref{SEC:PARAMETERS} for more details on
  parameter files.) Buttons for adding and deleting files are
  provided. On pressing the button add, the file dialog will be
  displayed. Note that the selected file has to contain a valid
  behaviour description. - Verifying, whether the load was successful
  is currently only done on leaving the dialog. On the positive side
  the currently loaded description is memorized in the file
  .PolicyEditorConfig.xml file in the users home directory for further
  runs of the policy editor.
\end{itemize}
\subsubsection{Help}
Currently this menu only contains two about dialogs.

\subsection{Editing the Policy Graph}

Clicking with the right mouse button into the editing area produces
a popup menu with the option to place a new action pattern. Enter the
name of the new pattern in the dialog box.

Clicking with the right mouse button into the name of an action
pattern produces a popup menu with the following options.
\begin{itemize}
\item Start pattern \\
  Select the pattern to be the start pattern. That is the pattern, the
  policy will begin with by default. Every policy has to have exactly
  one start pattern. The start pattern is marked with two asterix
  around the pattern name.
\item Add behaviour \\
  Add a behaviour to the action pattern. A list of available behaviour
  will pop up. Every behaviour can be instantiated only once per
  action pattern.
\item Add transition \\
  Click on another action pattern to link the two pattern with a
  transition. A dialog box will appear to enter the transition name.
\item Rename pattern \\
  A dialog box will appear to enter the new name of the action
  pattern.
\item Rename transition \\
  A dialog box will appear with the name and the target of the
  transitions. You can also delete a transition by deleting the name
  or the target of the transition. - Be careful, the new entry
  will only be accepted, if you left the entry field by clicking on
  another entry before hitting the okay button.
\item Delete pattern \\
  Deletes the pattern.
\end{itemize}

Clicking with the right mouse button into the green area of an action
pattern produces a popup menu with the following options.
\begin{itemize}
\item Set arbiter \\
  Select the arbiter for to the action pattern. A list of available arbiters
  will pop up. Every action pattern has to have an arbiter.
\item Delete arbiter \\
  Deletes the arbiter from the action pattern.
\end{itemize}

Clicking with the right mouse button onto a behaviour within an action
pattern produces a popup menu with the following options.
\begin{itemize}
\item Up \\
  Move the behaviour up one position in the list of behaviours. The
  position in the behaviour list represents the priority of the
  behaviour at the arbiter. - The higher, the more important.
\item Down \\
  Move the behaviour down one position in the list of behaviours.
\item Set parameters \\
  The parameters dialog will be displayed. It has three columns. In
  the left one, the parameter names are displayed as label. The middle column
  contains a single line entry field for the parameter. In the right column the
  measure of the parameter is displayed as label, if no measure is available
  the parameters type is displayed. The default value is
  displayed as bubble help associated with each entry field, if
  available. The parameters type is displayed as bubble help
  associated with each measure label. The dialog is strictly
  typed. Therefore the okay button is inactive if any of the entry fields
  are containing invalid input, for instance a value bigger than 180
  for an Angle type.
\item Delete behaviour \\
  Deletes the behaviour from the action pattern.
\end{itemize}

\subsection{Describing the Available Behaviours}
\label{SEC:BehaviourDescription}

To be able to use your own behaviours within the PolicyEditor, you
have to describe their properties (name and parameters) within a
parameters description file. The parameter framework is explained in
more detail in the previous chapter. Anyhow there are some minor
restrictions for behaviour parameter description files, that are
explained below:

$<$!--MiroParametersConfigDocument--$>$

\begin{description}
  \item[config]
  \begin{description}
  \item[config\_group]

    Attributes:
    \begin{itemize}
    \item \textit{name} The name of the config group. Within the
      behaviour parameters framework the valid names are \textbf{behaviour} and
      \textbf{arbiter}. Within each group only config items of the
      specified type are allowed to occur. Other groups are processed
      by the parameter auto code generation tool {\tt MakeParams} but
      ignored by the {\ PolicyEditor}.
    \end{itemize}

    \begin{description}
    \item[config\_item]

      Attributes:
      \begin{itemize}
      \item \textit{name} The name of the behaviour or the arbiter.
      \item \textit{parent} The name of the super class. That has to
        correspond to the corresponding super class of the behaviour.
      \item \textit{namespace}
      \end{itemize}
      \begin{description}
      \item[config\_parameter]

        Attributes:
        \begin{itemize}
        \item \textit{name}
        \item \textit{type} No nested types are allowed!
        \item \textit{default}
        \item \textit{measure}
        \item \textit{inherited}
        \end{itemize}
      \end{description}
    \end{description}
  \end{description}
\end{description}

\subsection{Example Behaviour Description File}

\subsection{Auto-generating Parameter Class Code}
\label{sec:makeParams}

Having written the description of your behaviours parameters within
XML, it is possible to auto-generate the necessary code for your
behaviours associated Parameters class. See the previous chapter for
details.

\section{Policy Controller}
\label{SEC:POLICY_CONTROLLER}

The policy controller is some kind of remote control for the behaviour
engine. It consists of a panel with four buttons and a menu bar. The
buttons allow to start, stop, suspend and resume the behaviour engine.

\subsection{The Menu}

\subsubsection{File}
\begin{itemize}
\item Connect robot \\
  Connect the policy controller with a behaviour
  engine of a robot. A dialog box will be displayed to enter the
  robots name (the naming context, its behaviour engine is registered
  at).
\item Load policy \\
  Load a policy file into the behaviour engine. A dialog box will be
  displayed to enter the file name. The file has to be locally
  accessible by the robots behaviour engine.
\item Send policy \\
  Send a policy to the robot.  The file dialog will be displayed to
  select the policy. The behaviours and arbiters within the policy
  have to be present within the behaviour and arbiter repositories in
  the behaviour engine running on the robot. - Otherwise the parsing
  will fail.
\item Quit \\
  Quit the policy editor.
\end{itemize}

\subsubsection{Edit}
\begin{itemize}
\item Send transition \\
  Send a transition to the behaviour engine. A dialog box will be
  displayed to enter the transition name. This is for debug purposes
  only, as the transition will only succeed if it is registered at the
  currently active action pattern - which in turn can change any time.
  There will be also no feedback whether the transition succeeded or
  not.
\item Send global transition \\
  Send a global transition to the behaviour engine. A dialog box will be
  displayed to enter the action pattern to activate. The action
  pattern has to be present within the current policy, otherwise the
  transition will fail.
\end{itemize}

%%% Local Variables: 
%%% mode: latex
%%% TeX-master: "miro_manual"
%%% End: 
