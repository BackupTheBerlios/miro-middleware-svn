\chapter{Parameter Sets}
\label{SEC:PARAMETERS}

                                % problem

Robot control programs and robotic algorithms are full of parameters.
Different robot types vary in shape and sensor configuration, but also
different robots of the same type tend to vary slightly. For instance
some of them are equipped with further sensory or actuatory devices
not present in the standard configuration. Also damage and repair of
parts of a robot during its lifetime result in an individual robot,
which has its own unique configuration.

                                % examples

Furthermore the environment provides a complete set of
parameters. Lighting conditions, the shape and location of rooms and
corridors, the position of obstacles like tables or stairs. The list
could be continued indefinitely. These parameters vary between
different environment, but also tend to change slightly over
time. Light bulbs are changed, cupboards added, tables moved etc.

A third source of parameters are defined by the task, the robot has to
perform or the scenario, the robot is designed to operate in.
Generalized tool boxes like planners or knowledge bases have to be
populated with the actions and objects of relevance and a reactive
execution engine need to be configured with what actions to take in
which situation.

                                % summary

The configuration of these parameter sets can be done either by hand
or tool supported or fully autonomously. Nevertheless it is a crucial
part of the setup of robot for every scenario. The handling of those
parameter sets therefore has to be well defined in order to keep the
robots control software maintainable and adaptable to new tasks.

                                % solution aproach

In this chapter we will therfore introduce the parameter framework of
\miro. The parameter framework was primarly designed to handle the
first source of parameters, namely the configuration of the robot and
its sensory and actuatory services for a task.  But meanwhile it is
also used for the third part of the above sketched three sources of
parameter sets: With slight extensions, it is used for the
parameterization and configuration management of the reactive
execution engine (the BAP framework) introduced in sectin
\ref{sec:BAP}. Environmental modeling for different scenarios lies
out of the scope of the problem set currently addressed by \miro.

                                % organization of the chapter
The remaining of this chapter is organized as follows.  First the
general concept of parameter sets and configuration files is sketched.
Then the location and naming of the robot configuration files of \miro
and how to create a configuration for an individual robot will be
explained. In the following section the description and implementation
of parameter sets and the mapping to the actual parameter files will
be discussed. In section \ref{SEC:DESCRIPTION_SYNTAX} the syntax and
semantics of the description language will be explained.

\section{Overview}

Parameter sets are a quite powerfull concept within \miro, that
therfore also encapsulates a lot of functionality. To facilitate the
usage of the parameter framework of miro, tools support the handling
of parameter sets for the different user communities. In this section
we will describe the usage of parameter sets from the end-user as well
as from the programmers perspective

\subsection{End-User Perspective}
 
An end-user has to handle parameter files for its robots. A parameter
file contains various parameter sets, thatare described by an XML
syntax. The server parses those files on startup, to initialize the
parameter sets of its services. Parameters in general are strictly
typed name value pairs.  They are grouped into sections within the
file. They can be nested, structured types, including arrays of
variable length as well as sets. The parameter names start with an
upper case initial letter.

To edit parameter files, the GUI-based \texttt{ConfigEditor} can be
used. It enforces the XML-syntax as well as the correctness of the
configuration section placement and the type correctness. It is
discussed in more detail in section \ref{sec:ConfigEditor}.

\subsection{Programmers Perspecitve}

From a programmers perspective, a parameter set is represented by a
struct within the target programming language. Currently only C++ is
supported as target. For this language, a parameter set is translated
into a class with all member variables declared public, just like
a C struct. The member variable names begin with a lower case initial
letter. Names consisting of multiple words are concatenated, the
beginning of a new word is indicated by an upper case letter.
Parameter classes support single, public inheritance. Apart from the
member variables, the following methods are declared:
\begin{description}
\item[Default Constructor]
  Initializes all member variables to their default values.
\item[void operator $\ll =$ (QDomNode)]
  Parses an XML node and sets the member variables accordingly.
\item[QDomElement operator $\gg =$ (QDomElement)]
  Appends the current
  values of the member variables as a child to the XML node.
\item[void printToStream(ostream\&)]
  Prints the current values of the member variables to the
  specified output stream.
\item[std::ostream\& operator$\ll$ (std::ostream\&, Miro::Parameters\&
  const)] is declared for the parameter parent class.
\end{description}
Optionally parameter set classes contain a static member method
{\tt instance()}, that returns a pointer to the global heap allocated
instance of the parameter set class (see also singleton pattern, and
double checked locking \cite{Schmidth:}).

As parameter set classes have a very generic structure, their coding,
including the XML and ostream operators can be automated. Therefore
parameter sets normally are described not within a programming
language, but within an XML syntax. The mapping to target programming
language is then performed by a small compiler, which emits the
necessary (i.e. C++) code.  The parameter set descriptions are also
used by the tools for GUI-based configuration editing. Those are the
\texttt{ConfigEditor} as well as the \texttt{PolicyEditor} of the
behaviour framework.

\section{Configuration Files}

Every service of \miro tries to find a configuration file on startup
to adapt itself to the individual robots configuration if necessary.
If no configuration file is found, the service has to use its default
settings. The locations the configuration files are expected are:
\begin{enumerate}
\item {\tt \$MIRO\_ROOT/etc}
\item The current directory.
\end{enumerate}
The name of the configuration file is expected to be the host name of
the computer the service is running on. The file extension is .xml:

{\tt \$HOST.xml}

For example in the case of our famous Sparrow99 soccer robot goal
keeper named haeltnix, the \miro service SparrowBase, started on the
computer mounted on the robot will try to load the configuration file:

{\tt \$MIRO\_ROOT/etc/haeltnix.xml}

The default name and location of the configuration file can be
overwritten by the command line option {\tt -MiroConfigFile} {\em
  filename} or {\tt -MCF} for short. The search order of directories
stays the same.  Additionally there can also be an absolute path
specified.

In {\tt \$MIRO\_ROOT/etc} there are sample configuration files for
every robot model currently supported by \miro. They are named
{\em model}.{\tt xml}. Copy the respective configuration file of 
your robot model to {\tt \$HOST.xml} to get a configuration file to 
start with. XML files are text files and quite easily
readable. Therefore opening the configuration file into your favorite
editor and changing the {\em value} attributes to different settings
should be sufficient to solve the first few configuration problems
like differing device names etc. For anything else, read on.

\subsection{Configuration File Syntax}

The available tags and there nesting capabilities look like follows:

$<$!--MiroConfigurationDocument--$>$

\begin{description}
\item[configuration]
  One file always describes one entire configuration. A
  configuration describes all the services parameters, that can be
  executed on the particular robot.
  
  There can only be one configuration tag per file.
  
  \begin{description}
  \item[section] Each component of \miro can have its own section
    within a configuration file. Configuration file sections are used
    to group services of similar scope. For instance the vision
    system, the configuration of the filter tree, as well as the
    the specification camera paramters, is considered to
    be a component within a \miro configuration file. Normally each
    service defines its own section within the configuration file.
    
    There are multiple section tags permitted within a configuration.
    
    Attributes:
    \begin{itemize}
    \item \textit{name} The name of the section (required).
    \end{itemize}
    
    \begin{description}
    \item[instance]Parameters of a component are grouped and
      described as set of parameters that belong together. Note that
      within the configuration file there is a hardware centric
      view taken. That is, sensors and actuators that result in multiple
      service interfaces, are most likely to be grouped into one
      component if they are controlled by the same low level
      controller board. For instance the parameters for the pioneer
      board, that controls the motors as well as the sonars etc., are
      grouped within just one component.
      
      An instance of a certain parameter set can be
      specified by the type and name attributes of the instance
      tag. There are multiple instance tags permitted within a
      section.
      
      Attributes: 
      \begin{itemize}
      \item \textit{name} Name of the parameter (required).
      \item \textit{type} The type name of the parameter. The
        parameter type definitions are described in section
        \ref{SEC:PARAMETER_TYPES}.
      \end{itemize}
    \end{description}
    
    \begin{description}
    \item[parameter] Parameters can be specified within the instance
      tag, for each available parameter of described the type.
        
      There are multiple nested parameter tags permitted, but each has
      to hold a name attribute that is unique within the enclosing tag.
        
        Attributes: 
        \begin{itemize}
        \item \textit{name} Name of the parameter.  
        \item \textit{value} The value of the parameter.
        \end{itemize}
        
        In case the parameter describes a structured type instead of a
        basic type (like bool, int, string etc.), parameter tags can
        be nested. In this case the value attribute is ignored.
        
      \end{description}  
  \item[parameter] For many parameter types there is a default
    instance available. Especially if there can only be one
    parameter set of this type per robot (i.e. the Robot
    parameters set, that holds the name of the robot etc.). This
    default instance is not referenced by the \texttt{instance}
    tag, but by a \texttt{parameter} tag, that holds the type
    name of the parameter set as its \texttt{name} attribute.
    
    There are multiple parameter tags permitted within a section.
    Apart from the unified \texttt{type/name} attribute, a parameter
    tag behaves just like an \texttt{instance} tag, and can hold
    multiple, probably nested \texttt{parameter} tags.
    
    Attributes: 
    \begin{itemize}
    \item \textit{name} Type name of the default parameter instance.
    \end{itemize}  
  \end{description}
\end{description}

\subsection{Example Configuration}

See {\tt \$MIRO\_ROOT/examples/params/TestConfig.xml}
for a small initial example.

TODO: include and comment the example code here.

Section tags are structurising elements and component instance tags
are mainly defined by the underlying hardware devices. But what
defines which parameters are accepted within by which component and of
what types they are? While the configuration files contain comments to
explain the purpose of the different parameters, there do also exist
formal definitions of the components parameter sets, as described
within the next section.

\section{Parameter Set Generation}
\label{SEC:PARAMETER_TYPES}

Parameter sets, described as component parameters within a
configuration file, correspond to strictly typed classes within C++.
Those structures can be initialized by parsing the configuration file.
To facilitate the generation of parameter sets, parameter classes
needn't to be hand coded, but are auto-generated from parameter
description files by a little parameter class compiler ({\tt
  MakeParams}). The parameter description files are once again based
on the XML syntax.

Furthermore the description files are not only used for the
configuration of the services, but also by other tools within
the \miro framework. The configuration editor used this information
for the generic editing of configuration files. Additionally, 
the behaviour framework (sec.\ 
\ref{SEC:BEHAVIOURS}) uses this functionality for parameterization of
behaviours and arbiters and the {\tt PolicyEditor} (sec.\ 
\ref{SEC:POLICY_EDITOR}  use the parameter description files for GUI
based editing of the behaviour parameters). To facilitate the reuse of
the parameter descriptions by multiple tools, the syntax of the file
is somehow obfuscated by additional tags and attributes.

\subsection{Description File Syntax}
\label{SEC:DESCRIPTION_SYNTAX}

Parameter classes support single inheritance.

$<$!--MiroParametersConfigDocument--$>$

\begin{description}
\item[config]
  A parameters config file describes the types, names and defaults
  for parameter classes as used within the \miro framework. 

  \begin{description}
  \item[config\_group] Configurations can be grouped together for
    various reasons.  A group within a description file in general
    corresponds to a section within the configuration file. Within a
    configuration file, a parameter set can only be instanciated
    within the section, named as the config group.

    Attributes:
    \begin{itemize}
    \item \textit{name} The name of the config group.
    \end{itemize}

    \begin{description}
    \item[config\_item] A configurable item to built a parameters
      class for. Each config item will be mapped to a parameter class
      by the compiler.

      Attributes:
      \begin{itemize}
      \item \textit{name} The name of the parameter class. The suffix
        Parameters will be added to the name.
      \item \textit{parent} The name of the super class. The suffix
        Parameters will be added to the name. Note that namespaces
        have to be fully qualified here, i they differ from the
        current namespace.
      \item \textit{instance} Whether or not a singleton instance
        of the parameter class shall be created (sec.\
        \cite{SINGLETON}). Default is {\em false}.
      \item \textit{final} Corresponds to the finalize keyword of
        java. It is only relevant for the editing tools, not for code
        generation. It indicates, that a toplevel instance of this
        parameter can be created.Default is {\em true}.
      \item \textit{dummy} Sometimes you have to fake a parameter
        class to the editing tools that does not exist as such. No
        code will be generated for such a class. Default is {\em
          false}.

      \end{itemize}
      
      \begin{description}
      \item[config\_parameter] A parameter for the class.

        Attributes:
        \begin{itemize}
        \item \textit{name} Guess what.
        \item \textit{type} Allowed basic types are: bool, char, int,
          unsigned int, short, unsigned short, long, unsigned long,
          double and std::string. There are two more
          predefined types: Angle and angle. {\em Angle} corresponds
          to a Miro::Angle instance. It is an angle $\phi$ normalized
          to $-\pi \le \phi \le \pi$ while {\em angle} corresponds to
          a none normalized angle represented by a simple double. Note
          that while the internal representation of angles is in
          radiant, the values specified in the configuration files are
          in degrees. But you can also specify parameter
          classes described within a parameter description file as
          parameter type. Additionally, {\em std::vector$<>$} and {\em
            std::set$<>$} are supported. Note
          however that we are currently lacking support for specifying
          default values for these none basic types.
        \item \textit{default} Default value as set in the
          constructor (optional).
        \item \textit{measure} The measure the type represents, such
          as: mm, msec, �, mm/s, �/s (optional).
        \item \textit{inherited true/false}
          This attribute is used to 
          overwrite defaults for inherited variables within the
          constructor of the class. If set to true, no new member
          variable will be added to the class, and only the
          attributes name and default will be evaluated.
        \end{itemize}
      \end{description}
    \end{description}
  \item[config\_global]
    Global configuration items those are mainly used for code
    generation. They specify things, that the compiler can not easily
    derive autonomously, such as additional include directives, etc.

    Attributes:
    \begin{itemize}
    \item \textit{name} The name of the config item. There are
      currently three items supported:
      \begin{description}
      \item[namespace] Specifies the namespace of the generated
        code. While multiple parameter classes can be described
        within one file, they all have to reside within the same
        namespace.
      \item[include] Specifies a user include directive.
      \item[Include] Specifies a system include directive.
      \item \textit{value} The name of the namespace, the local or
        global include directive.
      \end{description}
    \end{itemize}
  \end{description}
\end{description}

\subsection{Example Description}

See {\tt \$MIRO\_ROOT/examples/params/Parameters.xml}
for a small initial example.

TODO: include and comment the example code here.

\subsection{MakeParams}

The parameter description compiler has the following command line
parameters.
\begin{description}
  \item[-f $<$file$>$] name of the input file (default is {\em Parameters.xml})
  \item[-n $<$name$>$] base name of the output file (default is {\em Parameters})
  \item[-s $<$extension$>$] extension of the generated source file (default is {\em cpp})
  \item[-h $<$extension$>$]extension of the generated header file (default is {\em h})
  \item[-v] verbose mode
  \item[-?] help: emit this text and stop
\end{description}

\subsection{Example Header File}

See {\tt \$MIRO\_ROOT/examples/params/Parameters.h}
for a small initial example.

TODO: include and comment the example code here.

\subsection{Makefile magic}

To enable automatic generation of the C++ source and header files from
XML parameter descriptions, there does allready exist a rule within
the make files of \miro. Therefore, you can simply add the XML based
description file to the \texttt{target\_SOURCES} variable in the
corresponding \texttt{Makefile.am}. Make sure however, that it is
placed before the first source file, that includes the generated
header file, to ensure its timely generation. Furthermore the
resulting source and header files have to be added to the
BUILT\_SOURCES variable.

\section{Configuration Management Runtime Environment}
\ref{sec:ConfigRuntime}


\section{Config File Editor}
\ref{SEC:CONFIG_EDITOR}

For advanced config file editing, like filter trees of the video image
processing framework \ref{SEC:VIDEO_IMAGE_PROCESSING}, a simple text
editor is a fairly errorprone tool. XML editors do a better job in
this case, but for real comfort, \miro provides the config file
editor.

%%% Local Variables: 
%%% mode: latex
%%% TeX-master: "miro_manual"
%%% End: 
