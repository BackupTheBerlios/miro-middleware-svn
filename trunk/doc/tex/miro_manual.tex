% detect interpreter: pdflatex or latex
\newif\ifpdf
\ifx\pdfoutput\undefined
\pdffalse
\else
\pdftrue
\fi

\documentclass[10pt]{book}

\usepackage{graphicx} % enhanced graphics usage
\usepackage{a4} % A4 paperformat
\usepackage{isolatin1} % for input of 8 bits character
\usepackage{makeidx} % enable indexing
\usepackage{verbatim} % better verbatim environment
\usepackage{epsfig}
\usepackage{xspace} % add extra space at the end of the word if necessary
\usepackage{color} % provides standard LaTeX colors
\usepackage{fancyhdr}
\usepackage{float} % float environment enhancements
\usepackage{alltt} % defines alltt environment which is like verbatim, but allows some additional formating
\usepackage{listings} % pritty print of code-listings
\ifpdf
% this package should be loaded last
\usepackage[colorlinks=true,
            pdfstartview=FitV,
            linkcolor=blue,
            citecolor=blue,
            urlcolor=blue]{hyperref}
\fi

\graphicspath{{../fig/}}

%% insert the common definitions
\usepackage{a4} % A4 paperformat
\usepackage{isolatin1} % for input of 8 bits character
\usepackage{makeidx} % enable indexing
\usepackage{verbatim} % better verbatim environment
\usepackage{epsfig}
\usepackage{xspace} % add extra space at the end of the word if necessary
\usepackage{color} % provides standard LaTeX colors
\usepackage{fancyhdr}
\usepackage{float} % float environment enhancements
\usepackage{alltt} % defines alltt environment which is like verbatim, but allows some additional formating
\usepackage{listings} % pritty print of code-listings

%\usepackage[german]{babel} % multilingual package
%\usepackage[normal]{subfigure} % support for the inclusion of small subfigures
%\usepackage{latexsym} % some extra mathematical symbols
%\usepackage{exscale} % implements scaling of the `cmex' fonts


% configure the listings-environment
\lstloadlanguages{C++}
\lstset{
  basicstyle=\sf,
  commentstyle=\hfill \sl,
  language=C++,
  defaultdialect=ANSI,
  flexiblecolumns=true,
  indent=\parindent,
  extendedchars
  %%  texcl
  %% formfeed={}
}


% command definitions

% logo of the miro-software writing
\newcommand{\miro}{\textit{Miro}\xspace}
% logo of the sparrow writing
\newcommand{\sparrow}{\textit{Sparrow-99}\xspace}
% logo of the b21 writing
\newcommand{\bXXI}{\textit{B21}\xspace}


% font definitions
\newcommand{\Xbombastic}{\fontfamily{ppl}\fontseries{b}\fontsize{1.5in}{1.8in}\selectfont}


% color definitions
\definecolor{LightGrey}{rgb}{0.9,0.9,0.9}


% page definitions
\setlength{\parindent}{0pt} % Keine Absatzeinr�ckung
\setlength{\parskip}{7pt plus 1pt minus 1pt} % Absatz Abstand 7pt


% misc
\makeindex
\bibliographystyle{plain}

%=============================== Examples ====================================

%% code-parts as an extra paragraph
%% \begin{lstlisting}[frame=tb]{}
%% for (unsigned int i = 0; i < sonarScan.length(); ++i)
%%   cout << sonarScan.[i] << " ";
%% \end{lstlisting}

%% Included code-parts from a file
%% \lstinputlisting[frame=tb]{simpleTestClient.cc}

%=============================== Examples (end) ====================================

%%% Local Variables:  
%%% mode: latex
%%% TeX-master: "miro_manual"
%%% End: 


%% special definitions
\setcounter{tocdepth}{3}

%% =============================================================================

\begin{document}
%\maketitle

\thispagestyle{empty}
\begin{center}
  \vfill
  \includegraphics[width=5in]{signature}\\
  \vspace{30 mm}
  {\Xbombastic \miro}\\
  \vspace{10 mm}
  {\LARGE \textbf{Manual}}\\
  \vspace{10 mm}
  \textbf{Version 0.9.4}\\
  \vspace{10 mm}
  \today\\
  \vfill
\end{center}

\newpage

\begin{center}
  \includegraphics[width=5in]{pincolla}
  \bigskip
  For more paintings, see {\tt http://www.bcn.fjmiro.es/}
\end{center}

\pagestyle{headings}

\newpage
\tableofcontents

%% =============================================================================

\chapter{Introduction}

\miro is a distributed object oriented framework for mobile robot
control, based on CORBA (Common Object Request Broker
Architecture) \cite{OMG:CORBA-standart} technology. The \miro core
components have been developed in C++ for Linux. But due to the
programming language independency of CORBA further components can be
written in any language and on any platform that provides CORBA
implementations.

The \miro core components have been developed under the aid of ACE
(Adaptive Communications Environment) \cite{ACE-Paper, ACE-Manual}, an
object oriented multi-platform framework for OS-independent
interprocess, network and real time communication. They use TAO (The
ACE ORB) \cite{TAO-paper} as their ORB (Object Request Broker), a
CORBA implementation designed for high performance and real time
applications. Therefore \miro should be easily portable to any other
OS, where ACE and TAO run on.  These are many Unix clones, Windows NT
and some real time operating systems.

\miro is currently available for the RWI B21 platform, the Activmedia
Pioneer 1, and the \sparrow architecture developed at the University
of Ulm, we are convinced, that other ports can be done straight
forward.

\miro was built since the existing robot control architectures didn't
suffice our needs of usability, reliability, scalability and
portability.

We used C++ due to its advantages in big projects, since it was
especially designed for big projects. We have learned the hard way,
that this is a serious requirement, for projects like a mobile robot
control architecture.

We use multi processing, multithreading and the CORBA technology since
most robotics applications are inherently concurrent and distributed.
The hardware devices, like sensors and actuators run concurrently and
due to the constant lack of computing power especially in computer
vision, tend to reside on multiple computers. And as soon as
cooperative behavior of multiple autonomous robots is of interest, the
system as a whole is distributed anyways.

%We used an object oriented approach, since real live objects are
%concerned. And on the other hand, OO is just state of the art and you
%need good arguments to start a new project ignoring object
%orientation.

And last but not least we use ACE and TAO since these are multi
platform, high performance libraries which proved to be very
sophisticated in terms of usability, portability and scalability. ---
Additionally, they are open source libraries. They haven't
disappointed us yet, to the contrary.

\section{The \miro Group}

The \miro core developers are (in alphabetical order):
\begin{itemize}
  \item Stefan Enderle
  \item Gerhard Kraetzschmar
  \item Gerd Mayer
  \item Guillem Pages
  \item Stefan Sablatn\"og
  \item Steffen Simon
  \item Hans Utz
\end{itemize}

\section{Installation and Setup}

Ask your system administrator if there exists a central installation,
or whether it would be more convenient to prepare a central
installation. Prerequisites for installing \miro is that you have a
running QT (Version $>$ 2.2.x) as well as an ACE/TAO installation.

\subsection{ACE and TAO Installation}
\label{sec::aceInstall}

ACE/TAO is a large software package with many configuration
options affecting the build process as well as its runtime
features. Therefore we summarize here our experience with the
ACE/TAO installation.

At the time of this writing we suggest always to use the
latest beta of ACE/TAO instead of the latest stable version ---
currently 5.2.2/1.2.2. TAO is under rapid development and \miro
utilizes some of its latest features (like the CORBA Notification
Service).

The options that should be set additionally in the {\tt
  platform\_macros.GNU} file (to be found under
$<$ACE-directory$>$/include/makeinclude/) are the following:

\begin{lstlisting}[frame=tb]{}
debug=0
qt_reactor=1
\end{lstlisting}

Additionally, for memory footprint reduction you might also add the
following:

\begin{lstlisting}[frame=tb]{}
DEFFLAGS=-DACE_USE_RCSID=0
ACE_COMPONENTS=FOR_TAO
\end{lstlisting}

We also recommend to read the installation instructions provided
by ACE carefully. We admit, the installation isn't entirely trivial.

To compile and run programs that use the ACE/TAO toolkit, also the
environment variables {\tt ACE\_ROOT} and {\tt TAO\_ROOT} need
to be set to the appropriate root directories of your ACE/TAO
installation. For the bash shell the following lines in your local
{\tt .bashrc} file should do the job:

\begin{lstlisting}[frame=tb]{}
export ACE_ROOT=<path to the ACE directory>
export TAO_ROOT=$ACE_ROOT/TAO
export LD_LIBRARY_PATH=$ACE_ROOT/ace/:$LD_LIBRARY_PATH
\end{lstlisting}
%%$

Note that if you use a non-standard Qt installation, also the
environment variable \texttt{QTDIR} has to point to the correct
installation directory of the Qt library.

\subsection{\miro installation}

This section is an extended version of the INSTALL file contained
within \miro. If you have problems installing \miro, read on.

\subsubsection{Requirements}

As already described above, you need the following packages in order
to compile \miro:
\begin{itemize}
\item ACE ($>$= 5.2.2): Please refer to the installation instruction
  in \ref{sec::aceInstall} for building and installing ACE. The source
  tarball and CVS access is available at:\\
  \texttt{http://www.cs.wustl.edu/$\sim$schmidt/ACE.html}.
\item TAO ($>$= 1.2.2): Building and installation is also described in
  section \ref{sec::aceInstall}. TAO downloads are available on:\\
  \texttt{http://www.theaceorb.com/}.
\item Qt ($>$= 2.2.4): Installation should be quite easy, as Qt uses a
  configure script for automatic system checks. Qt is installed on
  most modern Unix/Linux systems anyway, since KDE relies on
  it. Recent versions of Qt can be found at:\\
  \texttt{http://www.trolltech.com/}.
\end{itemize}

If you want to compile the wrapper classes for the speech detection
system, you also need:
\begin{itemize}
\item Speech tools: The Edinburgh Speech Tools Library is a collection
  of C++ class, functions and related programs for manipulating the
  sorts of objects used in speech processing. It is to provide the
  underlying classes in the Festival Speech System. We use version
  1.2.3 that can be retrieved from:\\
  \texttt{http://www.cstr.ed.ac.uk/projects/speech\_tools}
\item Festival: The Festival Speech Synthesis System is a general
  multi-lingual speech synthesis system developed at CSTR. We
  successfully used version 1.4.2. See:\\
  \texttt{http://www.cstr.ed.ac.uk/projects/festival/}
\item Sphinx: The CMU Sphinx Group Open Source Speech Recognition, a
  real-time, large vocabulary, speaker independent speech recognition
  system. We use version Sphinx2-0.4, downloaded from:\\
 \texttt{http://www.speech.cs.cmu.edu/sphinx}
\end{itemize}

Depending on your camera system, you may need libraries for the
IEEE 1394 (aka Firewire) support:
\begin{itemize} 
\item libraw1394: This library provides direct access to the IEEE 1394
  bus through the Linux 1394 subsystem's raw1394 user space
  interface. It can be found one it's homepage:\\
  \texttt{http://www.linux1394.org/}\\
  or on SourceForge:\\
  \texttt{http://sourceforge.net/projects/libraw1394/}
\item libdc1394: It is a library that is intended to provide a high
  level programming interface for application developers who wish to
  control IEEE 1394 based cameras that conform to the 1394-based
  Digital Camera Specification. Available via:\\
  \texttt{http://sourceforge.net/projects/libdc1394/}
\end{itemize}
Both libraries are available as compiled packages for the most modern
Linux systems too.

To build the documentation, your system should provide the following
tools:
\begin{itemize}
\item Doxygen: It is a JavaDoc like documentation system for C++, C,
  Java and IDL. It can generate an on-line HTML documentation
  extracted directly from the sources, which makes it much easier to
  keep the documentation consistent with the source code. You can
  download from the homepage:
  \texttt{http://www.stack.nl/\~dimitri/doxygen/}\\
  or again from SourceForge:\\
  \texttt{http://sourceforge.net/projects/doxygen/}
\item The LaTeX package (either pdflatex or latex and dvips, bibtex
  and makeindex -- available via one of the CTAN server, e.g.\ 
  \texttt{http://dante.ctan.org}) and the convert image convertsion
  utility from the ImageMagick package \\
  (\texttt{http://imagemagick.sourceforge.net/}).
\end{itemize}

\subsubsection{Download}

As if you read this text, you already have a version of \miro, but if
you want to update this version, or need some additional information,
please have a look at
\texttt{http://smart.informatik.uni-ulm.de/MIRO/index.html}

\subsubsection{Compilation}

\miro is now shipped with a configure script, so the package can be
installed with the usual steps:

\begin{lstlisting}[frame=tb]{}
./configure
make 
(make install)
\end{lstlisting}

The configure script looks automatically for the necessary and
optional software packages, and will stop or print a warning if some
of them are not found. If you install e.g.\ ACE in an unusual place,
you can pass the configure script some additional options like for
example \texttt{--with-ACE=<ACE-root-dir>}. For Qt, their are more
flexible command-line parameters, because Qt is often distributed over
several directories (like e.g.\ in a standard Debian installation). A
complete list of command-line options are accessible via
\texttt{./configure --help}.

Another way of telling the configure script where the packages are
installed is the use of environment variables. For the required
packages, these are \texttt{ACE\_ROOT}, \texttt{TAO\_ROOT} and
\texttt{QTDIR} point to the base directory of each package. Because
ACE and TAO need this environment variables anyway, this is the most
comfortable way to point to the directories. For the speech detection
wrappers classes, \texttt{SPEECH\_TOOLS\_ROOT}, \texttt{SPHINX\_ROOT}
and \texttt{FESTIVAL\_ROOT} are possible. It can be necessary, to let
the \texttt{LD\_LIBRARY\_PATH} point to all the libraries anyway.

If the configure script did not find the different packages anyhow, or
if your system require unusual options for compilation or linking that
the configure script does not know about, you can give configure
initial values for variables by setting them in the environment. Using
a Bourne-compatible shell, you can do that on the command line like
this:

\begin{lstlisting}[frame=tb]{}
CFLAGS=-O2 LIBS=-lposix ./configure
\end{lstlisting}

Or on systems that have the `env' program, you can do it like this:

\begin{lstlisting}[frame=tb]{}
env CPPFLAGS=-I/usr/local/include LDFLAGS=-s ./configure
\end{lstlisting}

On runtime, some applications rely on two environment variables,
namely \texttt{MIRO\_ROOT} (the base directory) and \texttt{MIRO\_LOG}
(directory, where logging data are stored), so please set these
variables to the appropriate values too.

Additionally, you can enable or disable several features, like the
support for different robot platforms and for different video devices.

Supported robot platforms:
\begin{itemize}
\item The B21 port is the oldest one and should therefore be the most
  stable, regarding the interfaces. All of the robots hardware
  components are supported by \miro.
\item The Sparrow99 robots are our soccer robots. They are our testbed
  for multi-robot programming. Therefore most new technology is first
  tested on this platform. So even as you do not have a Sparrow99
  robot (we built them ourselfs), it might be interesting to look at
  the sources for this robot if you are looking for group
  communication technology etc.
\item The Pioneer1 platform is a complete port for the ActiveMedia
  robot series, based on the PSOS, P2OS and AROS protocols. But we
  could only test and extend it to the platforms available to us.
  \begin{itemize}
  \item The old Pioneer1 is supported, but lacks maintenance, as our
    model is not used at the lab anymore.
  \item The Performance PeopleBot is in active development and already
    mostly complete. Motors, Sonars, Bumpers and Video are working.
    PanTilt and Gripper are experimental, Zoom is to come next.
  \end{itemize}
\end{itemize}

Supported video devices:
\begin{itemize}
\item Bttv Frame Grabbers: These frame grabber cards are supported via
  video for Linux.  This is the standard way of connecting standard
  analog video cameras to the computer.
\item Firewire Digital Cameras: \miro supports the fire wire digital
  camera protocol using libraw1394 and libdc1394.
\item Matrox Meteor Frame Grabbers: These rather old frame grabber
  cards are also supported by \miro. This device however, is mostly
  unmaintained.
\end{itemize}

Finally, you can choose if the documentation should be build or not.
Even if you decide not to build it, but the configure script found all
the necessary tools, the Makefiles are prepared. So you can go to the
\texttt{doc/tex} and \texttt{doc/html} directory later on and build
the documentation there with a simple make.

After running the configure script, \miro show up a summary of what
will be compiled and which features will not. If this is not what you
desired, please check the messages coming up during the configure run
for packages, \miro did not find.

\subsubsection{Installation}

\miro can be installed with the a simple:

\begin{lstlisting}[frame=tb]{}
make install
\end{lstlisting}

By default, the package's files will be installed in \texttt{/usr/local/bin},
\texttt{/usr/local/lib} etc.  You can specify an installation prefix other
than \texttt{/usr/local} by giving configure the option
\texttt{--prefix=PATH} (note: if you use the prefix-option, you should
install \miro actually, otherwise libtool may look for the library in
the installation directory and therefore do not find it).

Beside that, \miro can be used already without installation. Therefore,
a make run install the libraries and the binaries during the
compilation into the \texttt{lib/} and \texttt{bin/} directory. This
enlarges the \miro directory, but can be quite useful, if you work on
\miro itself and a derived application at the same time.

Moreover with this trick it is possible, to use different \miro
versions in parallel at the same time. Therefore the build process
allow to compile the package in a different directory. Assuming you
have extracted \miro in a directory called \texttt{Miro}, you can
build it in two different directories with miscellaneous options:

\begin{lstlisting}[frame=tb]{}
mkdir Miro.B21
cd Miro.B21
../Miro/configure --enable-B21
make
cd ..
mkdir Miro.Sparrow99
cd Miro.Sparrow99
../Miro/configure --enable-Sparrow99
make
\end{lstlisting}

\subsubsection{Developer information}

The build-process within \miro is done using automake, autoconf and
libtool. Normally, there is no need to use these programs, as of
running the configure script is the only thing a user have to do.

The configure shell script attempts to guess correct values for
various system-dependent variables used during compilation. It uses
those values to create a Makefile in each directory of the package.
It also create one config.h file containing system-dependent
definitions that can be used in the source file to allow appropriate
compilation:

\begin{lstlisting}[frame=tb]{}
#ifdef HAVE_CONFIG_H
#include <config.h>
#endif
...
#ifdef MIRO_HAS_1394
...
#endif
\end{lstlisting}

If you want to change something within \miro like for example adding
additional files or directories, you have to change the proper
\texttt{Makefile.am}. Do \textbf{not} change the \texttt{Makefile.in}
or even the final \texttt{Makefile}, because changes there are getting
lost are the next automake run. The same applies, if you want to
change the behaviour of the configure script itself, for example to
detect new software packages or special version information, you have
to change the \texttt{configure.ac} and \textbf{not} the
\texttt{configure} file itself.

Assuming that you have written appropriate \texttt{Makefile.am} and
\texttt{configure.ac} files, you should be able to build your project
by entering the following commands:

\begin{lstlisting}[frame=tb]{}
aclocal -I config
autoheader
automake
autoconf
./configure
\end{lstlisting}

For every consecutive step, you can simply call the resulting
\texttt{config.status} script, that rerun the configure procedure with
the used configuration values.

Normally, it should be enough, to use the available examples within
\miro to add some small additions to the build process, but sometimes
a deeper understanding of the used tools might be necessary. Please
have a look at the manuals, available e.g.\ directly from the GNU
homepage: \texttt{http://www.gnu.org/manual/} or read the freely
available book on \texttt{http://sources.redhat.com/autobook/}.

\section{\miro Directory Structure}

\miro comes with the full source code and documentation as well as a
set of test programs and examples that should facilitate your first
steps when writing your own programs. To help you navigate through the
directory tree of a \miro distribution, we give a brief overview of
the directories present in the \miro root directory:

\begin{description}
\item[bin:]
  This directory contains links to the binaries of \miro.
  For an explanation of the individual binaries, see Chapter
  \ref{sec:services} about the robot services.
\item[doc:]
  Here, the available documentation is gathered. In the
  subdirectory {\tt tex} you find the postscript version of this
  manual and in the directory {\tt html} resides the auto generated
  online documentation of all \miro classes and their methods
  (the starting page of this documentation resides at: \\
  {\tt \$MIRO\_ROOT/doc/html/idl/index.html} respectively \\
  {\tt \$MIRO\_ROOT/doc/html/cpp/index.html}).
\item[etc:]
  Config files for the individual robots.
\item[examples:]
  Examples on how to use individual interfaces of \miro. If you want
  to write your own programs utilizing the \miro framework, this is a
  good place to look for inspirations.
\item[idl:]
  The IDL sources of the CORBA interfaces and data types.
\item[performance-tests:]
  Some tests that measure the performance of \miro's
  services. --- Not too much there at the moment.
\item[scripts:]
  Utilities for source code formatting and handling sequences of
  datafiles.
\item[src:]
  Here, all sources of the \miro services reside. New services for
  further robot platforms should go in here.
\item[templates:]
  Templates for Makefiles and the headers for source files. Copy the
  corresponding template, if you want to start a new subproject within
  the \miro source hierarchy. If you start your own new project on top
  of \miro, the Makefile templates might still be useful for you. They
  handle all the stuff concerning ACE/TAO and the multi-platform build
  process. See Chapter \ref{sec:makefile} for details.
\item[tests:]
  Small test programs to monitor or test isolated interfaces of the
  \miro robot control architecture.
\item[utils:]
  Utilities made for convenience. No magic to be expected here.
\end{description}

%=============================== Tutorial ====================================

\chapter{Definitions}


\section{Units}

When it comes to parameter passing, distances are specified in mm and
angles in radiant. Velocities are specified in mm/s and rad/s
respectively. 

\section{Coordinates}

\miro uses the cartesian coordinate space. Angles are specified
mathematically, that is, counter-clock wise.

%mm, rad


\begin{figure}[!ht]
  \begin{center}
    \ifpdf
    \includegraphics[width=8cm]{../fig/grid_coords.png}
    \else
    \includegraphics[width=8cm]{../fig/grid_coords.eps}
    \fi
  \end{center}
\end{figure}


%%% Local Variables: 
%%% mode: latex
%%% TeX-master: "miro_manual"
%%% End: 

\input{using}
\chapter{Available Services}
\label{sec:services}

The \miro framework abstracts the robots hardware devices as active
services, that export the sensors and actuators functionality via CORBA
interfaces that can be accessed transparently from other programs,
probably running on totally different machines. Which services there
are available depends on the individual robot type. To provide easy
access to the services interface, each service registers itself under
a standard name at the CORBA Naming Service. To allow multiple
robots to be accessed at a time, each individual robot creates its
one naming context within the Naming Service under which its services
register. The default naming context is {\tt Miro}, but a given robot
should register itself under its own name (for example {\tt stanislav}).
The naming context to use can be specified at the
command line of every service and example program via the
{\tt -MiroNamingContext} option.


\section{Command Line Options}
\label{sec:command-line}

For each service the following command line options can be specified:

\begin{description}
\item[-MiroConfigFile {\em fileName}:]
  The services are widely configurable. These parameters are specified
  in config files within the directory {\tt \$MIRO\_ROOT/etc}. The
  file name is derived from the computers {\tt \$HOST} environment
  variable: {\tt \$HOST}.xml. If you want to specify another
  configuration file, you can do with this command line option.
\item[-MiroNamingContext {\em contextName}:]
  The short form of this option is \textbf{-MNC}.
  As mentioned before all services register their interfaces at the
  naming service within a naming context. The default naming context
  is {\tt Miro}. In multi robot scenarios for instance it would be
  necessary to assign a different naming context to each robot. This
  can be done by this command line option. Note that since it is
  necessary to specify none default naming context names also at the
  client side, this option is also recognized by every client program.
\item[-MiroRebindIOR:]
  To prevent services from different robots to accidentally
  deregister each other from the naming service, an already existing
  entry in the naming service isn't unbound by a service, if it tries
  to register under the same name. Anyhow, if you need to overwrite an
  existing entry, specify this option at the command line.
\end{description}


\section{Individual Interfaces}

In the following, we list the different services and their
interface names in the naming service, all interfaces are
registered within the specified naming context in the naming
service. For multi robot scenarios the canonical naming convention
for the individual robots naming contexts is the robots name.
Don't get confused by the fact that many services are collocated
within a single binary, while for others there exists a dedicated
binary. Note that in general the specialized interfaces of
individual robots register at the Naming Service under their
ancestors name.  The interface documentation can be looked up
within the \miro reference and in the \miro online documentation
as well.  For some services specializations for individual robot
types do exist. Those are described in the subsequent sections.

\begin{description}
\item[Odometry:]
  The motion service registers the {\tt Miro::Odometry} interface as
  {\tt Odometry}. It encapsulates
  the odometry (dead reckoning) sensor.
\item[Motion:]
  The motion service registers the {\tt Miro::Motion} interface as
  {\tt Motion}.
\item[RangeSensor:] The range sensor interface is the general
  abstraction of all range sensor devices, such as sonar, laser range
  finders etc. It offers a method for querying the sensors physical
  configuration as well as the latest sensor reading.  See the {\tt
    QtRangeSensor} program in the utils directory for an example on
  how to use this information.  The range sensor interface becomes
  registered at the naming service under the name of the actual
  sensory device.
  \begin{description}
  \item[Sonar:] The sonar service registers as {\tt Sonar}.  It
    provides an interface to the very common ultrasonic sensors for
    robots.
  \item[Infrared:] It registers as
    {\tt Infrared}.
  \item[Tactile:] Some robots have bumpers as some sort of ''it's too
    late'' sensors.  Still it is better to stop when you crashed than
    just moving on. It is registered as {\tt Tactile}.  Since it can
    be interpreted as a (very limited) range sensor, it is also a {\tt
      Miro::RangeSensor}.
  \item[Laser:] The laser scanner service registers as {\tt Laser}. It
    provides very accurate 180� distance measurements, with up to
    48Hz.
  \end{description}
\item[Stall:]
  Similar to the bumpers, the stall detection monitors the robots
  motion and detects, when the robot is stuck in its movement. The
  {\tt Miro::Stall} interface registers at the naming service as
  {\tt Stall}.
\item[Video:] The video service registers the {\tt Miro::Video}
  interface as {\tt Video}. It provides camera images at a rate of up
  to 25 images per second captured by frame grabber cards. Currently
  supported are {\tt matrox meteor} and cards that are supported
  by {\em video for linux} (bttv8).  Since the bandwidth needed for
  uncompressed image data transportation exceeds the capabilities of
  most todays network devices and even memory copying can introduce
  excessive overhead, the video service enables access
  to the grabbed images via shared memory keys.
\item[PanTilt:] Cameras often are mounted on top of a pan-tilt unit,
  allowing the robot to look sideways while moving in another
  direction. Lightweight versions of such a device are panning or
  tilting units.  The associated services register the
  {\tt Miro::Pan}, {\tt Miro::Tilt} and {\tt Miro::PanTilt}
  interfaces as {\tt Pan, Tilt} and {\tt PanTilt} at the naming
  service.
\item[Buttons:]
  For simple user interaction, some robot types provide push
  buttons. The buttons service registers the {\tt Miro::Buttons}
  interface as {\tt Buttons}.
\item[Speech:]
  To provide a more natural way of communication, some robots are
  equipped with speech synthesizer cards. The speech service registers
  the {\tt Miro::Speech} interface as {\tt Speech}.
\end{description}


\section{Asynchronous Sensory Information}
\label{SEC:ASYNCHRONOUS_SENSORY_INFORMATION}

Another feature of the \miro framework is the asynchronous
distribution of sensory data via the CORBA Notification Service
\cite{OMG:Notify-Spec}. These event channels allow filtered and
priority based event processing for time critical sensory information
distribution under high load.  Therefore in every robots naming
context exists a reference {\tt EventChannel} under which the
robots notification service can be accessed.

The usage of the notification service within the \miro framework is
explained in more detail in \ref{sec:notify}. In the following we give
only a brief description. The notification service allows simple event
filtering on the bases of the domain name and the type name of the
event. This way the individual consumer can easily subscribe for the
messages it is interested in. As the domain name, the naming context
is used within the \miro framework (note that in a multi robot
scenario, an event channel can transport data from multiple robots).
In the following the type names of the events and the data contained
in the event are listed for the individual services:

\begin{description}
\item[Odometry:] The odometry servant and all derived services
  generates periodical {\tt Miro::MotionStatusIDL} events), that
  propagate the robots current position and velocities. The event type
  name is {\tt Odometry}.
\item[RangeSensor:] The range sensor servant and all derived services
  can generate periodical range sensor events. The event type name is
  that of the derived sensor interface (Sonar, Laser etc.). It is also
  returned by the {\tt getScanDescription} method within the {\tt
    ScanDescriptionIDL} struct. The event data is one of three types,
  as also specified within {\tt Miro::SensorDescriptionIDL}.
  \begin{description}
  \item[Miro::RangeScanEventIDL]
    The data of a range sensor that acquires data in a continuous scan
    pushes its data within this struct. (Currently, there does exist
    no actual sensor that behaves in this way, but anyways...)
  \item[Miro::RangeGroupEventIDL]
    The data of a range sensor that acquires data groupwise
    pushes its data within this struct. A range sensor group
    normally is formed by a set of sensors that are mounted on the
    same hight, pointing in different directions. The laser scanners
    are organized that way.
  \item[Miro::RangeBunchEventIDL] The data of a range sensor that
    acquires data in a discontinuous fashion pushes its data within
    this struct. Each sensor reading contains its own group id as
    well as its index within that group. Especially sonar sensor
    are organized that way, since neighbouring sensor mustn't be
    fired simultaneously to minimize crosstalk.
  \end{description}
  The event type name is that of the actual sensor: {\tt Sonar}, {\tt
    Laser}, {\tt Tactile}, etc. as also specified within {\tt
    Miro::SensorDescriptionIDL}.

\item[Sonar:] The sonar service generates range sensor events with the
  type name {\tt Sonar}. Note that the sonar sensors are fired
  interleavingly to avoid crosstalk between them. Therefore the
  payload of an event emitted by a sonar device is normally a
  \texttt{RangeBunchEventIDL}. That way you have to analyze multiple
  {\tt Sonar} messages to get a full sonar scan.
\item[Infrared:] The infrared service generates range sensor events
  with the type name {\tt Infrared}.
\item[Tactile:] The tactile service generates range sensor events with
  the type name {\tt Tactile}. Due to the hopefully low frequency
  at which those events occur it might be better to listen to the
  {\tt Tactile} events, instead of polling the tactile status. Tactile
  devices usually contain a \texttt{RangeBunchEventIDL} struct.
\item[Stall:] The stall service also emits events. These contain the
  type name {\tt Stall}. For the event data see the individual
  robots descriptions.  Due to the hopefully rare occasions at which
  those events occur it might be better to listen to these events,
  instead of polling the stall interface.
\item[Buttons:] They also provide the event triggered communications
  model that uses the event type {\tt Button}. The events payload is
  of type {\tt ButtonStatusIDL}.
\end{description}


\section{\bXXI}

The \bXXI robot has a large number of available sensors and actuators. The
relevant binaries for its specialized services are:

\begin{description}
\item[B21Base:] This is a collection of the robots main services. This
  binary incorporates the motion service as well as the sonar, tactile
  and infrared services. Furthermore it gives access to the four
  colored buttons, that are mounted on top of the robot. The
  corresponding interfaces are:
  \begin{itemize}
  \item {\tt Miro::B21Motion} \\
    The {\tt Miro::B21Motion} interface is derived from the
    {\tt Miro::Motion} interface and gives access to the
    specialized movement commands of the B21 robot. It registers under it
    ancestors name {\tt Motion}.
  \item {\tt Miro::Sonar} \\
    The derived {\tt Miro::RangeSensor} interface registers itself
    as {\tt Sonar}.
  \item {\tt Miro::Infrared} \\
    The derived {\tt Miro::RangeSensor} interface registers itself
    as {\tt Infrared}.
  \item {\tt Miro::Tactile} \\
    The derived {\tt Miro::RangeSensor} interface registers itself
    as {\tt Tactile}.
  \item {\tt Miro::B21Buttons} \\
    The B21 has four buttons, that can be used as a simple user
    interface. The {\tt Miro::B21Buttons} interface is supported by
    this service. It sends {\tt Miro::ButtonStatusIDL} events, if a
    button is pressed or released.
  \end{itemize}

\item[SickLaserService:]
  This is the service to access the SICK laser scanner. It supports
  the {\tt Miro::RangeSensor} interface.
\item[DtlkSpeech:]
  This service lets you control the DoubleTalk Speech cards. The
  interface is named {\tt Miro::Speech}
\item[VideoService:]
  It gives you access to the two frame grabbers of the robot. It
  supports the {\tt Miro::Video} interface.
\item[DpPantilt:]
  It supports the {\tt Miro::DirectedPerceptionPanTilt} interface,
  which is derived from {\tt Miro::PanTilt}. It therefore registers
  itself as {\tt PanTilt}
\end{description}

Note that the B21 has two internal computers, to which the
different sensors and actuators are attached. You have to start
the relevant binaries at the correct computer. The
sensory/actuatory distribution in our laboratory is as follows:
Start B21Base and LaserService on the left computer, VideoService
and DpPantilt and DtlkSpeech on the right.


\section{\sparrow}

The \sparrow soccer robots have a similar diversity of sensors and
actuators, but for efficiency and internal design reasons, they are
grouped together in less binaries:

\begin{description}
\item[SparrowBase:]
  Within this file the following services are grouped together:
  Motion, Stall, Kicker, Buttons, Sonar, Infrared and Pantilt. The
  corresponding interfaces are:
  \begin{itemize}
  \item {\tt Miro::SparrowMotion} \\
    The {\tt Miro::SparrowMotion} interface is derived from the
    {\tt Miro::Motion} interface and gives access to the
    specialized movement commands of the \sparrow robot. It registers
    under it ancestors name {\tt Motion}.
  \item {\tt Miro::Stall} \\
    The {\tt Miro::Stall} interface is registered as {\tt Stall}.
  \item {\tt Miro::Kicker} \\
    Due to its purpose as football robot, the \sparrow robot has a
    kicker, that can be accessed via this interface. It registers at
    the naming service as {\tt Kicker}.
  \item {\tt Miro::Buttons} \\
    The {\tt Miro::Buttons} interface registers as
    {\tt Buttons}.
  \item {\tt Miro::Sonar} \\
    The {\tt Miro::RangeSensor} interface registers itself
    as {\tt Sonar}.
  \item {\tt Miro::Infrared} \\
    The {\tt Miro::RangeSensor} interface registers itself
    as {\tt Infrared}.
  \item {\tt Miro::PanTilt} \\
    The {\tt Miro::SparrowPanTilt} interface is registered as
    {\tt PanTilt}.
  \end{itemize}

  Note that the
  stall service sends {\tt Miro::SparrowStallIDL} events.
\item[VideoService:]
  It gives you access to the frame grabber of the robot. It supports
  the {\tt Miro::Video} interface.
\end{description}


\section{Pioneer}

Due to the limited variety of sensors and actuators on the Pioneer robots, there exists only
the most fundamental services for this robot.

\begin{description}
\item[PioneerBase:]
  The following services are grouped together: Motion, Stall and Sonar.
  The corresponding interfaces are:
  \begin{itemize}
  \item {\tt Miro::PioneerMotion} \\
    The {\tt Miro::PioneerMotion} interface is derived from the
    {\tt Miro::Motion} interface and gives access to the
    specialized movement commands of the Pioneer robot. It registers
    under it ancestors name {\tt Motion}.
  \item {\tt Miro::PioneerStall} \\
    The {\tt Miro::PioneerStall} interface (derived from {\tt
      Miro::Stall}) is registered as {\tt Stall}.
  \item {\tt Miro::Sonar} \\
    The {\tt Miro::RangeSensor} interface registers itself
    as {\tt Sonar}.
  \end{itemize}

  Note that the stall service sends {\tt Miro::PioneerStallIDL} events.
\end{description}


\section{Frame Grabbers}

The {\tt videoService} provides an interface to several frame grabber
cards. All BTTV cards that are supported by the video for linux
project are also supported here (namely the Bt848/849/878/879 based
frame grabbers). The Matrox Meteor frame grabbers cards are also
supported, but note that kernel drivers are only available for the
older 2.2.x kernel series.

The idl-interface that is used by the {\tt videoService} is {\tt
  Miro::Video}. The Server registers itself as {\tt Video} in the
Naming Service. You can get the last grabbed image immediately by use
of the {\tt getImage()} method or wait for the next available image
with the {\tt getWaitImage()}. What you get is a handle to the image
that simply points to a memory range, where the color values for the
image (or the intensity value for each pixel if it is a grayscale
image) are listed for each single pixel. Additionally you get the
image dimensions, an associated time stamp and the robots position at
the moment of recording the picture.

As always there is a configuration section in the xml-file called {\tt
  video}, where the following parameters can be used:

\begin{description}
\item[device:] This section describes, which device-name the frame
  grabber card uses.
\item[grabber:] Determine which card your robot uses, either a {\tt
    bttv} based card or a matrox {\tt meteor} card.
\item[format:] With the format parameter you can choose, which
  standard you camera uses. Either the (european) {\tt pal} format,
  the (american) {\tt ntsc} format or the (asiatic) {\tt secam}
  format. With the {\tt auto} option, the format is choosen
  automatically.
\item[source:] This parameter defines, from which input source the
  server should read. Possible choises are: {\tt composite1}, {\tt
    composite2}, {\tt composite3}, {\tt svideo} and {\tt tuner}.
  %%{\tt usb} or {\tt 1394}.
\item[palette:] Palette can be one of the following: {\tt gray}, {\tt
    rgb}, {\tt rgba}, {\tt bgr} or {\tt bgra}. \\
  {\tt rgb} or {\tt bgr} indicate the 24-Bit mode, whereas {\tt rgba} and
  {\tt bgra} are 32-Bit modes.
\item[subfield:] With this parameter you can choose which subfields to
  use: {\tt odd}, {\tt even} or {\tt all}.
\item[width:] The {\tt width} that the image should have.
\item[height:] The {\tt height} that the image should have.
\item[connections:]
\item[buffers:]
\item[byteorder:] {\tt 0}
means native mode, {\tt 1} means rgb.
\item[pixelsize:] This
parameter specifies the size in bytes of each pixel (1, 3 or 4).
\item[upsidedown:] On some systems the camera is mounted upside
down. Set this parameter to true if this is the case.
\end{description}

Note, that not each card supports each option and that not every
combination of options can be used for each card. If in doubt,
refer to the technical manual of your frame grabber card.

The following example shows the configuration used for our hardware:

\begin{lstlisting}[frame=tb]{}
<video>
  <device>/dev/v4l/video0</device>
  <grabber>bttv</grabber>
  <format>pal</format>
  <source>composite1</source>
  <palette>bgr</palette>
  <subfield>odd</subfield>
  <width>384</width>
  <height>288</height>
  <connections>16</connections>
  <buffers>16</buffers>
  <byteorder>1</byteorder>
  <pixelsize>3</pixelsize>
</video>
\end{lstlisting}


\section{SICK Laser Scanner}

The {\tt sickLaserService} provides an interface to a laser range
finder of type Sick PLS.
%%pls200 (TODO: check this?).
This sensor is connected to the controlling PC via a serial line. It
is delivered either with a RS-232 or a RS-422 compliant interface.

The idl-interface implemented by the {\tt sickLaserService} is
{\tt Miro::Laser}, which is derived from
{\tt Miro::RangeSensor}. The server can be driven in two different
modes, either the sensor automatically provides a scan about 40 times
a second, or every scan is polled, which results in a maximum scan
frequency of 10 times a second. The second mode also allows to choose
a different (i.e. lower) scan frequency. At the moment no other
special features of the Sick PLS200 are supported.

The configuration of this service is provided in xml, within the
section {\tt sick}. The following parameters are understood:

\begin{description}
\item[device:]
  The filename of the device which should be used.
\item[baudrate:]
  The baudrate for communication with the sensor,
  possible values are: 9600, 19200, 38400 and 500000.
\item[stdcrystal:]
  For achieving the unusual baud rate of 500000
  baud, we use a serial adapter, that is equipped with a 16MHz
  crystal, instead of the standard 14.??MHz. Due to this reason, the
  standard baud rates must be generated by modifying the divisor of
  the UART. If you have an off-the-shelf serial adapter, use
  {\tt true} here, you will probably not be able to use 500000 baud
  then.
\item[continousmode:]
  If {\tt true} is provided for this
  parameter the sensor will provide data automatically, about 40 times
  a second. If {\tt false} is used, the service will poll a
  measurement after an interval that can be specified with the
  following parameter.
\item[pollintervall:]
  The interval between two consecutive
  measurements, measured in microseconds. This parameter has no effect
  if continousmode is set to {\tt true}.
\end{description}

The following example shows the configuration used for our hardware:

\begin{lstlisting}[frame=tb]{}
<sick>
  <device>/dev/laser</device>
  <baudrate>500000</baudrate>
  <continousmode>true</continousmode>
  <stdcrystal>false</stdcrystal>
  <pollintervall>100000</pollintervall><!-- pollinterval in usec -->
</sick>
\end{lstlisting}

As the laser provides a specialization of a range sensor, we also
provide a scandescription for it. See section \ref{scandescrption} for
details.


\subsection{Possible Problems}

Due to the unusual and high data rate of 500000~baud a special serial
interface card is required. The card recommended by Sick provides a
16550A compatible interface, for that reason the current
implementation is able to use the card through the standard linux
serial device drivers, which keeps the implementation of the service
more portable. Unfortunately this introduces a big drawback: receiving
up to 50000 characters per second, results in at least $50000/16=3125$
interrupts per second, this is an enormous load. It is essential, that
every interrupt is handled on time, because the sensor does neither
support hardware, nor software flow control, and provides a packet
with data every 25~ms. If you have the chance to use an interface card
with a larger FIFO buffer than the standard 16 bytes, this will
provide a large improvement in performance, stability and system
reactivity.  We solved most of our stability problems, that were
caused by lost packages without a larger FIFO: We enabled irqs
throughout IDE interrupt processing (this may be extremely dangerous,
depending on your configuration, see man page of {\em hdparm}). Another
solution could be using {\em irqtune} to give the serial irq a higher
priority, or using {\em RT-Linux}.

\textbf{Due to the adaptations of the divisor for the serial line,
the {\tt SickLaserService} has to be suid root. If you use a
standard crystal, and you do not use 500000~baud, this is not
necessary.}


\section{DoubleTalk Speech Card}


\section{Directed Perception Pantilt}

%\section{Higher-level Servers}

%In the current version of \miro, no high-level services are included.

%\subsection{MapServer}

%\subsection{RegionServer}

%\subsection{SaliencyServer}

%\subsection{ClassificationServer}


%%% Local Variables:
%%% mode: latex
%%% TeX-master: "miro_manual"
%%% End:

\input{clients}
\chapter{Writing Your First Programs}

This chapter tries to help you with the first steps of writing
programs that use the \miro framework. The CORBA environment and the
\miro framework seem to raise the bar for an easy entry into robot
programming. While this can hardly be denied they facilitate
tremendously the task of writing distributed programs. And since robot
control software is inherently distributed (ever thought of multiple
robots?) it seems the only way to go.

As you will see, most of the distributed programming complexity is
initially hidden from the user:

\begin{itemize}
\item The programmer simply calls methods of the devices interface.
\item The programmer communicates via proxy object to the remote
  service. The most tricky part is getting the object reference.
\item The distributed environment is transparent. The
  remote method invocation (RMI) is hidden entirely.
\end{itemize}


\section{Makefiles}
\label{sec:makefile}

\miro comes with it own Makefile framework that is derived from the
ACE/TAO Makefile hierarchy. Makefile templates reside in the directory
{\tt \$MIRO\_ROOT/templates}, that should facilitate the creation
of further subprojects within the \miro directory structure but can
just as well be used to start new projects outside the
{\tt \$MIRO\_ROOT} directory. The makefiles are designed to either
build subdirectories, libraries or binaries. To use a template, simply
copy it into the corresponding directory under the name
{\tt Makefile} and adapt it as explained in the following sections.


\subsection{Makefile.dir}

This Makefile starts subbuilds in the specified directories. List all
the subdirectories for your make process in the variable {\tt DIRS}.


\subsection{Makefile.bin}

This Makefile builds a single binary. The name of the binary to be
built is specified in the variable {\tt BIN}. A corresponding
source has to exist. Addition source files are specified in the
variable {\tt FILES}, without their suffix (.cpp). If the binary has
to be linked against additional libraries, those have to be added to the
variable {\tt LDLIBS}.


\subsection{Makefile.lib}

This Makefile builds a library. The name of the library can be
specified in the variable {\tt NAME}, the files building the
library in the variable {\tt FILES} (without the .cpp suffix).

If you want to build a static library, set the switch {\tt
  static\_libs\_only} to 1 (which can also be enabled globally within
your \texttt{platform\_macros.GNU} file). Note, that you neet
additional dependecy tracking to catch modified static libraries to
link your binaries against. See \texttt{/src/sparrowBase/Makefile} for
a possible solution of this problem.

\section{A Simple Sonar Client}

To discuss things using actual code, let's look at the simple task of
obtaining data from a sonar sensor device. Since this is a range
sensor, the sonar is queried via the generalized range sensor
interface. The only difference in querying an infrared or a sonar
device is the name under which these sensors are registered within the
naming service.

\lstinputlisting[frame=tb, caption={examples/sonar/SonarPoll1.cpp}]{SonarPoll1.cpp}
\label{lst:SonarPoll1}

A step by step walk trough the code:

\lstinputlisting[frame=tb, first=1, last=1]{SonarPoll1.cpp}

The file \lstinline!miro/Client.h! contains the definition of the class
\lstinline!Miro::Client!. See below.

\lstinputlisting[frame=tb, first=2, last=2]{SonarPoll1.cpp}

The file \lstinline!miro/RangeSensorC.h! provides the classes for the interface of the
sonar service. The 'C' at the end of the name stands for client. This
file is automatically generated from the idl-description of the interface.

\lstinputlisting[frame=tb, first=6, last=6]{SonarPoll1.cpp}

A standard main function is used.

\lstinputlisting[frame=tb, first=8, last=8]{SonarPoll1.cpp}

This class just wraps the few lines that are necessary at the beginning of any CORBA
application. It will therefore be normally instantiated in all \miro
applications first. This is a simple helper class that sets up the
CORBA environment in a standard way. Nothing tricky there.  It
initializes the orb etc. The call uses the command line arguments for
finding the naming service, and parsing other commands...

\lstinputlisting[frame=tb, first=9, last=10]{SonarPoll1.cpp}

This initializes a proxy object to the
sonar service. This proxy object is used as if it was the sonar service
itself. Let's look at the call a bit closer. \lstinline!Miro::RangeSensor\_var!
is the CORBA equivalent of a standard C++ auto pointer for a
\lstinline!Miro::RangeSensor! object. That is the proxy becomes automatically
destroyed if the pointer goes auto of scope. The method
{\tt resolveName()} is a template member function. Its argument
specifies the name that shall be resolved in the default naming
context of the naming service (see Chapter \ref{sec:using}). The
\lstinline!Miro::RangeSensor! in French brackets specifies the type of
reference that shall be returned by the call. Note that this will only
succeed if the reference stored under the name of Sonar refers indeed
to a {\tt Miro::RangeSensor} object or to a derived ancestor of this
class.

\lstinputlisting[frame=tb, first=11, last=11]{SonarPoll1.cpp}

The RangeSensor interface returns a pointer to the sensor scan, which
the caller obtains ownership of. So we use another auto pointer to
hold the return value of the call and to prevent us from memory leaks.
The IDL in the type name reflects the fact, that this is a IDL defined
data type. (This is just a naming convention of \miro that shall help
you to trace the roots to the documentation.) In the CORBA mapping for
C++ these types are mapped to plain C structs: No methods, no
inheritance, just public data members.

\lstinputlisting[frame=tb, first=13, last=13]{SonarPoll1.cpp}

This gets a sonar scan from the service. Note, that we do not see where 
the service runs.

\lstinputlisting[frame=tb, first=15, last=20]{SonarPoll1.cpp}

These line are used to print the received sonar scan to the console.

%\lstinputlisting[frame=tb, first=22, last=22]{SonarPoll1.cpp}

%Ends the program. Returning zero indicates, that no error occured.

%%\bigskip
%%- Note the {\tt Miro::} prefixes which define the namespace ...


\section{Using Namespaces}

By inserting the lines

\lstinputlisting[frame=tb, first=6, last=9]{SonarPoll2.cpp}

we map the relevant ``Miro'' data types from the \lstinline!Miro!
namespace into our global namespace. This shortens the type name
specifiers, but also hides from where they are coming from. Note that
a simple

\begin{lstlisting}[frame=tb]{}
using namespace Miro;
\end{lstlisting}

would do the same trick. But mapping a namespace completely is
generally not a good idea, since it tends to produce name conflicts
and secondly code reviewers can trace the origin of the types less
easily. Be especially careful in header files. You are flattening the
namespace for everyone that has to include your header file, which can
lead to bad surprises.

See the following listing which shows the same code again, but without
the \lstinline!Miro::! prefixes. The initial lines containing the includes
are skipped.

\lstinputlisting[frame=tb, first=6, caption={examples/sonar/SonarPoll2.cpp}]{SonarPoll2.cpp}
\label{lst:SonarPoll2}


\section{Adding Exception Handling}

Handling of error conditions itself is error prone. Exceptions make
the handling of error conditions easier, but errors are errors and
therefore stay somehow nasty. What's quite easy to achieve by the use
of exceptions is to print some diagnostic output and exit instead of
gracefully segfaulting. This is done in this example by enclosing the
code in the main function in a try/catch block.

Since \miro defines ostream operators for every \lstinline!Miro::...IDL!
type in \lstinline! miro/IO.h!, we include that file and abandon the
handcrafted streaming of the sonar data.

Note the different kinds off exceptions. There are \miro exceptions.
Exception types defined in \miro are beginning with a big E as a
naming convention. These indicate problems on the service side, like
hardware problems (may be the batteries?), bad service calls (trying
to accelerate the robot to warp 1?) or load problems. Then there are
CORBA exceptions. Those occur if there arise some communication
problems: A service went down, the robot is loosing the radio ethernet
connection etc. Since all \miro
exceptions derived from \lstinline!CORBA::UserException!, those also
are catched within the first \lstinline!catch! block of the exsample
code. You do not have to catch all exceptions. An uncaught exception
will lead to program termination. Just as if you'd catch them at the
end of main and exit...

Note also, that the instantiation of Client isn't within the try/catch
block. This is intentionally. Exceptions that can arise in the
construction of a Client instance are CORBA exceptions. The ostream
operators for CORBA exceptions used by TAO need an ORB
instance. Since the ORB is instantiated within the Client class, it
will not exist after destruction of the Client --- and this would be
done at the end of the try block. Therefore we catch exceptions within
the constructor of Client, print them to stderr and exit. There is
little to do for the client program anyhow, if it can't access the
services. 

\lstinputlisting[frame=tb, first=14, caption={examples/sonar/SonarPoll3.cpp}]{SonarPoll3.cpp}
\label{lst:SonarPoll3}

\section{An Asynchronous Sonar Client}
\label{sec:notify}

By now we were actively requesting for the data of a service. But
think of polling for tactile events that way. Having a good collision
avoidance, those events should hardly ever occure. Nevertheless, as
soon as there is a tactile signalling a collision, the robot should
emediately react to this event. By polling it would have to call the
tactile interface thousands of times, just not to miss the one event
it can't effort to miss. And since the sensors are actively collecting
their data, shouldn't they be able to trigger the data processing
within the robot? - Oh yes, they can.

For this purpose the notification framework within \miro does
exist. It is based on the CORBA Notification Service \cite{OMG:00-5},
and precustomized by some utility classes. They enable clients to
subscribe to arbitary events of a notification channel. The data gets
pulled to them as soon as it becomes available at the producer (e.g. a
range sensor device).

To show how asynchronous event processing works within \miro, lets
look at a small example. First we look at the code to handle the
events.

\lstinputlisting[frame=tb, first=24, last=53, caption={examples/sonar/SonarNotify.cpp}]{SonarNotify.cpp}
\label{lst:SonarNotify}

A step by step walk trough the code:

\lstinputlisting[frame=tb, first=24, last=24]{SonarNotify.cpp}

The class \lstinline!SonarNotify! provides a callback for the event
channel. It is derived from \lstinline!Miro::StructuredPushConsumer!,
which handles the registration at the notification service etc. All
the initialization and registering is done within the constructor of
this super class. The method \lstinline!push_structured_event! is the
callback that is called from the notification service.

\lstinputlisting[frame=tb, first=31, last=31]{SonarNotify.cpp}

The \lstinline!EventTypeSeq! class is for specifying the events that you
want to be subscribed for. It is an incremental protocol. So you make
a list of events you want to receive from now on and another with the
events you wish to no longer become informed about. Since we are just
subscribing, the second list is empty.

\lstinputlisting[frame=tb, first=33, last=35]{SonarNotify.cpp}

Events are subsrcibed by \lstinline!domain_name! and
\lstinline!type_name!. Never forget to set the length of a sequence
explicitly. Specifying the length as a constructor parameter just
reserves the number of elements, the length of the sequence is still
zero.

\lstinputlisting[frame=tb, first=37, last=37]{SonarNotify.cpp}

Tell the consumer admin what we want to subscribe for.

\lstinputlisting[frame=tb, first=48, last=49]{SonarNotify.cpp}

The payload of a structured event is contained within the
\lstinline!remainder_of_body! field with is of type
\lstinline!CORBA::Any!.  Therefore you can get a const pointer to the
data with overloaded the \lstinline!operator >>= ()!. The return value
is a \lstinline!CORBA::Boolean! indicating success of the operation.
Even though a \lstinline!CORBA::Any! can hold any IDL defined struct,
you can only extract type \lstinline!T! from a \lstinline!CORBA::Any!
if it actually contains an object of type \lstinline!T!.

\lstinputlisting[frame=tb, first=54, caption={examples/sonar/SonarNotify.cpp}]{SonarNotify.cpp}
\label{lst:SonarNotify2}

Lets now have a look at the \lstinline!main! function:

\lstinputlisting[frame=tb, first=57, last=57]{SonarNotify.cpp}

Since a push consumer is called (pushed) by the event producer, it is
actually a server instead of a client. Therefor the
\lstinline!Miro::Server!  class is instanciated. It performs the
necessary calls to init the CORBA environment. It is basicly the same
as the \lstinline!Miro::Client!  but we also need a POA to register
the consumer object.

\lstinputlisting[frame=tb, first=59, last=59]{SonarNotify.cpp}

Just like the former \lstinline!Miro::RangeSensor! interface, we
resolve the \lstinline!EventChannel! by name at the naming service.

\lstinputlisting[frame=tb, first=61, last=61]{SonarNotify.cpp}

Instanciate the push consumer. It does all the necessary
initialization within its constructor.

\lstinputlisting[frame=tb, first=63, last=63]{SonarNotify.cpp}

Now we enter the CORBA event loop. This will not return, until the
process is signaled by SIGINT or SIGTERM.

%%% Local Variables: 
%%% mode: latex
%%% TeX-master: "miro_manual"
%%% End: 

\chapter{Parameter Sets}
\label{SEC:PARAMETERS}

                                % problem

Robot control programs and robotic algorithms are full of parameters.
Different robot types vary in shape and sensor configuration, but also
different robots of the same type tend to vary slightly. For instance
some of them are equipped with further sensory or actuatory devices
not present in the standard configuration. Also damage and repair of
parts of a robot during its lifetime result in an individual robot,
which has its own unique configuration.

                                % examples

Furthermore the environment provides a complete set of
parameters. Lighting conditions, the shape and location of rooms and
corridors, the position of obstacles like tables or stairs. The list
could be continued indefinitely. These parameters vary between
different environment, but also tend to change slightly over
time. Light bulbs are changed, cupboards added, tables moved etc.

A third source of parameters are defined by the task, the robot has to
perform or the scenario, the robot is designed to operate in.
Generalized tool boxes like planners or knowledge bases have to be
populated with the actions and objects of relevance and a reactive
execution engine need to be configured with what actions to take in
which situation.

                                % summary

The configuration of these parameter sets can be done either by hand
or tool supported or fully autonomously. Nevertheless it is a crucial
part of the setup of robot for every scenario. The handling of those
parameter sets therefore has to be well defined in order to keep the
robots control software maintainable and adaptable to new tasks.

                                % solution aproach

In this chapter we will therfore introduce the parameter framework of
\miro.                                % organization of the chapter
The remaining of this chapter is organized as follows.  First the
general concept of parameter sets and configuration files is
sketched. Then the
location and naming of the robot configuration files of \miro and how to
create a configuration for an individual robot will be explained. In
the following section the description and implementation of
parameter sets and the mapping to the actual parameter files will
be discussed. In section \ref{SEC:DESCRIPTION_SYNTAX} the syntax and
semantics of the description language will be explained.

\section{Overview}

Parameter sets are organized as follows.
 
From a user perspective parameter sets are described by an XML file,
that the service parses on startup, to initialize its parameter set
classes. Parameters in general are strictly typed name value pairs.
They are grouped into sections within the file. Also, they can be
nested, structured types. The parameter names start with an upper case
initial letter.

From a programmers perspective, a parameter set is represented by a
class within C++ with all member variables declared public, just like
a C struct. The member variable names begin with a lower case initial
letter. Names consisting of multiple words are concatenated, the
beginning of a new word is indicated by an upper case letter.
Parameter classes support single, public inheritance. The following
methods are declared:
\begin{description}
\item[Default Constructor]
  Initializes all member variables to their default values.
\item[void operator $\ll =$ (QDomNode)]
  Parses an XML node and sets the member variables accordingly.
\item[QDomElement operator $\gg =$ (QDomElement)]
  Appends the current
  values of the member variables as a child to the XML node.
\item[void printToStream(ostream\&)]
  Prints the current values of the member variables to the
  specified output stream.
\end{description}
Optionally parameter set classes contain a static member method
{\tt instance()}, that returns a pointer to the global heap allocated
instance of the parameter set class.

As parameter set classes have a very generic structure, their coding,
especially the XML and ostream operators can be automated. Therefore
parameter sets normally are described not within a programming
language, but within an XML syntax. The mapping to C++ is then
performed by a small compiler, which emits the necessary code
fragments. The parameter set description can also used by tools for
comfortable editing of configurations by the user. This is done by the
policy editor (see section \ref{SEC:POLICY_EDITOR}) for editing
behaviour parameter sets.

Within \miro this framework is used for two parts of the above
sketched three sources of parameter sets. For the description of the
robots configuration and for the robots task
description. Environmental modeling for different scenarios lies out
of the scope of the problem set addressed by \miro.

\section{Configuration Files}

Every service of \miro tries to find a configuration file on startup
to adapt itself to the individual robots configuration if necessary.
If no configuration file is found, the service has to use its default
settings. The locations the configuration files are expected are:
\begin{enumerate}
\item {\tt \$MIRO\_ROOT/etc}
\item The current directory.
\end{enumerate}
The name of the configuration file is expected to be the host name of
the computer the service is running on. The file extension is .xml:

{\tt \$HOST.xml}

For example in the case of our famous Sparrow99 soccer robot goal
keeper named haeltnix, the \miro service SparrowBase, started on the
computer mounted on the robot will try to load the configuration file:

{\tt \$MIRO\_ROOT/etc/haeltnix.xml}

The default name and location of the configuration file can be
overwritten by the command line option {\tt -MiroConfigFile} 
{\em filename}. The search order of directories stays the same.
Additionally there can also be an absolute path specified.

In {\tt \$MIRO\_ROOT/etc} there are sample configuration files for
every robot model currently supported by \miro. They are named
{\em model}.{\tt xml}. Copy the respective configuration file of 
your robot model to {\tt \$HOST.xml} to get a configuration file to 
start with. XML files are text files and quite easily
readable. Therefore opening the configuration file into your favorite
editor and changing the {\em value} attributes to different settings
should be sufficient to solve the first few configuration problems
like differing device names etc. For anything else, read on.

\subsection{Configuration File Syntax}

The available tags and there nesting capabilities look like follows:

$<$!--MiroConfigurationDocument--$>$

\begin{description}
\item[configuration]
  One file always describes one entire configuration. A
  configuration describes all the services parameters, that can be
  executed on the particular robot.
  
  There can only be one configuration tag per file.
  
  \begin{description}
  \item[section] Each component of \miro can have its own section
    within a configuration file. Configuration file sections are kind
    of a weak concept. For instance the vision system is considered to
    be a component within a \miro configuration file. Normally each
    service defines its own section within the configuration file.
    
    There can be multiple section tags within a configuration.
    
    Attributes:
    \begin{itemize}
    \item \textit{name} The name of the section.
    \end{itemize}
    
    \begin{description}
    \item[parameter] Parameters of a component are grouped and
      described as set of parameters that belong together. Note that
      within the configuration file \miro takes a hardware centric
      view. That is, sensors and actuators that result in multiple
      service interfaces, are most likely to be grouped into one
      component if they are controlled by the same low level
      controller board. For instance the parameters for the pioneer
      board, that controls the motors as well as the sonars etc., are
      grouped within just one component.
      
      There can be multiple parameter tags within one section.
      
      Attributes: 
      \begin{itemize}
      \item \textit{name} Name of the parameter.
      \end{itemize}  
      
      \begin{description}
      \item[parameter] Parameters can be specified within the configuration
        file for each component.
        
        There can be multiple nested parameter tags.
        
        Attributes: 
        \begin{itemize}
        \item \textit{name} Name of the parameter.  
        \item \textit{value} The value of the parameter.
        \end{itemize}
        
        In case the parameter describes a structured type instead of a
        basic type (like bool, int, string etc.), parameter tags can
        be nested. In this case the value attribute is ignored.
        
      \end{description}  
    \end{description}
  \end{description}
\end{description}

\subsection{Example Configuration}

See {\tt \$MIRO\_ROOT/examples/params/TestConfig.xml}
for a small initial example.

TODO: include and comment the example code here.

Section tags are structurising elements and component parameter tags
are mainly defined by the underlying hardware devices. But what
defines which parameters are accepted within by which component and of
what types they are? While the configuration files contain comments to
explain the purpose of the different parameters, there do also exist
formal definitions of the components parameter sets, as described
within the next section.

\section{Parameter Set Generation}

Parameter sets, described as component parameters within a configuration file
correspond to strictly typed classes within C++. Those structures can
be initialized by parsing the configuration file. To facilitate the
generation of parameter sets, parameter classes needn't to be
hand coded, but are auto-generated from parameter description files by a
little parameter class compiler ({\tt MakeParams}). The parameter
description files are once again based on the XML syntax.

Furthermore the description files are not only used for the
configuration of the services of \miro but also by other tools within
the \miro framework. The behaviour framework (sec.\ 
\ref{SEC:BEHAVIOURS}) uses this functionality for parameterization of
behaviours and arbiters and the {\tt PolicyEditor} (sec.\ 
\ref{SEC:POLICY_EDITOR} uses the parameter description files for GUI
based editing of the behaviour parameters). To facilitate the reuse of
the parameter descriptions by multiple tools, the syntax of the file is
somehow obfuscated by additional tags and attributes.

\subsection{Description File Syntax}
\label{SEC:DESCRIPTION_SYNTAX}

Parameter classes support single inheritance.

$<$!--MiroParametersConfigDocument--$>$

\begin{description}
\item[config]
  A parameters config file describes the types, names and defaults
  for parameter classes as used within the \miro framework. 

  \begin{description}
  \item[config\_group]
    Configurations can be grouped together for various reasons. 
    A group within a description file in general corresponds to
    a section within the configuration file.

    Attributes:
    \begin{itemize}
    \item \textit{name} The name of the config group. This corresponds
      normally to the section used later within the configuration file,
      but for base classes, the config\_group can be chosen
      arbitrarily.
    \end{itemize}

    \begin{description}
    \item[config\_item] A configurable item to built a parameters
      class for. Each config item will be mapped to a parameter class
      by the compiler.

      Attributes:
      \begin{itemize}
      \item \textit{name} The name of the parameter class. The suffix
        Parameters will be added to the name.
      \item \textit{parent} The name of the super class. The suffix
        Parameters will be added to the name.
      \item \textit{instance} Whether or not a singleton instance
        of the parameter class shall be created (sec.\
        \cite{SINGLETON}). Default is {\em false}.
      \item \textit{final} Corresponds to the finalize keyword of
        java. It is only relevant for the editing tools, not for 
        code generation. Default is {\em true}.
      \item \textit{dummy} Sometimes you have to fake a parameter
        class to the editing tools that does not exist as such. No
        code will be generated for such a class. Default is {\em
          false}.

      \end{itemize}
      
      \begin{description}
      \item[config\_parameter] A parameter for the class.

        Attributes:
        \begin{itemize}
        \item \textit{name} Guess what.
        \item \textit{type} Allowed basic types are: bool, char, int,
          unsigned int, short, unsigned short, long, unsigned long,
          double and std::string. There are two more
          predefined types: Angle and angle. {\em Angle} corresponds
          to a Miro::Angle instance. It is an angle $\phi$ normalized
          to $-\pi \le \phi \le \pi$ while {\em angle} corresponds to
          a none normalized angle represented by a simple double. Note
          that while the internal representation of angles is in
          radiant, the values specified in the configuration files are
          in degrees. But you can also specify parameter
          classes described within a parameter description file as
          parameter type. Additionally, {\em std::vector$<>$} and {\em
            std::set$<>$} are supported. Note
          however that we are currently lacking support for specifying
          default values for these none basic types.
        \item \textit{default} default value as set in the
          constructor. (optional)
        \item \textit{measure} The measure the type represents, such
          as: mm, msec, �, mm/s, �/s. (optional)
        \item \textit{inherited true/false}
          This attribute is used to 
          overwrite defaults for inherited variables within the
          constructor of the class. If set to true, no new member
          variable will be added to the class, and only the
          attributes name and default will be evaluated.
        \end{itemize}
      \end{description}
    \end{description}
  \item[config\_global]
    Global configuration items those are mainly used for code
    generation. They specify things, that the compiler can not easily
    derive autonomously, such as additional include directives, etc.

    Attributes:
    \begin{itemize}
    \item \textit{name} The name of the config item. There are
      currently three items supported:
      \begin{description}
      \item[namespace] Specifies the namespace of the generated
        code. While multiple parameter classes can be described
        within one file, they all have to reside within the same
        namespace.
      \item[include] Specifies a user include directive.
      \item[Include] Specifies a system include directive.
      \item \textit{value} The name of the namespace, the local or
        global include directive.
      \end{description}
    \end{itemize}
  \end{description}
\end{description}

\subsection{Example Description}

See {\tt \$MIRO\_ROOT/examples/params/Parameters.xml}
for a small initial example.

TODO: include and comment the example code here.

\subsection{MakeParams}

The parameter description compiler has the following command line
parameters.
\begin{description}
  \item[-f $<$file$>$] name of the input file (default is {\em Parameters.xml})
  \item[-n $<$name$>$] base name of the output file (default is {\em Parameters})
  \item[-s $<$extension$>$] extension of the generated source file (default is {\em cpp})
  \item[-h $<$extension$>$]extension of the generated header file (default is {\em h})
  \item[-v] verbose mode
  \item[-?] help: emit this text and stop
\end{description}

\subsection{Example Header File}

See {\tt \$MIRO\_ROOT/examples/params/Parameters.h}
for a small initial example.

TODO: include and comment the example code here.

\subsection{Makefile magic}

To enable automatic generation of the C++ source and header files from
XML parameter descriptions, there does allready exist a rule within
the make files of \miro. That is, for every file named *Parameters.cpp
| h, make checks the existance of the corresponding *Parameters.xml
file and if necessary, invokes MakeParams accordingly. To prevent make
from removing the generated files at the end of the task, you have to
define an additional .PRECIOUS target like this:

.PRECIOUS: MyParameters.cpp MyParameters.h

%%% Local Variables: 
%%% mode: latex
%%% TeX-master: "miro_manual"
%%% End: 

\chapter{Video Image Processing}

Video image acquisition is one of the most standardized sensory
devices used in robotics. Therefore \miro offers a common
{\tt VideoService} that can be adapted by the configuration file to any
supported video device.

Also, image processing is computationally expensive and many standard
filters exist in literature as well as some high performance
implementations such as \cite{IPP}. To facilitate the unified usage
within robotics scenarios, the video service provides a filter tree
framework.

Real-time image streams need a lot of bandwidth. This is usually
higher than the bandwidth available on most mainstream network
devices. Therefor the {\tt Video}-interface of \miro also provides
methods for usage of shared memory for sharing of images with its
clients on the same machine. Those methods are also designed for
minimizing copying overhead.

This section is organized as follows. First, the supported image
devices along with their bugs and features are discussed. Then the
filter tree framework is presented. First from an end user
perspective and second from a developer perspective. In section
\ref{SEC:VIDEO_INTERFACE} the video interface itself along with its
two use cases and helper classes is discussed.

\section{Video Device Access}

The VideoService currently supports the following devices for images
acquisition:

\subsection{Bttv Frame Grabbers}

These frame grabber cards are supported via video for linux \cite{}.
They are the standard way of connecting standard analog video cameras
to the computer.

Note, that the default number of frames used by the kernel driver for
video capture are two. In order to get the full frame rate (25/30Hz),
this has to be set to 4 buffers. This can be specified as a module
parameter.

\subsection{Firewire Digital Cameras}

\miro supports the fire wire digital camera protocol using
libraw1394\cite{} and libdc1394 \cite{}.
  
Note that most cameras 1394 controllers do not support being bus
master for dma transfers. Therefore the 1394 controller of the
computer has to be the highest numbered node. This can be configured
by a module parameter. Otherwise, unplugging the firewire cable from
the camera for some seconds helps.

\subsection{Matrox Meteor Frame Grabbers}

These rather old frame grabber cards are also supported by \miro.
Kernel drivers can be found at \cite{}. This device however, is mostly
unmaintained within \miro.

\section{Video Filter Trees}

For image processing many standard filters do exist that usually
perform a function $M_{n+1} = f_i(M_{n})$. Where $M_n$ denotes the input
image (matrix), $M_{n+1}$ the output image and $f_i$ is the actual
filtering function. 

Often filter chains are used. I.e. first undistorting
the image $f_0$ and then performing some color indexing $f_2$. So the
filtering function producing the output image $M_o$ from the input
image $M_i$ could be expressed as:  $$M_o = f_2(f_1(M_i))$$. 

More complex vision processing could also include the computation of
an edge image $M_e$ in addition to the color indexing above. This can
be done i.e. by producing a grey image $f_3$ and applying a canny
filter $f_4$, leading to the second filter chain: 
$$M_e = f_4(f_3(f_1(M_i)))$$.

To avoid computing the undistorted image twice, a filter tree should
be used instead of two separate filter chains.

% TODO: filter tree image.

\section{VideoService}

\miro comes with a set of basic filters. The term filter is used here
in a very general way, as almost every processing unit of the video
service is designed as a filter. Actually the image acquisition from
the different video devices are implemented as a filter, forming the
first set of filters available in \miro. The second set of filters are
basic image conversions, like byte order swapping, YUV to RGB
transformations etc. The third set of filters are the classic image
filters. Unluckily, there do not exist any of them currently in \miro.
It introduces further library dependencies for the middleware, which
we have to be careful about. And i.e. for Ipp based filters, we also
have a licensing problem. How to extend the video service to process
your own filters is covered in the following subsection.

Filter trees and their individual parameters can be specified within
the configuration file of the robot. 

\subsection{VideoService Parameters}

The video service expects its configuration parameters with the
section Video of the configuration file. Its parameter is
also called Video. The
video service has the following parameters:

\begin{description}
\item[Width] The width of the input image (unsigned int).
  It has to be supported directly by the used video device.
\item[Height] The height of the input image (unsigned int).
  It has to be supported directly by the used video device.
\item[Palette] The palette of the input image (std::string).
  It has to be supported directly by the used video device.
  The recognized palettes of the video service are:
  \begin{description}
  \item[grey] 8 bit grey values.
  \item[grey16] 16 bit grey values.
  \item[rgb] 24 bit RGB format.
  \item[bgr] 24 bit BGR format (byte swapping).
  \item[rgba] 32 bit RGB format (plus alpha channel).
  \item[abgr] 32 bit BGR format (plus alpha channel).
  \item[yuv] 24 bit YUV format.
  \item[yuv411] compressed YUV format from 1394.
  \item[yuv422] compressed YUV format from 1394.
  \end{description}
\item[Filter] The root filter of the filter tree. The filter parameter
  has three entries:
  \begin{description}
  \item[Type] The filter type name (std::string).
  \item[Name] The name of the corresponding configuration file
    parameter entry (std::string).
    \item[Successor] The list of successor filters (Filter).
  \end{description}
\end{description}

\subsection{Video Filter Parameters}

For each filter in the filter tree, filter parameters can be given
within the parameter whose name matches the filter name given in the
filter tree. Those are also located within the section Video.

The base parameter set of each filter is as follows:

\begin{description}
\item[Name] The name of the filter (std::string).
  Somehow redundant. --- Will have to check.
\item[InterfaceInstance] Whether the filter has an instance of the
  video interface associated with it (bool).
  Each filter can have an instance of the video interface. That way
  every single node of the filter tree can be passed on to the client.
\item[Interface] The parameters of the interface. For each interface
  the following parameters can be specified:
  \begin{description}
  \item[Name] The name under which the interface is registered at the
    naming service (std::string). The default is Video.
  \item[Buffers] The number of memory buffers, reserved by the interface
    (unsigned int). The default is 4. While a client is accessing an
    image, the video service guarantees, that it does not become
    overridden with a new image from the video stream. If clients hold
    multiple images (i.e. for temporal integration) it has to be made
    sure, that there is still a buffer left, to put the next image from
    the video stream into.
  \end{description}
\end{description}

\subsection{Video Device Parameters}

As mentioned above, all supported video devices are implemented as
filters. Therfore they inherit the base parameters. Note however, that
the device filters are not capable of having an interface
instance. Sharing memory mapped files is a bit tricky, and we didn't
want to go into that, yet. The common DeviceFilter parameters are:

\begin{description}
\item[Device] The fully qualified path of the video device
  (std::string). The individual device filters set this parameter to
  a common default (like /dev/video/0 for bttv and /dev/video1394/0 for
  firewire).
\end{description}

The available video device filters are (listed by type name):

\begin{description}
\item[VideoDeviceBttv] The bttv device filter defines the following
  additional parameter:
  \begin{description}
  \item[Subfield] The subfield chosen (std::string). As the PAL and
    NTSC images are interleaved, the bttv frame grabbes can select,
    which of the half images to scan, if the height of the output
    image is half the height of the full image (384). Possible values
    are odd, even and all. Note however, that many frame grabber cards
    do not support this feature. In this case, use the FilterHalfImage
    described below.
  \end{description}
\item[VideoDevice1394] The firewire digital camera standard gives
  access to many of the internal camera parameters. These can be
  configured in the parameter section of the 1394 device. For all
  these values -1 denotes, that this parameter should be controlled by
  the camera.
  \begin{description}
  \item[Buffers] The number of image buffers used for dma transfers
    (unsigned int).
  \item[Brightness] (int).
  \item[Exposure] (int).
  \item[Focus] (int). Default is auto.
  \item[Framerate] (int). Default is 30Hz.
  \item[Gain] (int).
  \item[Gamma] (int).
  \item[Hue] (int).
  \item[Iris] (int).
  \item[Saturation] (int).
  \item[Sharpness] (int).
  \item[Shutter] (int).
  \item[Temperature] (int).
  \item[Trigger] (int).
  \item[WhiteBalance0] (int).
  \item[WhiteBalance1] (int).
  \end{description}
\item[VideDeviceMeteor] The matrox meteor device filter does not
  define any additional parameters.
\end{description}

\subsection{Basic Video Filters and Their Parameters}

\miro also provided the following basic filters with the standard
VideoService.

\begin{description}
\item[FilterCopy] As the video device filters cannot map the image
  data directly into shared memory for access by the clients, this
  filter is provided. It copies the image internally, to allow for an
  interface instance.
\item[FilterSwap3] Byte swapping for 24 bits per pixel images (BGR to RGB).
\item[FilterSwap4] Byte swapping for 32 bits per pixel images (ABRG to
  RGBA).
\item[FilterFlip] We have a camera that's mounted upside down. This
  filter flips the image. Also useful if your fly imitating robot has
  successfully landed on the ceiling.
\item[FilterHalfImage] This is a filter to extract a half image from an
  interlaced image as the bttv frame grabbers provide. It copies each
  second scan line. It defines the
  following additional parameter:
  \begin{description}
  \item[Odd] Take the second half image by starting with the second
    line (bool). Default is false.
  \end{description}
\end{description}

\subsection{Configuration Example}

Putting it all together the configuration section for the {\tt
  VideoService} might look like the following. This is actually the
configuration of our Performance PeopleBot. It has an analog video
camera and a bttv frame grabber card. The camera is mounted upside
down.

\begin{verbatim}
<!-- The video configuration section. -->
<section name="Video" >

 <!-- Parameter section of the VideoService -->
 <parameter name="Video" >

  <parameter value="bgr" name="Palette" />         <!-- Input image palette. -->
  <parameter value="384" name="Width" />           <!-- Input image width. -->
  <parameter value="288" name="Height" />          <!-- Input image height. -->
  <parameter name="Filter">                        <!-- Filter tree root. -->
   <parameter value="DeviceBTTV" name="Type" />    <!-- It's a bttv device. -->
   <parameter value="DeviceBTTV" name="Name" />    <!-- Params section name. -->   
   <parameter name="Successor" >                   <!-- Filter tree leafs. -->
    <parameter name="Filter" >
     <parameter value="FilterSwap3" name="Type" /> <!-- Byte swapping filter -->
     <parameter value="FilterSwap3" name="Name" />
     <parameter name="Filter" >                    <!-- Filter tree leafs. -->
      <parameter value="FilterFlip" name="Type" /> <!-- Upside down filter. -->
      <parameter value="FilterFlip" name="Name" />
     </parameter> 
    </parameter>
   </parameter>
  </parameter>
 </parameter>

 <!-- Parameter section of the bttv device. -->
 <parameter name="DeviceBTTV">                   
  <parameter value="/dev/video0" name="Device" /> <!-- Path of the device. -->
 </parameter>

 <!-- Parameter section of the byte swapping filter. -->
 <parameter name="FilterSwap3">
  <parameter value="true" name="InterfaceInstance" /> <!-- Video interface. -->
 </parameter>

 <!-- Parameter section of the upside down filter. -->
 <parameter name="FilterFlip">
  <parameter value="true" name="InterfaceInstance" /> <!-- Video interface. -->
  <parameter name="Interface">                        <!-- Interface params. -->
   <parameter value="Flipped" name="Name" />          <!-- Interface name. -->
  </parameter>
 </parameter> 

</section>
\end{verbatim}

\section{QtVideo}

{\tt QtVideo} is a test client for the {\tt VideoService} and the {\tt
  Video} interface. It displays an image stream from a {\tt Video}
interface instance and has an additional button, to save snapshots to
disk.

The {\tt QtVideo} tool accepts the following command line parameters:
\begin{description}
\item[-n] Name of the {\tt Video} interface instance within the CORBA
  naming service. Default is {\em Video}.
\item[-r] Remote access of the images. QtVideo will be using the
  methods for location transparent image access of the {\tt Video}
  interface. These are much slower, than the methods using shared
  memory buffers, but are the only option when running QtVideo on
  another machine.
\item[-v] Verbose mode.
\item[-?] Emitting command line help and exits.
\end{description}

\section{Video Interface}

The {\tt Video} interface is used to manage the access of image data
by client programs. It supports location transparent as well as
optimized local image access. Additionally connection management is
used, to switch of filter subtrees that are not accessed by any other
program.

The general access pattern of a client of a {\tt Video} object looks
like follows:
\begin{enumerate}
\item Get a {\tt Video} interface IOR from the naming service.
\item Connect to the {\tt Video} object.
\item Get images until done.
\item Disconnect from the {\tt Video} object.
\end{enumerate}

Note that due to the performance centric design of the Video
interface, the VideoService can easily be jammed by client programs
violating the access protocol of the interface. To facilitate the
correct usage, some helper classes are provided by \miro. These are
discussed in section \ref{SEC:C++HelperClasses}.

\subsection{Location Transparent Image Access}

To access an image as a client running on a different computer than
the {\tt VideoService}, the {\tt Video} interface offers the methods
\begin{itemize}
\item {\tt exportSubImage}
\item {\tt exportWaitSubImage}
\end{itemize}
They both return a copy of the image as return value. The first method
returns immediately the current image, while the second one waits until
a new image becomes available before returning. The image will be
scaled down to the size specified by method parameters.

Note, that due to the copying and network overhead of those methods,
they are only useful for debugging and monitoring purposes.

\subsection{Local Image Access}

For clients running on the same machine as the {\tt VideoService}, the
{\tt Video} interface offers the following methods for image access
via shared memory buffers:
\begin{itemize}
\item {\tt acquireCurrentImage}
\item {\tt acquireNextImage}
\item {\tt releaseImage}
\end{itemize}
While the first method returns immediately the buffer of the current
image, the second one waits until a new image becomes available before
returning. Note that the clients have to release each image buffer
after processing. Otherwise the VideoService will soon run out of
buffers, to share new images with its clients.

\subsection{C++ Helper Classes}

\subsection{Example Video Client}

\lstinputlisting[frame=tb, first=12, caption={examples/video/VideoExample.cpp}]{VideoExample.cpp}
\label{lst:VideoExample}

\section{Writing Filters}

\subsection{The Filter Base Class}

\subsection{Virtual Methods to Overwrite}

\subsection{Parameter Macros}

\subsection{Example Video Filter}

%%% Local Variables: 
%%% mode: latex
%%% TeX-master: "miro_manual"
%%% End: 

\chapter{Behaviour Engine}

Controlling the actuators of an autonomous mobile robot is one of the
central aspects of mobile robot research. As \miro exposes the
interfaces to the motor controllers etc. of the mobile platform it
enables researchers to easily evaluate new approaches to model sensor
actor control loops. Be aware that at this point the latencies
introduced by various levels of the robot architecture can become a
critical issue. To eliminate the network latency and prevent yourself
from occasional network bandwidth problems we recommend that you run
your control programs collocated on the same computer with the service
that accesses the actuators device.

\miro supports the behavioural control paradigm introduced by Brooks
\cite{Brooks} by its own behaviour engine. It is designed to allow for
a quick start into behaviour robotics writing your own behaviours. Yet,
due to its open and extensible design it is also capable of handling
sophisticated control tasks as demonstrated by its use within the
RoboCup-scenario by our \sparrow robots.

\section{The Concept of Behaviours}

The basic idea of the behaviour approach to robot control is as
follows. Instead of the sense-plan-act paradigm of classical AI, the
task is splitted into a set of reactive behaviours, that each try to
fulfill a small subtask of the problem set. For each of the task only
a very limited part of world modeling is needed (often, even raw
sensor readings are sufficient). By combining the output of the
various behaviours by an arbiter, the emergent higher level behaviour
of the system is achieved, solving the recommended high level task.

An accepted bottleneck of this approach is the arbitration and
calibration of large sets of behaviours necessary to fulfill different
aspects of a high level task. Therefore in the \miro framework the
behaviour engine also support the hierarchical decomposition of
behaviour sets by allowing them to be grouped in so called action
patterns that can be activated alternately by so called transition
messages. A set of action patterns is called a policy within the \miro
framework.

\section{Introductory Examples}

Lets look at the simple action pattern, who's high level task is to
explore the environment by performing a random walk. This is easily
splitted into two distinct behaviours. The first subtask is not to
collide with the environment. This can be achieved by an avoid
behaviour, that reads the front sonar sensors to determine how far it
is away from the nearest obstacle. If the minimal distance is below
some threshold, it tells the arbiter to turn away into another
direction. The second behaviour would be a wander behaviour. It selects
from time to time just randomly some translational and rotational
velocity, making the robot move around. It does not have to care about
obstacles, since those are taken care of by the avoid behaviour.

The task of choosing the actual velocities to be applied to the motors
is performed by the arbiter. It therefore plays a central role in the
behaviour approach. There exist various kinds of arbiters, all
choosing different policies for this task. But a simple priority based
arbiter suffices for many scenarios. In \miro there currently just
exists a priority based arbiter, but since it is an extensible
framework you can easily plug in your own one. The priorities in this
example could be applied straight forward. The avoid behaviour has
higher priority as the wander behaviour.

Note the easy extensibility of this approach. For example, if we have
some bump sensors that indicate collisions with the environment. We
could just add another, let's call it emergency stop behaviour and
assign it the highest priority.  If one of the bumpers is pressed it
makes the robot stop and wait for rescue by one of the operators.

As a further extension we assume we have two tasks. The first is the
random walk described above, the second would be a wall following
behaviour, that simply is capable to drive the robot in a defined
distance along a wall. The decision what to do is provided by an
external source, say, a button located on the robot and pressed by the
operator when demonstrating the management the capabilities of the
newly bought autonomous mobile robot. The second action pattern looks
quite like the first one, except that the wander behaviour is
exchanged by a wall following behaviour, that drives the robot along
the wall. (Note that this behaviour does not have to care about walls
thar are blocking the way at the end of the corridor, since this can
be taken care of by the avoid behaviour.) These two action patterns
now form a policy. The pressing of the button sends a transition
message, that disables the currently running action pattern and
enables the other one.

The action pattern / transition message mechanism also fits naturally
for coupling reactive behavioural control with deliberative planning
architectures. The transition could also be raised by a path planner
and the corresponding action patterns could be 'move to point', 'drive
through door' or 'dock at power supply'. Indeed this was already done
within \cite{Hans-Diplom}.

\section{Implementing a Behaviour}

All behaviours are derived from the base class {\tt Miro::Behaviour}. The
central method of this class is the method {\tt action()}. It is to be
overwritten by the programmer and has to contain the behavioural code.
The method becomes invoked by
the behaviour framework, and is expected to call an arbitration method
or send a transition message. Note the inversion of control flow. A
behaviour is not allowed to jump into some infinite loop, but the
{\tt action()} method is called continously as long as the behaviour is
active within some action pattern.

From an implementation perspective, there are three kinds of
behaviours, depending on how the data, the behaviour bases its
decisions on, is delivered.

\subsection{Miro::TimedBehaviour}

This is the base class for a timer scheduled behaviour. A behaviour
derived from this class runs with all its brothers and sisters
cooperatively multi-threaded in one thread of control. The pace at
which its {\tt action()} method is called is selectable by a parameter of
the base class.

This is the most simple form of behaviour design and in many cases most
straight forward. It is especially suitable for behaviours that poll
their sensory information of the world model, like for instance the
current scan of the laser range finder for collision avoidance. Also
the above mentioned wander behaviour - which doesn't use any sensory
information would be best implemented as a child of the
{\tt Miro::TimedBehaviour} class.

See {\tt \$MIRO\_ROOT/examples/b21Behaviours/wander.cc/hh} for a
complete source code example.

\subsection{Miro::EventBehaviour}

This is the base class behaviours using asynchronous sensory
information published by the Notification Service. It subscribes for
the events it likes to get pushed and the action method is called
whenever a new message arrives. A pointer to the current structured
event is then available as member variable.

The emergency stop behaviour described
above would be a good candidate for such a behaviour, since it would
have to poll excessively as a timed behaviour in order to minimize the
latency between a bumper pressing a the actual stop, while it would
only need to call an arbitration method in very rare occasions.

See {\tt \$MIRO\_ROOT/examples/b21Behaviours/tactileStop.cc/hh} for a
complete source code example.

\subsection{Miro::TaskBehaviour}

Behaviours derived from this calls run within their own thread of
control, not blocking others even if they need fairly long for their
decisions. Note that such a behaviour is likely to be miss designed,
since it contradicts the behavioural approach to need excessive time
to come to a decision. But if you have need for something like this,
since you are doing something we didn't think of, this is the class to
base your behaviour on. Note however, that since behaviours are shut
down cooperatively, also a task behaviour is not allowed to loop
indefinitely within its action method. If the behaviour is still part
of the currently running action pattern, its action method will be
call immediately after giving control back to the behaviour framework.

\subsection{Behaviour Parameters}

Behaviours usually get used within different action patterns. But they
are often expected to behave slightly different within each
constellation. Therefore each behaviour has an associated
{\tt BehaviourParameters} class which is designed to hold the different
parameter sets for the different use cases of a behaviour. These
parameter classes become initialized on startup of the behaviour
framework and are expected to be static during the run of an entire
policy. (How to handle dynamic parameters like destination coordinates
of a 'move to position' behaviour is explained in section \ref{sec:init})

Each behaviour is expected to have its own {\tt BehaviourParameters}
class factory method, for dynamic instantiation. In the constructor,
the default settings of the parameters are expected to be set. There
are two more methods for each {\tt BehaviourParameters} class to be
populated. The first is a method to parse its parameters from an xml
DOM tree, the second is for debug output. If your behaviour doesn't
need any additional parameters, you can skip this work. Its parents
Parameters class will then be used. --- The {\tt BehaviourParameters}
framework is about to be subject to automatic code generation (see
section \ref{sec:makeParams}). So hopefully you don't have to worry
about it too much.

\subsection{Behaviour Initialization}
\label{sec:init}

Before an action pattern becomes activated all its behaviours {\tt
  init()} methods are called after the call to its {\tt init()}
method.. This allows behaviours to initialize their per task
parameters (like destination coordinates) in a convenient way. Note
however, that the {\tt init()} method can be called while the
behaviour is already active (see section \ref{sec:active-inactive} and
so its {\tt action()} method can be concurrently running. Therefore a mutex
is needed to avoid race conditions.

Parameters that are valid for the whole lifetime of a behaviour, such
as references to the robots services or other objects within the
behaviours address space are best to be passed during construction
of the behaviour, forming a so called initializing constructor.

\subsection{Behaviour Activation and Deactivation}
\label{sec:active-inactive}

When a behaviour is to become active due to it being part of an action
pattern its {\tt open()} method is called. If the behaviour is no longer
part of the next to be running action pattern its {\tt close()} method is
called. If a transition from one action pattern to another is
performed and the behaviour is part of both behaviour sets, than no
calls to {\tt close()} and {\tt open()} methods will be issued. Only the
behaviours {\tt init()} method will be called, to allow it to update its
parameter sets. This is useful to avoid unnecessary behaviour
shutdown.

Note also, that the {\tt close()} method can be called while the behaviour
is concurrently within the {\tt action()} method. --- Its call to the
arbiter will then just be ignored. On the call of {\tt open()} however it
is guaranteed that the behaviour currently is not running.

\section{Arbiters}

The arbiter framework is similar to the behaviour classes. It will be
explained in more detail as soon as we do some more work on arbiters.

\section{Building Action Patterns}

``Now I built all my behaviours. What code do I have to write to make
them an action pattern?'' Well, you don't have to write code. Action
patterns and policies are defined within an xml file. Allowing for
fast and convenient modifications. Which is especially cool when
debugging. So let us take a look at how to put those things together.

\subsection{The Policy File}

A policy file can have the following tags and attributes:

$<$!--MiroPolicyDocument--$>$

\begin{description}
  \item[policy]
    One file allways describes one entire policy. A policy defines a
    state machine that consists of a set of action patterns. Each
    action pattern describing on state of the machine.
  
  \begin{description}
  \item[actionpattern]
    An action pattern describes a set of behaviours, that can run
    simultaniously and produce some emergent higher level behaviour.
    

    Attributes:
    \begin{itemize}
    \item \textit{name} The name of the action pattern.  
    \item \textit{start (true/false)} The action pattern to be active
      at startup is to be marked true, the others false or unmarked.
    \item \textit{x, y} For GUI purpose, just ignore them.
    \end{itemize}
  
    \begin{description}
    \item[behaviour]
      That's what all the chapter is about. Read it again.

      Attributes: 
      \begin{itemize}
      \item \textit{name} Name of the behaviour.
      \end{itemize}  

      \begin{description}
      \item[parameter]
        Behaviours parameters can be specified within the policy file
        for each action pattern. 

        Attributes: 
        \begin{itemize}
        \item \textit{name} Name of the parameter.  
        \item \textit{value} The value of the parameter.
        \end{itemize}
      
      \end{description}  

    \item[transition]
      Message that triggers a transtition to another
      action pattern.  

      Attributes: 
      \begin{itemize}
      \item \textit{target} The action pattern to activate next.
      \end{itemize}  

    \item[arbiter]
      To decide which action to choose from the different outputs of
      the different behaviours one needs to arbitrate one way or the other.

      Attributes: 
      \begin{itemize}
      \item \textit{name} The name of the arbiter to use.
      \end{itemize}
    \end{description}
  \end{description}
\end{description}

\subsection{The Repositories}

A policy can be built on the basis of an xml description. For the
parser of this description to be able to construct the policy, it has
to be able to refer to instances of behaviours and arbiters that are
mentioned within the xml file by their name.  For this purpose a {\tt
  Miro::BehaviourRepository} and a {\tt Miro::ArbiterRepository} class
do exist.  At these Repositories an instance of each behaviour and
arbiter has to be registered. The name of the behaviour has to be
reported by the {\tt behaviourName()} method.  Since we anly need one
instance of each of these repositories, there do exist a global
instance of eacht. A pointer to such an instance can be obtained by
the classes static method {\tt instance()}.

\subsection{A Behaviour Engine}

All the initialization stuff necessary before constructing an instance
of the Policy class can be done within the main function of your
program. This task consists mostly of obtaining the needed object
references, instancing the behaviours and arbiters, instancing
behaviours and arbiters and registering them at their respective
repositories.  This can be bundled within a so called behaviour engine
class as can be seen in the behaviour example at:

{\tt \$MIRO\_ROOT/examples/b21Behaviours/behaviourEngine.cc|hh}

\section{The Policy Editor}

Editing large xml files is tedious and error prone. Therefore \miro
offers an GUI based editor, with which you can build and edit your
policies. You can also edit the parameters defined within the
Parameters classes of your behaviours.

\subsection{Describing the Available Behaviours}

To be able to use your own behaviours within the PolicyEditor, you
have to describe their properties (name and parameters) within another
xml file, witch looks like this:

$<$!--MiroParametersConfigDocument--$>$

\begin{description}
  \item[config]
    A parameters config file describes the types, names and defaults
    for parameter classes as used within the \miro framework. 

  \begin{description}
  \item[config\_group]
    Configurations can be grouped together for various reasons. 

    Attributes:
    \begin{itemize}
    \item \textit{name} The name of the config group. Within the
      behaviour parameters framework the valid names are \textbf{behaviour} and
      \textbf{arbiter}. Within each group only config items of the
      specified type are allowed to occure.
    \end{itemize}

    \begin{description}
    \item[config\_item] A configurable item to built a parameters
      class for.

      Attributes:
      \begin{itemize}
      \item \textit{name} The name of the behaviour or the arbiter.
      \item \textit{parent} The name of the super class.
      \item \textit{namespace} The namespace, the class resides.
      \end{itemize}
  
      \begin{description}
      \item[config\_parameter] A parameter for the class.

        Attributes:
        \begin{itemize}
        \item \textit{name} Guess what.
        \item \textit{type} int, double, bool, std::string, etc.
        \item \textit{default} default value (set in the constructor)
        \item \textit{measure} Possible types: mm, msec, �, mm/s, �/s.
          Those will be used for input checking by the GUI interface.
        \item \textit{inherited true/false}
          This attribute is used to 
          overwrite defaults for inherited variables within the
          constructor of the class. If set to true, no new member
          variable will be added to the class, and only the
          attributes name and default will be evaluated.
        \end{itemize}

        text tag: some description for debug output or bubble help
      \end{description}
    \end{description}
  \end{description}
\end{description}


\subsection{Auto-generating Parameter Class Code}
\label{sec:makeParams}

Having written the description of your behaviours parameters within
xml, it is possible to auto-generate the necessary code for your
behaviours associated Parameters class. This is done by the tool {\tt
  makeParams} which will be soon available within the \miro bin
directory.


%%% Local Variables: 
%%% mode: latex
%%% TeX-master: "miro_manual"
%%% End: 

\input{writing_a_server}

%------------------------------------------------------------------------------


\bibliography{miro_manual}
\printindex

\end{document}
