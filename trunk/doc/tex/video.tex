\chapter{Video Image Processing}

Video image acquisition is one of the most common sensory
devices used in robotics. Therefore \miro offers a common
{\tt VideoService} that can be adapted by the configuration file to any
supported video device.

Also, image processing is computationally expensive and many standard
filters exist in literature as well as some high performance
implementations such as \cite{IPP}. To facilitate the unified usage
within robotics scenarios, the video service provides a filter tree
framework.

Real-time image streams need a lot of bandwidth. This is usually
higher than the bandwidth available on most mainstream network
devices. Therefore the {\tt Video}-interface of \miro also provides
methods for usage of shared memory for sharing of images with its
clients on the same machine. Those methods are also designed for
minimizing copying overhead.

This section is organized as follows. First, the supported image
devices along with their bugs and features are discussed. Then the
filter tree framework is presented. First from an end user
perspective and second from a developer perspective. In section
\ref{SEC:VIDEO_INTERFACE} the video interface itself along with its
two use cases and helper classes is discussed.

\section{Video Device Access}

The VideoService currently supports the following devices for images
acquisition:

\subsection{Bttv Frame Grabbers}

These frame grabber cards are supported via video for linux \cite{}.
This is the standard way of connecting standard analog video cameras
to the computer.

Note, that the default number of frames used by the kernel driver for
video capture are two. In order to get the full frame rate (25/30Hz),
this has to be set to 4 buffers. This can be specified as a module
parameter.

\subsection{Firewire Digital Cameras}

\miro supports the fire wire digital camera protocol using
libraw1394\cite{} and libdc1394 \cite{}.
  
Note that most cameras 1394 controllers do not support being bus
master for dma transfers. Therefore the 1394 controller of the
computer has to be the highest numbered node. This can be configured
by a module parameter. Otherwise, unplugging the firewire cable from
the camera for some seconds helps.

\subsection{Matrox Meteor Frame Grabbers}

These rather old frame grabber cards are also supported by \miro.
Kernel drivers can be found at \cite{}. This device however, is mostly
unmaintained within \miro.

\section{Video Filter Trees}

For image processing many standard filters do exist that usually
perform a function $M_{n+1} = f_i(M_{n})$. Where $M_n$ denotes the input
image (matrix), $M_{n+1}$ the output image and $f_i$ is the actual
filtering function. 

Often filter chains are used. I.e. first undistorting
the image $f_0$ and then performing some color indexing $f_2$. So the
filtering function producing the output image $M_o$ from the input
image $M_i$ could be expressed as:  $$M_o = f_2(f_1(M_i))$$

\begin{figure}[!ht]
  \begin{center}
    \includegraphics[width=8cm]{filterchain}
  \end{center}
\end{figure}

More complex vision processing could also include the computation of
an edge image $M_e$ in addition to the color indexing above. This can
be done i.e. by producing a grey image $f_3$ and applying a canny
filter $f_4$, leading to the second filter chain: 
$$M_e = f_4(f_3(f_1(M_i)))$$

To avoid computing the undistorted image twice, a filter tree should
be used instead of two separate filter chains.

\begin{figure}[!ht]
  \begin{center}
    \includegraphics[width=8cm]{filtertree}
  \end{center}
\end{figure}

As an additional complication, many sophisticated filters use more
then one input filter. For instance the Canny filter computes first
two images of the Zobel operator (one for the x axis $z_x$ and one for the y
axis $z_y$). So the above described filter tree is actually an acyclic
directed graph of filters.

\begin{figure}[!ht]
  \begin{center}
    \includegraphics[width=8cm]{filtergraph}
  \end{center}
\end{figure}

\section{VideoService}

The program provided by \miro for image acquisition and preprocessing
is called {\tt VideoService}.

\miro comes with a set of basic filters. The term filter is used here
in a very general way, as almost every processing unit of the video
service is designed as a filter. Actually the image acquisition from
the different video devices is implemented as a filter, forming the
first set of filters available in \miro. The second set of filters are
basic image conversions, like byte order swapping, YUV to RGB
transformations etc. The third set of filters are the classic image
filters. Unluckily, there do not exist any of them currently in \miro.
It introduces further library dependencies for the middleware, which
we have to be careful about. And i.e. for Ipp based filters, we also
have a licensing problem. How to extend the video service to process
your own filters is covered in the following subsection.

Filter graphs and their individual parameters can be specified within
the configuration file of the robot. The VideoService actually does
not support every possible directed acyclic graph. The implications on
control flow and synchronisation are too cumbersome. Instead the
filters in the VideoService are organised as follows.

The filters are organised as a filter tree. That is, every filter has
one predecessor and $n$ successors. The filtertree is processed in
depth first order. This scheme is extended by so called filter
links: A filter can specify additional input filters. To guaranty the
correct order processing the filters, those additional filters have to
reside before the specifying filter, in terms of the depth first
evaluation scheme of the filter tree. Figure \ref{fig:legalilligal}
tries to visualize this constrain.

\subsection{VideoService Parameters}

The video service expects its configuration parameters with the
section Video of the configuration file. Its parameter is
also called Video. The
video service has the following parameters:

\begin{description}
\item[Width] The width of the input image (unsigned int).
  It has to be supported directly by the used video device.
\item[Height] The height of the input image (unsigned int).
  It has to be supported directly by the used video device.
\item[Palette] The palette of the input image (string).
  It has to be supported directly by the used video device.
  The recognized palettes of the video service are:
  \begin{description}
  \item[grey] 8 bit grey values.
  \item[grey16] 16 bit grey values.
  \item[rgb] 24 bit RGB format.
  \item[bgr] 24 bit BGR format (byte swapping).
  \item[rgba] 32 bit RGB format (plus alpha channel).
  \item[abgr] 32 bit BGR format (plus alpha channel).
  \item[yuv] 24 bit YUV format.
  \item[yuv411] compressed YUV format from 1394.
  \item[yuv422] compressed YUV format from 1394.
  \end{description}
\item[Filter] The root of the filter tree. The filter parameter
  has three entries:
  \begin{description}
  \item[Type] The filter type name (string).
  \item[Name] The name of the corresponding configuration file
    parameter entry (string).
    \item[Successor] The list of successor filters (Filter).
    \item[BackLink] The list of additional predecessor filters,
      specified by name (string). Note that this is the name of the
      filter instance, not the name of the filters video interface.
  \end{description}
\end{description}

\subsection{Video Filter Parameters}

For each filter in the filter tree, filter parameters can be given
within the parameter whose name matches the filter name given in the
filter tree. Those are also located within the section Video.

The base parameter set of each filter is as follows:

\begin{description}
\item[Name] The name of the filter (string).
  Somehow redundant. --- Will have to check.
\item[InterfaceInstance] Whether the filter has an instance of the
  video interface associated with it (bool).
  Each filter can have an instance of the video interface. That way
  every single node of the filter tree can be passed on to the client.
\item[Interface] The parameters of the interface. For each interface
  the following parameters can be specified:
  \begin{description}
  \item[Name] The name under which the interface is registered at the
    naming service (string). The default is Video. Note that if
    you have more than one Video object, you have to specify a unique
    name for each instance.
  \end{description}
\item[Buffers] The number of memory buffers, reserved by the buffer
  (unsigned int). The default is 4. While a client is accessing an
  image, the video service guarantees, that it does not become
  overridden with a new image from the video stream. If clients hold
  multiple images (i.e. for temporal integration) it has to be made
  sure, that there is still a buffer left, to put the next image from
  the video stream into.
\end{description}

\subsection{Video Device Parameters}

As mentioned above, all supported video devices are implemented as
filters. Therfore they inherit the base parameters. Note however, that
the device filters are not capable of having an interface
instance. Sharing memory mapped files is a bit tricky, and we didn't
want to go into that, yet. The common DeviceFilter parameters are:

\begin{description}
\item[Device] The fully qualified path of the video device
  (string). The individual device filters set this parameter to
  a common default (like /dev/video/0 for bttv and /dev/video1394/0 for
  firewire).
\end{description}

The available video device filters are (listed by type name):

\begin{description}
\item[VideoDeviceBttv] The bttv device filter defines the following
  additional parameter:
  \begin{description}
  \item[Subfield] The subfield chosen (string). As the PAL and
    NTSC images are interleaved, the bttv frame grabbers can select,
    which of the half images to scan, if the height of the output
    image is half the height of the full image (384). Possible values
    are odd, even and all. Note however, that many frame grabber cards
    do not support this feature. In this case, use the FilterHalfImage
    described below.
  \end{description}
\item[VideoDevice1394] The firewire digital camera standard gives
  access to many of the internal camera parameters. These can be
  configured in the parameter section of the 1394 device. For all
  these values -1 denotes, that this parameter should be controlled by
  the camera.
  \begin{description}
  \item[Buffers] The number of image buffers used for dma transfers
    (unsigned int).
  \item[Brightness] (int).
  \item[Exposure] (int).
  \item[Focus] (int). Default is auto.
  \item[Framerate] (int). Default is 30Hz.
  \item[Gain] (int).
  \item[Gamma] (int).
  \item[Hue] (int).
  \item[Iris] (int).
  \item[Saturation] (int).
  \item[Sharpness] (int).
  \item[Shutter] (int).
  \item[Temperature] (int).
  \item[Trigger] (int).
  \item[WhiteBalance0] (int).
  \item[WhiteBalance1] (int).
  \end{description}
\item[VideDeviceMeteor] The matrox meteor device filter does not
  define any additional parameters.
\item[VideoDeviceDummy] A dummy device which reads an image specified
  by the {\em Device} parameter. Useful for offline testing.
\end{description}

\subsection{Basic Video Filters and Their Parameters}

\miro also provided the following basic filters with the standard
VideoService.

\begin{description}
\item[FilterCopy] As the video device filters cannot map the image
  data directly into shared memory for access by the clients, this
  filter is provided. It copies the image internally, to allow for an
  interface instance.
\item[FilterSwap3] Byte swapping for 24 bits per pixel images (BGR to RGB).
\item[FilterSwap4] Byte swapping for 32 bits per pixel images (ABRG to
  RGBA).
\item[FilterFlip] We have a camera that's mounted upside down. This
  filter flips the image. Also useful if your fly imitating robot has
  successfully landed on the ceiling.
\item[FilterHalfImage] This is a filter to extract a half image from an
  interlaced image as the bttv frame grabbers provide. It copies each
  second scan line. It defines the
  following additional parameter:
  \begin{description}
  \item[Odd] Take the second half image by starting with the second
    line (bool). Default is false.
  \end{description}
\end{description}

\subsection{Configuration Example}

Putting it all together the configuration section for the {\tt
  VideoService} might look like the following. This is actually the
configuration of our Performance PeopleBot. It has an analog video
camera and a bttv frame grabber card. The camera is mounted upside
down.

\begin{verbatim}
<!-- The video configuration section. -->
<section name="Video" >

 <!-- Parameter section of the VideoService -->
 <parameter name="Video" >

  <parameter value="bgr" name="Palette" />         <!-- Input image palette. -->
  <parameter value="384" name="Width" />           <!-- Input image width. -->
  <parameter value="288" name="Height" />          <!-- Input image height. -->
  <parameter name="Filter">                        <!-- Filter tree root. -->
   <parameter value="DeviceBTTV" name="Type" />    <!-- It's a bttv device. -->
   <parameter value="DeviceBTTV" name="Name" />    <!-- Params section name. -->   
   <parameter name="Successor" >                   <!-- Filter tree leafs. -->
    <parameter name="Filter" >
     <parameter value="FilterSwap3" name="Type" /> <!-- Byte swapping filter -->
     <parameter value="FilterSwap3" name="Name" />
     <parameter name="Filter" >                    <!-- Filter tree leafs. -->
      <parameter value="FilterFlip" name="Type" /> <!-- Upside down filter. -->
      <parameter value="FilterFlip" name="Name" />
     </parameter> 
    </parameter>
   </parameter>
  </parameter>
 </parameter>

 <!-- Parameter section of the bttv device. -->
 <parameter name="DeviceBTTV">                   
  <parameter value="/dev/video0" name="Device" /> <!-- Path of the device. -->
 </parameter>

 <!-- Parameter section of the byte swapping filter. -->
 <parameter name="FilterSwap3">
  <parameter value="true" name="InterfaceInstance" /> <!-- Video interface. -->
 </parameter>

 <!-- Parameter section of the upside down filter. -->
 <parameter name="FilterFlip">
  <parameter value="true" name="InterfaceInstance" /> <!-- Video interface. -->
  <parameter name="Interface">                        <!-- Interface params. -->
   <parameter value="Flipped" name="Name" />          <!-- Interface name. -->
  </parameter>
 </parameter> 

</section>
\end{verbatim}

\section{QtVideo}

{\tt QtVideo} is a test client for the {\tt VideoService} and the {\tt
  Video} interface. It displays an image stream from a {\tt Video}
interface instance and has an additional button, to save snapshots to
disk.

The {\tt QtVideo} tool accepts the following command line parameters:
\begin{description}
\item[-n] Name of the {\tt Video} interface instance within the CORBA
  naming service. Default is {\em Video}.
\item[-r] Remote access of the images. QtVideo will be using the
  methods for location transparent image access of the {\tt Video}
  interface. These are much slower, than the methods using shared
  memory buffers, but are the only option when running QtVideo on
  another machine.
\item[-v] Verbose mode.
\item[-?] Emitting command line help and exits.
\end{description}

\section{Video Interface}

The {\tt Video} interface is used to manage the access of image data
by client programs. It supports location transparent as well as
optimized local image access. Additionally connection management is
used, to switch off filter subtrees that are not accessed by any other
program.

The general access pattern of a client of a {\tt Video} object looks
like follows:
\begin{enumerate}
\item Get a {\tt Video} interface IOR from the naming service.
\item Connect to the {\tt Video} object.
\item Get images until done.
\item Disconnect from the {\tt Video} object.
\end{enumerate}

Note that due to the performance centric design of the Video
interface, the VideoService can easily be jammed by client programs
violating the access protocol of the interface. To facilitate the
correct usage, some helper classes are provided by \miro. These are
discussed in section \ref{SEC:C++HelperClasses}.

\subsection{Location Transparent Image Access}

To access an image as a client running on a different computer than
the {\tt VideoService}, the {\tt Video} interface offers the methods
\begin{itemize}
\item {\tt exportSubImage}
\item {\tt exportWaitSubImage}
\end{itemize}
They both return a copy of the image as return value. The first method
returns immediately the current image, while the second one waits until
a new image becomes available before returning. The image will be
scaled down to the size specified by method parameters.

Note, that due to the copying and network overhead of those methods,
they are only useful for debugging and monitoring purposes.

\subsection{Local Image Access}

For clients running on the same machine as the {\tt VideoService}, the
{\tt Video} interface offers the following methods for image access
via shared memory buffers:
\begin{itemize}
\item {\tt acquireCurrentImage}
\item {\tt acquireNextImage}
\item {\tt releaseImage}
\end{itemize}
While the first method returns immediately the buffer of the current
image, the second one waits until a new image becomes available before
returning. Note that the clients have to release each image buffer
after processing. Otherwise the VideoService will soon run out of
buffers, to share new images with its clients.

\subsection{C++ Helper Classes}

To facilitate the usage of the {\tt Video} interface and the adherence
of the connect/disconnect, acquire/release protocol, \miro provides
two simple helper classes. They are defined in the file {\tt
  \$(MIRO\_ROO)/src/miro/VideoHelper.h}

\begin{description}
\item[Miro::VideoConnection] This class connects to a {\tt Video}
  interface instance on construction and disconnects on destruction.
  The constructor takes a pointer to the {\tt Video} object, to
  connect to as argument.
\item[Miro::VideoAcquireImage] This class acquires an image from a
  {\tt Video} interface instance on construction and releases it on
  destruction. The constructor takes a reference to a {\tt
    Miro::VideoConnection} object, as first argument, the second
  selects whether the current or the next image is to be acquired.
\end{description}

\subsection{Example Video Client}

\lstinputlisting[frame=tb, first=12, caption={examples/video/VideoExample.cpp}]{VideoExample.cpp}
\label{lst:VideoExample}

\section{Video Broker Interface}

In addition to the Video interfaces of individual filters, the
VideoService also offers an interface that accesses the whole filter
tree of the service. It is called VideoBroker and also registers as
VideoBroker at the NamingService. It offers methods for synchronised
simultanious access to multiple filters and for inspection of the
filter tree.

\subsection{Synchronised Image Access}

The client side evaluation of the results of the various filters often
requires access to multiple filters, that originate from the same
input image. The following methods are available in the VideoBroker
interface for this purpose. They work quite like the corresponding
methods of the Video interface of the  individual filters.

\begin{itemize}
\item {\tt acquireNextImageSet}
\item {\tt releaseImageSet}
\end{itemize}

\subsection{Filter Tree Meta Information}

Statistical information about the filter tree can be optained at the
VideoBroker interface via the method:

\begin{itemize}
\item {\tt filterTreeStats}
\end{itemize}

It returns the number of connections, the processing time of the
filter as well as the processing time of the corresponding filter
subtree (not including the processing time of the filter), the name of
the filter, the ior and the name of its Video interface (if
available), the filter successors as well as the successor links.

\section{Writing Filters}

Sooner or later, it will become handy for a user to add 
custom filters to the video image processing framework. This is a two
step process. First the filter has to be implemented as a class
derived from the {\tt Filter} base class. Second a customized version
of the {\tt VideoService} has to be built. The later has two
reasons. The first is, that \miro currently does not contain a plug in
architecture for dynamically loadable modules. The second is, that
dynamic loadable modules require the usage of shared libraries. As
this is has some runtime performance impact, that might not really
desirable for performance critical tasks like image
processing. Nevertheless, patches are welcome.

\subsection{The Filter Base Class}

The base class of all filters is {\tt Video::Filter}. It provides
methods for filter tree setup and configuration, a harness for
recursive filter tree processing and buffer management in interaction
with a {\tt Video} interface instance.

The most important protected methods and data members for child classes are.
\begin{description}
\item[inputBuffer] Returns a pointer to the input buffer.
\item[inputFormat\_] Specifies the data format of the input buffer.
\item[outputBuffer] Returns a pointer to the output buffer.
\item[outputFormat\_] Specifies the data format of the output buffer.
\end{description}

\subsection{Methods to Overwrite}

The most important methods for child classes are.
\begin{description}
\item[Constructor] The constructor takes a struct of type {\tt
    Miro::ImageFormatIDL} as parameter. It describes the input format
  of the filter (hence the output format of the predecessor filter).
  The constructor of the {\tt Filter} base class copies this parameter
  into the two member variables {\tt inputFormat\_} and {\tt
    outputFormat\_}. The derived filter has to ensure, that the input
  format is a valid format for the filter and modify the member {\tt
    outputFormat\_}, to correctly describe the output format of the
  filter. If the input format is not valid a {\tt Miro::Exception} has
  to be thrown.
\item[process] The actual filtering method. Put your filter code here.
\end{description}

\subsection{Configuration and Parameter Processing}

If your filter needs additional parameters for configuration, your can
add them to the configuration file framework by specifying them in a
parameter file description, deriving them from {\tt
  Miro::FilterParameters}. You have to overwrite the factory methods
for {\tt Miro::FilterParamters} in the {\tt Filter} class to return an
instance of your derived parameters class. This can be automated by
using the macros:
\begin{description}
\item[FILTER\_PARAMETERS\_FACTORY(X)] For usage within the class definition.
\item[FILTER\_PARAMETERS\_FACTORY\_IMPL(X)] For usage within the class implementation.
\end{description}
Where X is the name of your derived Filter class. 

For initialization and cleanup the following two methods exist.
\begin{description}
\item[init] Called on initialization of the filter tree. It provides a
  pointer to the parameters object initialized from the configuration
  file. Use {\tt dynamic\_cast} to convert it to the type of your
  derived parameter class.
\item[fini] Called before destruction of the filter tree.
\end{description}

\subsection{Enabling and disabling features}

The interface also contains some flags that allow the
derived classes to specify the existens of certain properties for the
base class. Overwrite their settings in the constructor. Those are namely:
\begin{description}
\item[interfaceAllowed] Specifies, whether the filter result can be
  exported by a video interface. The default for Video::Filter is
  true. For Video::Device it is false (see section
  \ref{sec:VideoDevice} for details).
\item[inplace] Not implemented yet. (Allways false)
\end{description}

\subsection{Filter Meta Information: FilterImageParameters}

Sophisicated filters may also extract additional information from the
images, such as a region of interest etc. To hand this information to
subsequent filters, an additional interface does exist. It is based on
the same mechanisms as the filter parameters, except that they can't
be specified within configuration files. The base class is empty.

The factory methods also reside in the corresponding {\tt Filter}
class and are wrapped by the two macros:
\begin{description}
\item[IMAGE\_PARAMETERS\_FACTORY(X)] For usage within the class definition.
\item[IMAGE\_PARAMETERS\_FACTORY\_IMPL(X)] For usage within the class implementation.
\end{description}
Where X is the name of your derived Filter class. The base class is
named {\tt FilterImageParameters} and the macros expect the name of
any child class to end on {\tt ImageParameters}.

The {\tt Filter} interface provides accessor methods for the
parameters of the individual buffers:
\begin{description}
\item[inputBufferParameters] Returns a const pointer to the parameters
  of the input buffer.
\item[inputBufferLinkParameters] Returns a const pointer to the
  parameters of the linked input buffer specified by its index.
\item[outputBufferParameters] Returns a pointer to the parameters of
  the output buffer. Note that the parameter instances is not
  initialized with defaults before processing of the filter. That is,
  the filters {\tt process} method is responsible, that all data
  members of the image parameters instance contain meaningfull values
  on return.
\end{description}


\subsection{Example Video Filters}

To illustrate the framework, a simple example filter is provided under
{\tt \$(MIRO\_ROOT)/examples/videoFilter}.

\subsection{Calculating a Gray Image}

{\tt FilterGray} implements a simple gray
filter for images in 24 bit rgb format. The gray value is calculated
as the average of the weighted sum of the three color values. The
weighting can be specified by the filters parameters.

Start the example service in the directory videoFilter with \\
{\tt GrayVideoService -MiroConfigFile GrayVideoService.xml}

It registers two interfaces at the naming service: {\tt Video} and
{\tt Gray}. To examine the original and the filtered image, use {\tt
  QtVideo} and {\tt QtVideo -n Gray}.

\subsection{Image time series}

{\tt FilterDiff} illustrates the use of multiple input filters. It
calculates the difference of two input images. For that purpose it
locks images of one image source to provide a view into the past. The
number of images locked can also be specified by the filters
parameters. Note however, that it has to be less then the number of
output buffers provided by the input buffer.

Start the example service in the directory videoFilter with \\
{\tt GrayVideoService -MiroConfigFile DiffVideoService.xml} \\
Note that the usage of an input filter link is a bit artificial in
this configuration, as both input filters are the same.

It registers two interfaces at the naming service: {\tt Video} and
{\tt DiffImage}. To examine the original and the filtered image, use {\tt
  QtVideo} and {\tt QtVideo -n DiffImage}.

\section{Writing a new Input Device}

The root of a filter tree is a video device. A video device is
actually not an image filter but an image producer. But as mentioned
above, it is implemented as as specialization of the video filter
class. Some additional requirements and features do exist to handle
this case.

\subsection{BufferManger}

The video framework does use a buffer manager class to organize the
concurrent acces to the images of a filter by the filters and remote
clients. Video devices normally aint capable of having directly a
video interaface instance attached. The video interface uses shared
memory to allow for zero copy acces to the clients. Video devices
often use memory mapped io to acquire the images. Combining the two is
at least tricky, if possible at all. So most video devices just skip
it.

Instead the buffer manager class is used for synchronising the access
to the mmaped images between the filters and the hardware.



%%% Local Variables: 
%%% mode: latex
%%% TeX-master: "miro_manual"
%%% End: 
