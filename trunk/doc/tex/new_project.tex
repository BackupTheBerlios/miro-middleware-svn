\chapter{Project setup with Automake and Autoconf}
\label{sec:new_project}

This chapter tries to give a short introduction on how to set up your
own project that uses \miro by the help of automake and autoconf.
This text was created using autoconf 2.52 and automake 1.5, so your
local installation may differ slithly.

First describe the the mode of operation of automake and autoconf is
explained by a real-world example from the \miro source
base. Afterwards the creation of the root configuration file of the
autoconf tool, {\tt configure.ac} is discussed. Afterwards the
different Makefile.ac templates, shipped with \miro are
introduced. After a short survey on the actual buid directives, the
chapter concludes with links to internet resources on the auto-tools.

\section{Introductory Example}

First a short introduction into the whole structure of automake and
autoconf, that uses a two-stage process.

\begin{itemize}
\item With automake you can (normaly) easily describe, how the files
within a folder should be processed. The commands for this are written
in a file called Makefile.am. As a minimal example look for example in
{\tt \$MIRO\_ROOT/\-src/\-can/\-Makefile.am}:

\begin{verbatim}
  lib_LIBRARIES = libcan.a
  libcan_a_SOURCES = Parameters.xml CanConnection.cpp CanMessage.cpp ...
\end{verbatim}

This generates a (static) library named {\tt libcan.a}, which is build
from the assigned sources {\tt Parameters.xml, CanConnection}...  All
other stuff within the Makefile.am is e.g.\ which files are installed
into which directory, if the user applies a {\em "`make install"'\/}
({\tt can\_include\_HEADERS}, note the prefix of the macro), or which
files are generated automatically during the build-process and can
therefor removed if a {\em "`make clean"'\/} is desired ({\tt
BUILT\_SOURCES} and {\tt CLEANFILES}). These Makefile.am are converted
into {\tt Makefile.ac} using the {\em automake\/} program.

\item {\em autoconf\/} generates the real {\tt Makefile} for the {\tt
Makefile.ac} files. Therefor it has to know the paths and options
necessary for the build process and which files, libraries and
programs it has to compile at all. Exactly that is, what the
configure-script tries to detect and guess automatically. Of course no
one wants to write a configure-script directly (take a look at it, it
looks really, really ugly. Moreover a program should compile on a
couple of different systems, each having their own quirks and
peculiarities). Therefor one writes a {\tt configure.ac} files, that
is finally converted into the configure-script. The whole autoconf
stuff is based on so called macros, from which are a lot of them
allready included:

\begin{verbatim}
  AC_CHECK_LIB(raw1394, raw1394_get_libversion, ac_have_libraw1394=yes, [ac_have_libraw1394=no])
  AC_CHECK_HEADERS(errno.h fcntl.h stdlib.h)
  AC_CHECK_PROG(ac_has_dvips, dvips, yes, no)
\end{verbatim}

There are e.g.\ macros to test the existence of libraries, header files
or executable programs. If you need a more complicated test, you can
write them with a language called M4 (M4's syntax is quite close to a
normal shell syntax). The results of all tests are write in a file
called config.h additionally. This file can be included into the
C-files and are treated as normal c-defines (e.g. {\tt \#define
MIRO\_HAS\_METEOR 1}).
\end{itemize}


\section{Create {\tt configure.ac}}

The easiest way to start with is using the autoscan program (from the
autoconf-package). This will create a file called configure.scan that
you should rename to configure.ac. ({\em autoscan\/} is also a good
tool to check your existing configure.ac for missing tests). Now look
into your new configure.ac file. The first line ({\tt AC\_INIT})
describe your project, so fill it with useful values, a name for your
package, your version and an email address for bug-reports. The next
two lines ({\tt AC\_CONFIG\_SCRDIR}, {\tt AC\_CONFIG\_HEADER}) should be
fine.

The remainder of the file is group into different parts. First part is
the check for programs. {\em autoconf\/} offers several ready to use checks
for this, like e.g. {\tt AC\_PROG\_CC} that checks several behaviours of
your installed C-compiler or {\tt AC\_PROG\_LN\_S }that looks how to
create a symlink. The automatically included macros should be fine.

The next paragraph checks for libraries. Note, that the philosophy of
{\em autoconf\/} is to check for the possibility of creating and
compiling a program with a certain function using this library, not
the pure existence of this library. So look for a function that should
be within the different libraries and insert them in the second field
of the {\tt AC\_CHECK\_LIB} macros (the first name is the library name
itself). Possibly you have to remove lines for libraries, that are in
fact builded by your project itself (Note, that it may be difficult to
use C++ specific stuff here).

Third, header files are checked. Beside some prefabed macros for
standard headers (e.g. {\tt AC\_HEADER\_STC}) you can insert additional
files into the {\tt AC\_CHECK\_HEADERS} macros. Insert them as a space
separated list.

Next two paragraph contains checks for typedef and structures and
library functions. The automatically included macros should be fine,
otherwise include additional functions into the {\tt AC\_CHECK\_FUNCS}
macro. The functions here refer to standart C functionality, not to
mix up with special functions provided by additional libraries as
described above.

The last, long macro {\tt AC\_CONFIG\_FILES} contains a space separated
list of all the Makefiles the configure script has to produce. This
means, if you add a directory into your project later on, you have to
insert the path here.



\subsection{Tests for more complex packages (and facilities)}

Most checks for the functionallity of your system should be covered
with the above, autogenerated checks. But to build a project that
relies on other, more complex packages (in our case \miro, ACE, TAO,
Qt, etc), it is necessary to add additionally tests to the configure
script.

Additionally checks are written in a language called M4. It is up to
you, where to save these files, but I prefer to add an extra directory
named config, so the following explanation will assume this.

First you have to check for the existence of e.g. ACE, TAO and
Qt. Simply copy the necessary files {\tt search\_ace.m4},{\tt
search\_tao.m4} and so on from {\tt \$MIRO\_ROOT/\-config/} to your local
config directory and add {\tt AC\_SEARCH\_ACE}, {\tt AC\_SEARCH\_TAO} and
so on (the name of the function macros) to the {\tt configure.ac}
script (for example directly above the {\tt AC\_CONFIG\_FILES} macro.

To be sure to add all the necessary libraries and options that are
used to compile ACE/\-TAO, additional test are prepared within the file
{\tt feature\_ace.m4} (again copy this file into your local folder and
add the macro name into {\tt configure.ac}).

To check for Miro and parts of the functionality, we prepared two
files {\tt \$MIRO\_ROOT/\-templates/\-search\_miro.m4} and {\tt
\$MIRO\_ROOT/\-templates/\-features\_miro.m4}. Again use them the usual way,
e.g. insert {\tt AC\_FEATURES\_BTTV\_MIRO} to be sure that Miro is
compiled with support for {\em BTTV\/} frame grabber cards.

If you need to write your own tests and checks, please feel free to
use and modify these files. Additional macros that can be used out of
the box or as examples for own tests, look at:

\begin{verbatim}
http://wwww.gnu.org/software/ac-archive/
http://ac-archive.sourceforge.net/
\end{verbatim}

\section{Create {\tt Makefile.am} for directories, libraries and executables}

The next step is to prepare a Makefile.am within each directory that
contain files that need to be compiled. The easisest way is again to
use the examples within {\tt \$MIRO\_ROOT/\-templates/}. For example to
create a Makefile.am that should call other Makefiles in other
subdirs, take {\tt \$MIRO\_ROOT/\-templates/\-Makefile.am.dir}, rename it to
{\tt Makefile.am} and add the names of the subdirectories. There are also
examples for sources that should be compiled into a static library
(Makefile.am.lib.a) into a shared library (Makefile.am.lib.so) or
different executables (Makefile.am.bin). All examples contain some
comments to guide you, what to fill in.

Remeber to be sure, that each directory with a {\tt Makefile.am} is
listed in the {\tt AC\_CONFIG\_FILES} macro in the {\tt configure.ac}
script.


\subsection{Makefile.am.dir}
This Makefile starts subbuilds in the specified directories. It is an
example for conditional compilation (depending on the
configuration). If you don't have conditional branches, you can left
the second variable empty or reduce the Makefile.am to a single line,
like e.g. \texttt{SUBDIRS = firstDir secondDir}.

\subsection{Makefile.am.bin}
This Makefile builds a single or several binaries. The name of the
binaries to be built are specified in the variable {\tt bin\_PROGRAMS}.
There have to be an extra line for each binary to be build starting
with the name of the binary itself and an additional
\texttt{\_SOURCES}-suffix. The sources are specified with their suffix
(.cpp). Please also list all the header files here, otherwise they
don't get distributed by an call to \texttt{make dist}. If the
programs need to be linked against several libraries, there is again
one line per binary containing the name of the binary plus an
additional \texttt{\_LDADD}-suffix.

\subsection{Makefile.am.qt}
This is a special Makefile.am template for Qt programs. For
simplicity, the Makefile is designed for a single binary per
directory, but beside that, most of the above section also applies
here. The only exception is, that we have to distinguish between
normal source files (variable \texttt{sources}) and files (variable
\texttt{tomocsources}) that need to be compiled with Qt's Meta Object
Compile (\texttt{moc}). Be sure to \textbf{not} include the header
files into the \texttt{tomocsources} variable, but to the
\texttt{sources} variable, otherwise they get "cleaned".

\subsection{Makefile.am.lib.a}
This Makefile builds a static library. The name of the library can be
specified in the variable \texttt{lib\_LIBRARIES}, the files building
the library in the variable \texttt{sources} (with the .cpp suffix; or
.xml in case of an xml-file for \miro's parameter framework). Headers
that are not automatically considered because they don't have a
corresponding cpp-file should be specified in the variable
\texttt{extraheader}. Don't forget to name the prefix for the
\texttt{\_SOURCES}-macro with the name of the resulting library.
Therefore please substitued all occuring dots with underscores (e.g.\ 
the usual \texttt{.a} a the end of an static library).  Additionally,
replace the prefix for the macro \texttt{\_include\_HEADERS} with the
name of the directory, to which the header files should be installed
(see also the description of \texttt{Make-rules} below).

\subsection{Makefile.am.lib.so}
Makefiles for shared libraries look quite the same as for static
libraries, to most of the above applies here too. The only difference
from a users perspective is the name of the variable
\texttt{lib\_LTLIBRARIES} where you name your library (note the
\texttt{LT} within the variable (and the targets at the end of the
\texttt{Makefile.am}) which indicates that the library is build using
the libtool package).

\subsection{Conditionally Compiled Sub-projects}

If you programs or libraries need to be compiled differently depending
on configuration options, please have a look at an existing
\texttt{Makefile.am}, e.g.\ in
\texttt{\$MIRO\_ROOT/\-examples/\-behaviours/\-engine/} or
\texttt{\$MIRO\_ROOT/\-src/\-miro/}.

\subsection{Create other {\tt Makefile.am}}

Because auto-conf/\-make themself are quite unflexible if you intend to
do something, that the autotools are not designed to, you can also
insert an "`all-local"' target into the {\tt Makefile.am} that is
treated as a normal Makefile part. Although this has to be necessarry
sometimes, be careful with this, auto-conf/make may fail to calculate
the correct dependencies which in turn may lead to incomplete and
corrupt builds. Save places to write your own stuff are for example in
the documentation directory. See {\tt \$MIRO\_ROOT/\-doc/\-html} and {\tt
\$MIRO\_ROOT/\-doc/\-tex} for examples on building a doxygen or latex
documentation.

\subsection{Make-rules}
Last you may have a look at the \texttt{Make-rules} file within the
\texttt{\$MIRO\_ROOT/\-config/} directory. This file is included into each
\texttt{Makefile.am}, so rules or variables that are needed globally,
can be inserted here. If you want to define additional installation
subdirectories for the header files, you can see examples here.

\section{Build the beast}

To finally build the new project, you have to call the different
programs, which in turn convert the prepared files into a real,
make-based project.

Assuming that you have written appropriate {\tt Makefile.am} and {\tt
configure.ac} files, you should be able to build your project by
entering the following commands:

\begin{description}
\item[autoheader] Creates \texttt{config.h.in}.
\item[aclocal -I config] Adds \texttt{aclocal.m4} to directory.
  Defines some m4 macros used by the auto tools and the self-written
  tests from the config-directory.

\item[autoconf] Creates \texttt{configure} from \texttt{configure.ac}
\item[automake] Creates \texttt{Makefile.ac} from \texttt{Makefile.am} 
\item[./\-configure] Creates \texttt{Makefile} from \texttt{Makefile.in}
\item[make]
\end{description}

Just repeat the last 5 steps to completely rebuild the project. Most
projects have an \texttt{autogen.sh} script that runs everything up to
the configure step.


\section{Additional sources of help}

In case of problems, first have a look at the offical manuals of both,
automake and autoconf. You can find them on the GNU webpage:

\begin{verbatim}
http://www.gnu.org/software/automake/manual/automake.html
http://www.gnu.org/software/autoconf/manual/autoconf-2.57/autoconf.html
\end{verbatim}

There is also a book available under the terms of the Open Publication
License:

\begin{verbatim}
http://sources.redhat.com/autobook/download.html
\end{verbatim}

%%% Local Variables: 
%%% mode: latex
%%% TeX-master: "miro_manual"
%%% End: 
