\chapter{Event Channel logging}
\label{sec:ecLogging}

The acquisition of data during the run of a robot application from
various levels of sensor processing is an essential feature for
debugging, evaluation and performance assessment. The event based
communication paradigm of \miro is designed to distribute raw sensor
data as well as higher level system events, like the latest belief
state of the robot. The stream of events resembles therefore a quite
complete trace of the system state during the run of a robot program.
\miro provides functionalitiy to log such an event stream generically
to a file. The data can not only be reread into the system, but the
events can also be redistributed over the event channel, allowing for
detailed offline analysis. In \cite{Utz+Mayer+Kraetzschmar:04} the
various configuration and application scenarios are described along
with a detailed performance analysis of the facility. This chapter is
concerned with a more technical view, like the setup of the logging
client, the replay as well as the maintenance of logged data and the
file format.

\section{Logging Client Class}

\section{Parameters}

\section{Standalone Clients}

\subsection{Command Line Parameters}

\subsection{Event Channel Client}

\subsection{NMC Client}

\section{LogPlayer}

\section{File Format}

\section{Test Programs}

%%% Local Variables: 
%%% mode: latex
%%% TeX-master: "miro_manual"
%%% End: 
