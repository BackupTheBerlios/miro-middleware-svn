\chapter{Event Channel logging}
\label{sec:ecLogging}

The acquisition of data during the run of a robot application from
various levels of sensor processing is an essential feature for
debugging, evaluation and performance assessment. The event based
communication paradigm of \miro is designed to distribute raw sensor
data as well as higher level system events, like the latest belief
state of the robot. The stream of events resembles therefore a quite
complete trace of the system state during the run of a robot program.
\miro provides functionalitiy to log such an event stream generically
to a file. The data can not only be reread into the system, but the
events can also be redistributed over the event channel, allowing for
detailed offline analysis. In \cite{Utz+Mayer+Kraetzschmar:04} the
various configuration and application scenarios are described along
with a detailed performance analysis of the facility. This chapter is
concerned with a more technical view, like the setup of the logging
client, the replay as well as the maintenance of logged data and the
file format.

\section{Logging Client}

The logging client for the notification service ist called
\texttt{LogNotifyConsumer} and is located in the service library
\texttt{libmiroSvc}. It can be used as a normal event channel client
within the own application or as a standalone program.

\subsection{Parameters}

The location of the log files can be defined by the environment
variable \texttt{MIRO\_LOG}. The default file name is defined as:

\$(MIRO\_LOG)/<domain name>\_<time string>.log

where <domain name> is the name of the robot and <time string> is the
time the recording of the log file was started in the format
\emph{yyyy.mm.dd-hh.mm}. An alternative file name can be specified at
the command line or as constructor parameter respectively.

Other parameters can be specified in the config file. They are located
in the section Notification. Those are:
\begin{description}
\item[MaxFileSize] The maximum size of the log file. Log files can become
  quite large over time. So it is good to have an upper limit. Also,
  the implementation uses a memory mapped file for maximum throughput.
  So the log file, along with the application should fit into the
  available physical memory. Otherwise swapping will jeopardize the
  overall system performance. The default is 150 MB. This is
  sufficient for 20 minute runs of our soccer robots, logging almost
  everything that is happening.
\item[TCRFileSize] The log file contains a type code repository, that
  holds descriptions for all types, stored within the log file (CORBA
  type codes, to be exact). The default maximum size for the type code
  repository is 1 MB. As a the number of different payload types is
  usually limited for one robot (about a dozend) and the type code
  size is between 0.5 -- 1.5 KB, this should be sufficient for most
  applications.
\item[Event] A vector of domain name, type name pairs that shall be logged.
\item[TypeName] A vector of type names that shall be logged. The
  domain name is supposed to be the domain name of the robot. Usually
  a robot only logs its own events.
\end{description}

\section{Standalone Logging Client}

The LogNotify client is a standalone program, that can be used as an
ad hoc solution for logging data. It offers the possibility to connect
directly to a selected event channel, or use the IP-multicast based
event channel federation described in section \ref{sec:NMC}.

\subsection{Command Line Parameters}

The command line parameters are the following:

\section{LogPlayer}

The \texttt{LogPlayer} is a GUI-based application for the replay of
logged data. It offers timely replay, slow motion and single stepping
as well as simple maintenance operations like cutting and event
filtering. It can also load multiple log files and play them
synchronized, distributing the events over mutliple event channels. A
screenshot of the \texttt{LogPlayer} can be seen in figure
\ref{fig:logplayer}.

The main panel consits of five button, a digital clock, a slider bar
and a dail. The clock shows the time stamp of the coursor within the
file. The slider shows the relative positon of the cursor within the
log stream and can be used to repositon the cursor. The dail resembles
the replay speed. Twelve o'clock is the original timing.
Counterclockwise is slow motion and clockwise fast forward.

\subsection{Main Panel}

The buttons on the main panel do the following:
\begin{description}
\item[Stop] Stop the replay of the log file and reset the coursor to
  the beginning of the log stream.
\item[Play] Play the log file from the current cursor position in the
  speed selected on the dail.
\item[Pause] Pause replay. The cursor position remains unaffected.
\item[Forward] Single step one event forward.
\item[Backward] Single step one event backward.
\end{description}

\subsection{Menu}

The menu bar has five entries: File, Edit, Events, Tools and Settings.
The file menu offers file operations. The edit menu offers simple
cutting functionality. The events menu allows to filter certain events
from the log file. The tools menu holds tool windows for further log
file event inspection and the settings menu holds the configuration.

\subsubsection{File Menu}
\begin{description}
\item[Open/Close...] Displays the file set dialog that allows to add and
  remove log files, from the set.
\item[Save as...] Allows to save the current log file set, with event
  filtering and cutting applied to a new file.
\end{description}

\subsubsection{Edit Menu}
\begin{description}
\item[Cut Front] Removes everything from the beginning to the current
  cursor position from the log file.
\item[Cut Back] Removes everything from the current cursor position to
  the end of the log file.
\item[Undo all] Undo all logfile cutting.
\end{description}

\subsubsection{Events Menu}
The events menu holds an entry for every robot, that was found in the
currently loaded log file set. That is, for every distinct domain
name. For each domain name a submenu with this robots events in the event
stream is available. That is, every distinct type name. Each type name
entry is checkable and can be switched on and off. Each unchecked
event type is skipped on replay of the log file.

\subsubsection{Tools Menu}
This menu offers the event view, which is a window, listing a sequence
of events around the coursor position. This is uesefull for event
exact cut operations etc.

\subsubsection{Settings Menu}
The only setting currently available at the menu is the length of the
history of the event view. - More to come --- promised.


\section{File Format}

The file format consists of a header block of 8 bytes: The first four
represent the log format magic cooky: MLOG. The next two are the
version number in intel byte order. Followed by a two byte byte
order flag, as defined by the CORBA common data representation format.

\begin{lstlisting}
struct LogHeader
{
  unsigned long id; // MLOG: 0x474f4c4d
  unsigned short version; // Version 3: 0x003
  unsigned short byteOrder; //Host byte order
};
\end{lstlisting}

The current version of the log file format is 3. In basic, all
following data, including the events are stored in CORBA CDR stream
format. What follows is in general a variable length array (CORBA
sequence) of type:

\begin{lstlisting}
struct EventEntry;
{
  TimeT timeStamp;
  unsigned long eventSize;
  CosNotification::StructuredEvent event;
};

typedef sequence<EventEntry> EventArray;
\end{lstlisting}

The timeStamp field denotes the time, the event was received by the
logging consumer. It is used for the timely replay and synchronization
of log files. 

The eventSize field denotes the size of the event field within the log
file. This is used to speed up initial parsing of log files for the
\texttt{LogPlayer}. 

The event field contains the event, as delivered by the event channel.
As a footpring optimization, the type codes, stored with every
CORBA::Any, that is, remainder\_of\_body field of the events are stored as

\begin{lstlisting}
typedef sequence<TypeCode> TypeCodeArray;
\end{lstlisting}

at the end of the log file, forming the type code repository. The type
code in the remainder\_of\_body field is replaced by the index of the
type code within the type code array.

The offset of the type code sequence within the log file is stored at
the beginning of the CDR stream. That way the type code repository can
be read before parsing of the events. This defines the log file
format version 3, as follows:

\begin{lstlisting}
struct LogFile
{
  LogHeader header;
  unsigned long tcrOffset;
  EventArray events;
  TypeCodeArray tcr;
};
\end{lstlisting}

\section{Test and Example Programs}

The programs provided in the tests directory for the LogNotification
format, are in basic two configuration files, that restrict the size
of the log file and the type code repository, to provoke overflow of
the log files. 

A utility program for log files is LogTruncate. If a log client dies
discracefully with a segfault, the log is still saved and usable, but
has the size specified by the MaxFileSize parameter, regardless of the
actual footprint of the stored events. LogTruncate, removes the unused
0 bytes and truncates the log file to a reasonable size.

%%% Local Variables: 
%%% mode: latex
%%% TeX-master: "miro_manual"
%%% End: 
