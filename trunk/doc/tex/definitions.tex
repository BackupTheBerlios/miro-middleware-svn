\chapter{Definitions}


\section{Units}

When it comes to parameter passing, distances are specified in mm and
angles in radiant. Velocities are specified in mm/s and rad/s
respectively. 

There have been discussions recently to unify the data formats used in
the different robot programming environments, such as Player/Stage,
OROCOS, Orca, etc. \miro supports this idea and therefore we will most
probably switch to the general ``kilogram, second, meter'' system in
the near future. As long as not documented otherwise, the basic data
type for specifying these values will be a float value (IEEE floating
point value of 32 bit).

\section{Coordinates}

\miro uses the cartesian coordinate space. Angles are specified
mathematically, that is, counter-clock wise, using radiant.

%mm, rad


\begin{figure}[!ht]
  \begin{center}
    \includegraphics[width=8cm]{grid_coords}
  \end{center}
\end{figure}


%%% Local Variables: 
%%% mode: latex
%%% TeX-master: "miro_manual"
%%% End: 
