\chapter{Writing a \miro Service}

The programming of a \miro service is divided into two step:
Accessing the low level hardware device (such as the Can bus) and
writing the high level interfaces. Note that many hardware devices
incorporate multiple sensor modalities.


\section{High-Level Server Programming}

\begin{figure}[!ht]
  \includegraphics[width=\textwidth]{client_server}
  \caption{Client-Server Architecture Overview}
  \label{FIG_CLIENT_SERVER_ARCHITECTURE}
\end{figure}

In order to write a server you have to do the following steps:
\begin{enumerate}
  \item Describe your interface in IDL.
  \item Translate the IDL description to C++ files.
  \item Implement your methods.
\end{enumerate}


\subsection{Copy the \miro server template directory}

- Which files are in there


\subsection{Describing an interface in IDL}

- edit interface description (.idl) \\
- idl/Makefile 

\subsection{Translating the IDL description to C++}

- call TAO\_idl (in {\tt \$(TAO\_ROOT)/TAO\_IDL})

\subsection{Implementing your own methods}

- in ...Impl \\
- don't forget to document !!


\section{Low-Level Server Programming}

Low-Level Servers are Servers that talk to hardware devices. They
usually use a device driver that is compiled into the kernel or loaded 
as a module. They provide handling for special protocols which have to 
be followed and provide the services of the hardware device in an
abstracted way, so that they are useful.

Hardware devices in a UNIX system are represented by file
descriptors. There are two groups, character devices and block devices,
character devices are stream based and provide for sequential reading
and writing, while block based devices provide random access to block
addresses. At the moment all hardware devices used by miro use the
character device abstraction.

A useful \miro low-level server must provide two functionalities:
interaction on the CORBA level with (potential multiple) clients that
use the service concurrently and handling of all communication with
the device, which may be asynchronous to the requests. This requires a
multithreaded design. Fortunately the ACE framework provides design
patterns that fulfill these needs. For a schematic view of a \miro
low-level server see figure
\ref{FIG_CLIENT_SERVER_ARCHITECTURE_LOW_LEVEL}.


\subsection{The Device Framework}

\subsubsection{Connection}

\subsubsection{EventHandler}

\subsubsection{Consumer}

\subsection{The Configuration Framework}

\subsection{Parameter}

\subsection{XML parsing}

\subsection{The Reactor and Events}

The ACE reactor class provides an abstraction of the popular
{\tt select} method in common UNIX programming environments. It is
able to invoke an event handler when data is readable from the file
descriptor\footnote{it can also handle timer events and asynchronous
  writes, see \cite{ACE-Manual} for details}. It provides low
latency and the thread does not consume any CPU power while waiting
for data to come. To keep the responsiveness of this architectural
element high\footnote{which is essential to avoid flooded input
  queues}, only few computation should be done inside of the event
handler, usually this event handler only assembles full packets and
passes these on to a separate task which accounts for higher level
processing. It is important to note that the reactor and the event
handler share a single thread of execution, so don't expect to get
feedback from the reactor within the event handler. 


\subsection{Using Tasks}

Tasks are the second important design pattern provided by ACE that we use.
A task is an assembly of one or more threads, that share a common
message queue, which can be used to pass requests to the threads. We
usually use a task for handling of complete packets assembled by the
event handler, which puts them on the message queue. According to the
type of the packet the tasks takes appropriate action. This usually
consists in retrieving data from the packet and signaling it on a
condition variable. Condition variables are the main mechanism of the
(CORBA) server implementation methods to wait for data requests or
events that are provided by the hardware. The ACE task class provides
an easy scaling for multiple working threads that concurrently work on 
the different requests.

\begin{figure}[!ht]
  \includegraphics[width=\textwidth]{client_server_low_level}
  \caption{Client-Server Architecture}
  \label{FIG_CLIENT_SERVER_ARCHITECTURE_LOW_LEVEL}
\end{figure}


\subsection{Thread/Task Synchronization}

\subsection{ACE logging}

\subsection{A Simple Example}

%%% Local Variables: 
%%% mode: latex
%%% TeX-master: "miro_manual.tex"
%%% End: 


